\section{Conclusion}
This study has covered the analysis of a simple system to determine the optimal currents that ought to be injected to raise the positive sequence voltage and decrease the negative sequence voltage in case of a fault. Four types of faults have been considered. Despite the particularities, since the system is mainly inductive, the optimal active currents tend to be close to zero, while the optimal reactive currents become substantial. As a direct consequence, we conclude by saying that injecting only reactive powers, as imposed by the grid codes, is likely to be a convenient strategy for systems where lines are highly inductive. In all cases (constant impedance, changing $R/X$ ratio and changing $X_c$) the objective function related to the ROPT situation has been close to the OPT one. However, it is mandatory to determine the type of fault in order to properly deduce the optimal values of the injected currents.

In addition to that, this work proposes two methodologies to arrive to the optimal solution. One has consisted of computing combinations where currents can take a wide range of values. When the intervals are small enough, such computationally intensive approach matches with the solution coming from solving directly the optimization problem. The results indicate that the double line to ground fault is the hardest in terms of minimizing the objective function, while the line to ground fault can present multiple minimums. 

Varying the $R/X$ ratio shows that the optimal currents can vary considerably. However, the objective functions do not experience substantial differences, as the OPT case is always slightly better of than the ROPT one. The same conclusion can be extracted from the submarine cable under study. In this last situation, it seems that the injection of real currents may dominate over the imaginary currents. 

\newpage
\printbibliography
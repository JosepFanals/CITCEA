\section{Submarine cable}
Having seen the influence of varying the ratio between the resistive and inductive part of the Thevenin impedance, we can now go a step further by including a hypothetical submarine cable between the grid and the point where the fault occurs. The systems takes the form shown in Figure \ref{fig:sys_pc}.

\begin{figure}[!htb] \centering
\begin{circuitikz}[european]
\thicklines

\draw (3,0) to [sV, v_=$\underline{V}_{th}$] (3,2);
\draw (-0,2) to [R, l=$\underline{Z}_{th}$] (3,2);
\draw (0,2) to [R, l_=$\underline{Z}_s$] (-2,2);
\draw (0.25,2) to [R, l_=$\underline{Z}_p$] (0.25,0);
\draw (-2.25,2) to [R, l=$\underline{Z}_p$] (-2.25,0);
\draw (-2,2) to [short] (-3,2);

\draw (3.25,0) to [short] (2.75,0);
\draw (0,0) to [short] (0.5,0);
\draw (-2.5,0) to [short] (-2.0,0);

\draw (-3.25,0) to [short] (-2.75,0);
\node at (-6,1.3) {PCC};
\node at (-6,2.7) {$\underline{V}_c$};
\draw (-3,2) to [short] (-3.2,1);
\draw (-3.2,1) to [short] (-2.8,1);
\draw[-{Latex[length=3mm]}] (-2.8,1) to [short] (-3,0);
\draw (-6,2) to [R, l=$\underline{Z}_a$] (-3,2);
\draw[line width=0.65mm] (-6,2.5) to [short] (-6,1.5);
\draw[line width=0.65mm] (-3,2.5) to [short] (-3,1.5);
\draw (-9,2) to [R, l=$\underline{Z}_c$, i=$\underline{I}$] (-6,2);
\draw (-10.0,2) to [sdcac] (-9.0,2);


\end{circuitikz}
\caption{Single-phase representation of the simple system under a fault with a submarine cable}
\label{fig:sys_pc}
\end{figure}
This system can be studied without added complexity once a Thevenin equivalent is formed on the right hand side of the fault. Opting for this will imply that we will be able to recycle the expressions developed before. The new Thevenin voltage and impedance, denoted by $\underline{V}'_{th}$ and $\underline{Z}'_{th}$, are given by:
\begin{equation}
    \begin{cases}
        \underline{V}'_{th} = \dfrac{\underline{Z}_p\underline{Z_p}}{2\underline{Z}_{th}\underline{Z}_p + \underline{Z}_p\underline{Z}_s + \underline{Z}_p\underline{Z}_p + \underline{Z}_{th}\underline{Z}_s}\underline{V}_{th},\\[18pt]
        \underline{Z}'_{th} = \dfrac{\underline{Z}_p\underline{Z}_p\underline{Z}_{th} + \underline{Z}_s\underline{Z}_p\underline{Z}_p + \underline{Z}_{th}\underline{Z}_{p} + \underline{Z}_{th}\underline{Z}_s\underline{Z}_p}{2\underline{Z}_{th}\underline{Z}_p + \underline{Z}_p\underline{Z}_s + \underline{Z}_p\underline{Z}_p + \underline{Z}_{th}\underline{Z}_s}.
    \end{cases}
\end{equation}
The distance of the cable is going to take multiple values in order to weight its influence. Realistic data for the $\underline{Z}_s$ and $\underline{Z}_p$ impedances are taken from \cite{cheah2017offshore} and adapted to per unit values. They are shown in Table \ref{tab:imp_cable}.
\begin{table}[!htb]\centering
    \begin{tabular}{ccc}
        \hline
        Magnitude & Value & Units\\
        \hline
        $\underline{Z}_s$ & $6.674\cdot 10^{-5} + j2.597\cdot 10^{-4}$ & pu/km\\
        $\underline{Z}_p$ & $-j77.372$ & pu$\cdot$km\\
        \hline
    \end{tabular}
    \caption{Impedances values to be used in the submarine cable analysis}
    \label{tab:imp_cable}
\end{table}

Figure \ref{fig:dist_th} displays the resulting Thevenin voltage and impedance for a varying cable distance.
\begin{figure}[!htb]\centering \footnotesize
    \begin{subfigure}[!htb]{.4\textwidth}
      \centering
          \begin{tikzpicture}[trim axis right,trim axis left]
              \pgfplotsset{width=7cm, height=6.0cm}
              \begin{axis}[grid=major, xlabel={Distance (km)}, ylabel={Voltage (pu)}, /pgf/number format/.cd, legend style={at={(0.95,0.35)},anchor=south east,legend columns=1, draw=none, inner sep=0pt,fill=gray!10}, xtick={0,20,...,100}, ytick={0.0,0.2,...,1.2}, axis line style = very thick, yticklabel style={/pgf/number format/fixed, /pgf/number format/precision=2}, xmin=0, xmax=100]
              \addplot[very thick, black] table[x=x, y=y, col sep=comma] {Data/cable/Vth_re.csv};
              \addplot[very thick, gray] table[x=x, y=y, col sep=comma] {Data/cable/Vth_im.csv};
               \legend{$\Re (\underline{V}'_{th})$, $\Im (\underline{V}'_{th})$};
              \end{axis}
          \end{tikzpicture}
    \end{subfigure}
    \hspace{1.5cm}
    \begin{subfigure}[!htb]{.4\textwidth}
        \centering
            \begin{tikzpicture}[trim axis right,trim axis left]
                \pgfplotsset{width=7cm, height=6.0cm}
                \begin{axis}[grid=major, xlabel={Distance (km)}, ylabel={Impedance (pu)}, /pgf/number format/.cd, legend style={at={(0.95,0.35)},anchor=south east,legend columns=1, draw=none, inner sep=0pt,fill=gray!10}, xtick={0,20,...,100}, axis line style = thick, yticklabel style={/pgf/number format/fixed, /pgf/number format/precision=4}, scaled y ticks = false, ytick={0, 0.02,..., 0.1}, xmin=0, xmax=100]
                \addplot[very thick, black] table[x=x, y=y, col sep=comma] {Data/cable/Zth_re.csv};
                \addplot[very thick, gray] table[x=x, y=y, col sep=comma] {Data/cable/Zth_im.csv};
                \legend{$\Re (\underline{Z}'_{th})$, $\Im (\underline{Z}'_{th})$};
                \end{axis}
            \end{tikzpicture}
      \end{subfigure}
    \caption{Influence of the cable distance on the Thevenin voltage and impedance}
    \label{fig:dist_th}
  \end{figure}
  
The real part of the voltage remains dominant across all distances, although it tends to increase with larger distances. The contrary pattern is experienced by the imaginary part. About the impedances, as one could expect, both of them grow with longer distances. The reactive part is considerably bigger than the real one for all distances. 

% We now deal with a submarine cable, which is modelled with a $\pi$ equivalent, where the series impedance is still $\underline{Z}_a$ while the parallel impedance is $\underline{Z}_c=-jX_c$. As follows, we sweep across $X_c$.

\subsection{Balanced fault}
Figure \ref{fig:3x1_s} depicts the optimal currents for the balanced fault.

\begin{figure}[!htb]\centering \footnotesize
    \begin{subfigure}[!htb]{.4\textwidth}
      \centering
          \begin{tikzpicture}[trim axis right,trim axis left]
              \pgfplotsset{width=7cm, height=5.5cm}
              \begin{axis}[grid=major, xlabel={$X_c$}, ylabel={${I}^+_{re}$}, /pgf/number format/.cd, legend style={at={(0.98,0.15)},anchor=south east,legend columns=1, draw=none, inner sep=0pt,fill=gray!10}, xtick={0,20,...,100}, ytick={0.0,0.05,...,0.25}, axis line style = very thick, yticklabel style={/pgf/number format/fixed, /pgf/number format/precision=2}]
              \addplot[very thick, black] table[x=x, y=y, meta=label, col sep=comma] {Data/submarine/I1_re_3x.csv};
              \addplot[very thick, gray] table[x=x, y=y, meta=label, col sep=comma] {Data/submarine/RI1_re_3x.csv};
              % \legend{BF, OPT};
              \end{axis}
          \end{tikzpicture}
    \end{subfigure}
    \hspace{1.5cm}
    \begin{subfigure}[!htb]{.4\textwidth}
        \centering
            \begin{tikzpicture}[trim axis right,trim axis left]
                \pgfplotsset{width=7cm, height=5.5cm}
                \begin{axis}[grid=major, xlabel={$X_c$}, ylabel={${I}^+_{im}$}, /pgf/number format/.cd, legend style={at={(0.98,0.15)},anchor=south east,legend columns=1, draw=none, inner sep=0pt,fill=gray!10}, xtick={0,20,...,100}, ytick={-1,-0.995,...,-0.975}, axis line style = very thick, yticklabel style={/pgf/number format/fixed, /pgf/number format/precision=3}]
                \addplot[very thick, black] table[x=x, y=y, meta=label, col sep=comma] {Data/submarine/I1_im_3x.csv};
                \addplot[very thick, gray] table[x=x, y=y, meta=label, col sep=comma] {Data/submarine/RI1_im_3x.csv};
                % \legend{OPT};
                \end{axis}
            \end{tikzpicture}
      \end{subfigure}

      \vspace{0.5cm}

      \begin{subfigure}[!htb]{.4\textwidth}
        \centering
            \begin{tikzpicture}[trim axis right,trim axis left]
                \pgfplotsset{width=7cm, height=5.5cm}
                \begin{axis}[grid=major, xlabel={$X_c$}, ylabel={${I}^-_{re}$}, /pgf/number format/.cd, legend style={at={(0.98,0.15)},anchor=south east,legend columns=1, draw=none, inner sep=0pt,fill=gray!10}, xtick={0,20,...,100}, ymin = -0.2, ymax=0.2, axis line style = very thick]
                \addplot[very thick, black] table[x=x, y=y, meta=label, col sep=comma] {Data/submarine/I2_re_3x.csv};
                \addplot[very thick, gray] table[x=x, y=y, meta=label, col sep=comma] {Data/submarine/RI2_re_3x.csv};
                % \legend{BF, OPT};
                \end{axis}
            \end{tikzpicture}
      \end{subfigure}
      \hspace{1.5cm}
      \begin{subfigure}[!htb]{.4\textwidth}
          \centering
              \begin{tikzpicture}[trim axis right,trim axis left]
                  \pgfplotsset{width=7cm, height=5.5cm}
                  \begin{axis}[grid=major, xlabel={$X_c$}, ylabel={${I}^-_{im}$}, /pgf/number format/.cd, legend style={at={(0.98,0.15)},anchor=south east,legend columns=1, draw=none, inner sep=0pt,fill=gray!10}, xtick={0,20,...,100}, ymin = -0.2, ymax=0.2, axis line style = very thick]
                  \addplot[very thick, black] table[x=x, y=y, meta=label, col sep=comma] {Data/submarine/I2_im_3x.csv};
                  \addplot[very thick, gray] table[x=x, y=y, meta=label, col sep=comma] {Data/submarine/RI2_im_3x.csv};
                  % \legend{BF, OPT};
                  \end{axis}
              \end{tikzpicture}
        \end{subfigure}

        \vspace{0.2cm}

    \begin{center}
        \begin{subfigure}[!htb]{0.7\textwidth}
                \begin{tikzpicture}[trim axis right,trim axis left]
                    \pgfplotsset{width=12.5cm, height=5.5cm}
                    \begin{axis}[grid=major, xlabel={$X_c$}, ylabel={$f$}, /pgf/number format/.cd, legend style={at={(0.98,0.2)},anchor=south east,legend columns=1, draw=none, inner sep=0pt,fill=gray!10}, xtick={0,10,...,100}, axis line style = very thick, yticklabel style={/pgf/number format/fixed, /pgf/number format/precision=3}]
                   \addplot[very thick, black] table[x=x, y=y, meta=label, col sep=comma] {Data/submarine/ff_3x.csv};
                   \addplot[very thick, gray] table[x=x, y=y, meta=label, col sep=comma] {Data/submarine/Rff_3x.csv};
                    \legend{OPT, ROPT};
                    \end{axis}
                \end{tikzpicture}
        \end{subfigure}
    \end{center}
    \caption{Influence of the currents on the objective function for the balanced fault and a submarine cable. OPT: solution to the optimization problem, ROPT: solution to the optimization problem restricted to only injecting reactive power.}
    \label{fig:3x1_s}
  \end{figure}

\subsection{Line to ground fault}
Figure \ref{fig:LGx1_s} depicts the optimal currents for the line to ground fault.
\begin{figure}[!htb]\centering \footnotesize
    \begin{subfigure}[!htb]{.4\textwidth}
      \centering
          \begin{tikzpicture}[trim axis right,trim axis left]
              \pgfplotsset{width=7cm, height=5.9cm}
              \begin{axis}[grid=major, xlabel={$R/X$}, ylabel={${I}^+_{re}$}, /pgf/number format/.cd, legend style={at={(0.98,0.15)},anchor=south east,legend columns=1, draw=none, inner sep=0pt,fill=gray!10}, xtick={0,20,...,100}, ytick={0,0.02,...,0.1}, axis line style = very thick, yticklabel style={/pgf/number format/fixed, /pgf/number format/precision=2}]
              \addplot[very thick, black] table[x=x, y=y, meta=label, col sep=comma] {Data/submarine/I1_re_LG.csv};
              \addplot[very thick, gray] table[x=x, y=y, meta=label, col sep=comma] {Data/submarine/RI1_re_LG.csv};
              % \legend{BF, OPT};
              \end{axis}
          \end{tikzpicture}
    \end{subfigure}
    \hspace{1.5cm}
    \begin{subfigure}[!htb]{.4\textwidth}
        \centering
            \begin{tikzpicture}[trim axis right,trim axis left]
                \pgfplotsset{width=7cm, height=5.9cm}
                \begin{axis}[grid=major, xlabel={$R/X$}, ylabel={${I}^+_{im}$}, /pgf/number format/.cd, legend style={at={(0.98,0.15)},anchor=south east,legend columns=1, draw=none, inner sep=0pt,fill=gray!10}, xtick={0,20,...,100}, ytick={-0.8,-0.7,...,-0.3}, axis line style = very thick]
                \addplot[very thick, black] table[x=x, y=y, meta=label, col sep=comma] {Data/submarine/I1_im_LG.csv};
                \addplot[very thick, gray] table[x=x, y=y, meta=label, col sep=comma] {Data/submarine/RI1_im_LG.csv};
                % \legend{OPT};
                \end{axis}
            \end{tikzpicture}
      \end{subfigure}

      \vspace{0.5cm}

      \begin{subfigure}[!htb]{.4\textwidth}
        \centering
            \begin{tikzpicture}[trim axis right,trim axis left]
                \pgfplotsset{width=7cm, height=5.9cm}
                \begin{axis}[grid=major, xlabel={$R/X$}, ylabel={${I}^-_{re}$}, /pgf/number format/.cd, legend style={at={(0.98,0.15)},anchor=south east,legend columns=1, draw=none, inner sep=0pt,fill=gray!10}, xtick={0,20,...,100}, axis line style = very thick, scaled y ticks=false, yticklabel style={/pgf/number format/fixed,/pgf/number format/precision=2}]
                \addplot[very thick, black] table[x=x, y=y, meta=label, col sep=comma] {Data/submarine/I2_re_LG.csv};
                \addplot[very thick, gray] table[x=x, y=y, meta=label, col sep=comma] {Data/submarine/RI2_re_LG.csv};
                % \legend{BF, OPT};
                \end{axis}
            \end{tikzpicture}
      \end{subfigure}
      \hspace{1.5cm}
      \begin{subfigure}[!htb]{.4\textwidth}
          \centering
              \begin{tikzpicture}[trim axis right,trim axis left]
                  \pgfplotsset{width=7cm, height=5.9cm}
                  \begin{axis}[grid=major, xlabel={$R/X$}, ylabel={${I}^-_{im}$}, /pgf/number format/.cd, legend style={at={(0.98,0.15)},anchor=south east,legend columns=1, draw=none, inner sep=0pt,fill=gray!10}, xtick={0,20,...,100}, ytick={-0.7,-0.6,...,-0.2}, axis line style = very thick]
                  \addplot[very thick, black] table[x=x, y=y, meta=label, col sep=comma] {Data/submarine/I2_im_LG.csv};
                  \addplot[very thick, gray] table[x=x, y=y, meta=label, col sep=comma] {Data/submarine/RI2_im_LG.csv};
                  % \legend{BF, OPT};
                  \end{axis}
              \end{tikzpicture}
        \end{subfigure}

        \vspace{0.2cm}

    \begin{center}
        \begin{subfigure}[!htb]{0.7\textwidth}
                \begin{tikzpicture}[trim axis right,trim axis left]
                    \pgfplotsset{width=12.5cm, height=5.9cm}
                    \begin{axis}[grid=major, xlabel={$R/X$}, ylabel={$f$}, /pgf/number format/.cd, legend style={at={(0.98,0.2)},anchor=south east,legend columns=1, draw=none, inner sep=0pt,fill=gray!10}, xtick={0,10,...,100}, ytick={0.13,0.14,...,0.17}, axis line style = very thick, yticklabel style={/pgf/number format/fixed, /pgf/number format/precision=3}]
                   \addplot[very thick, black] table[x=x, y=y, meta=label, col sep=comma] {Data/submarine/ff_LG.csv};
                   \addplot[very thick, gray] table[x=x, y=y, meta=label, col sep=comma] {Data/submarine/Rff_LG.csv};
                    \legend{OPT, ROPT};
                    \end{axis}
                \end{tikzpicture}
        \end{subfigure}
    \end{center}
    \caption{Influence of the currents on the objective function for the line to ground fault and a submarine cable. OPT: solution to the optimization problem, ROPT: solution to the optimization problem restricted to only injecting reactive power.}
    \label{fig:LGx1_s}
  \end{figure}

\subsection{Line to line fault}
Figure \ref{fig:LLx1_s} depicts the optimal currents for the line to line fault.

\begin{figure}[!htb]\centering \footnotesize
    \begin{subfigure}[!htb]{.4\textwidth}
      \centering
          \begin{tikzpicture}[trim axis right,trim axis left]
              \pgfplotsset{width=7cm, height=6.0cm}
              \begin{axis}[grid=major, xlabel={$X_c$}, ylabel={${I}^+_{re}$}, /pgf/number format/.cd, legend style={at={(0.98,0.15)},anchor=south east,legend columns=1, draw=none, inner sep=0pt,fill=gray!10}, xtick={0,20,...,100}, ymax = 0.2, ymin=-0.2, axis line style = very thick, scaled y ticks=false, yticklabel style={/pgf/number format/fixed, /pgf/number format/precision=2}]
              \addplot[very thick, black] table[x=x, y=y, meta=label, col sep=comma] {Data/submarine/I1_re_LL.csv};
              \addplot[very thick, gray] table[x=x, y=y, meta=label, col sep=comma] {Data/submarine/RI1_re_LL.csv};
              % \legend{BF, OPT};
              \end{axis}
          \end{tikzpicture}
    \end{subfigure}
    \hspace{1.5cm}
    \begin{subfigure}[!htb]{.4\textwidth}
        \centering
            \begin{tikzpicture}[trim axis right,trim axis left]
                \pgfplotsset{width=7cm, height=6.0cm}
                \begin{axis}[grid=major, xlabel={$X_c$}, ylabel={${I}^+_{im}$}, /pgf/number format/.cd, legend style={at={(0.98,0.15)},anchor=south east,legend columns=1, draw=none, inner sep=0pt,fill=gray!10}, xtick={0,20,...,100}, ytick={-1,-0.75,...,0}, axis line style = very thick]
                \addplot[very thick, black] table[x=x, y=y, meta=label, col sep=comma] {Data/submarine/I1_im_LL.csv};
                \addplot[very thick, gray] table[x=x, y=y, meta=label, col sep=comma] {Data/submarine/RI1_im_LL.csv};
                % \legend{OPT};
                \end{axis}
            \end{tikzpicture}
      \end{subfigure}

      \vspace{0.5cm}

      \begin{subfigure}[!htb]{.4\textwidth}
        \centering
            \begin{tikzpicture}[trim axis right,trim axis left]
                \pgfplotsset{width=7cm, height=6.0cm}
                \begin{axis}[grid=major, xlabel={$X_c$}, ylabel={${I}^-_{re}$}, /pgf/number format/.cd, legend style={at={(0.98,0.15)},anchor=south east,legend columns=1, draw=none, inner sep=0pt,fill=gray!10}, xtick={0,20,...,100}, ytick={-1,-0.75,...,0}, axis line style = very thick]
                \addplot[very thick, black] table[x=x, y=y, meta=label, col sep=comma] {Data/submarine/I2_re_LL.csv};
                \addplot[very thick, gray] table[x=x, y=y, meta=label, col sep=comma] {Data/submarine/RI2_re_LL.csv};
                % \legend{BF, OPT};
                \end{axis}
            \end{tikzpicture}
      \end{subfigure}
      \hspace{1.5cm}
      \begin{subfigure}[!htb]{.4\textwidth}
          \centering
              \begin{tikzpicture}[trim axis right,trim axis left]
                  \pgfplotsset{width=7cm, height=6.0cm}
                  \begin{axis}[grid=major, xlabel={$X_c$}, ylabel={${I}^-_{im}$}, /pgf/number format/.cd, legend style={at={(0.98,0.15)},anchor=south east,legend columns=1, draw=none, inner sep=0pt,fill=gray!10}, xtick={0,20,...,100}, axis line style = very thick, yticklabel style={/pgf/number format/fixed, /pgf/number format/precision=2}]
                  \addplot[very thick, black] table[x=x, y=y, meta=label, col sep=comma] {Data/submarine/I2_im_LL.csv};
                  \addplot[very thick, gray] table[x=x, y=y, meta=label, col sep=comma] {Data/submarine/RI2_im_LL.csv};
                  % \legend{BF, OPT};
                  \end{axis}
              \end{tikzpicture}
        \end{subfigure}

        \vspace{0.2cm}

    \begin{center}
        \begin{subfigure}[!htb]{0.7\textwidth}
                \begin{tikzpicture}[trim axis right,trim axis left]
                    \pgfplotsset{width=12.5cm, height=6cm}
                    \begin{axis}[grid=major, xlabel={$X_c$}, ylabel={$f$}, /pgf/number format/.cd, legend style={at={(0.98,0.5)},anchor=south east,legend columns=1, draw=none, inner sep=0pt,fill=gray!10}, xtick={0,10,...,100}, axis line style = very thick, yticklabel style={/pgf/number format/fixed, /pgf/number format/precision=2}]
                   \addplot[very thick, black] table[x=x, y=y, meta=label, col sep=comma] {Data/submarine/ff_LL.csv};
                   \addplot[very thick, gray] table[x=x, y=y, meta=label, col sep=comma] {Data/submarine/Rff_LL.csv};
                    \legend{OPT, ROPT};
                    \end{axis}
                \end{tikzpicture}
        \end{subfigure}
    \end{center}
    \caption{Influence of the currents on the objective function for the line to line fault and a submarine cable. OPT: solution to the optimization problem, ROPT: solution to the optimization problem restricted to only injecting reactive power.}
    \label{fig:LLx1_s}
  \end{figure}

\subsection{Double line to ground fault}
Figure \ref{fig:LLGx1_s} depicts the optimal currents for the balanced fault.

\begin{figure}[!htb]\centering \footnotesize
    \begin{subfigure}[!htb]{.4\textwidth}
      \centering
          \begin{tikzpicture}[trim axis right,trim axis left]
              \pgfplotsset{width=7cm, height=6.0cm}
              \begin{axis}[grid=major, xlabel={$X_c$}, ylabel={${I}^+_{re}$}, /pgf/number format/.cd, legend style={at={(0.98,0.15)},anchor=south east,legend columns=1, draw=none, inner sep=0pt,fill=gray!10},  xtick={0,20,...,100}, ytick={-0.01,-0.0075,...,0}, axis line style = very thick, yticklabel style={/pgf/number format/fixed, /pgf/number format/precision=4}, scaled y ticks=false]
              \addplot[very thick, black] table[x=x, y=y, meta=label, col sep=comma] {Data/submarine/I1_re_LLG.csv};
              \addplot[very thick, gray] table[x=x, y=y, meta=label, col sep=comma] {Data/submarine/RI1_re_LLG.csv};
              % \legend{BF, OPT};
              \end{axis}
          \end{tikzpicture}
    \end{subfigure}
    \hspace{1.5cm}
    \begin{subfigure}[!htb]{.4\textwidth}
        \centering
            \begin{tikzpicture}[trim axis right,trim axis left]
                \pgfplotsset{width=7cm, height=6.0cm}
                \begin{axis}[grid=major, xlabel={$X_c$}, ylabel={${I}^+_{im}$}, /pgf/number format/.cd, legend style={at={(0.98,0.15)},anchor=south east,legend columns=1, draw=none, inner sep=0pt,fill=gray!10}, xtick={0,20,...,100}, ytick={-1,-0.75,...,0}, yticklabel style={/pgf/number format/fixed, /pgf/number format/precision=5}, axis line style = very thick]
                \addplot[very thick, black] table[x=x, y=y, meta=label, col sep=comma] {Data/submarine/I1_im_LLG.csv};
                \addplot[very thick, gray] table[x=x, y=y, meta=label, col sep=comma] {Data/submarine/RI1_im_LLG.csv};
                % \legend{OPT};
                \end{axis}
            \end{tikzpicture}
      \end{subfigure}

      \vspace{0.5cm}

      \begin{subfigure}[!htb]{.4\textwidth}
        \centering
            \begin{tikzpicture}[trim axis right,trim axis left]
                \pgfplotsset{width=7cm, height=6.0cm}
                \begin{axis}[grid=major, xlabel={$X_c$}, ylabel={${I}^-_{re}$}, /pgf/number format/.cd, legend style={at={(0.98,0.15)},anchor=south east,legend columns=1, draw=none, inner sep=0pt,fill=gray!10}, xtick={0,20,...,100}, ytick={-1,-0.75,...,0}, axis line style = very thick, yticklabel style={/pgf/number format/fixed, /pgf/number format/precision=5}, scaled y ticks=false]
                \addplot[very thick, black] table[x=x, y=y, meta=label, col sep=comma] {Data/submarine/I2_re_LLG.csv};
                \addplot[very thick, gray] table[x=x, y=y, meta=label, col sep=comma] {Data/submarine/RI2_re_LLG.csv};
                % \legend{BF, OPT};
                \end{axis}
            \end{tikzpicture}
      \end{subfigure}
      \hspace{1.5cm}
      \begin{subfigure}[!htb]{.4\textwidth}
          \centering
              \begin{tikzpicture}[trim axis right,trim axis left]
                  \pgfplotsset{width=7cm, height=6.0cm}
                  \begin{axis}[grid=major, xlabel={$X_c$}, ylabel={${I}^-_{im}$}, /pgf/number format/.cd, legend style={at={(0.98,0.15)},anchor=south east,legend columns=1, draw=none, inner sep=0pt,fill=gray!10}, xtick={0,20,...,100}, ytick={0.01,0.04,...,0.16}, axis line style = very thick, yticklabel style={/pgf/number format/fixed, /pgf/number format/precision=2}, scaled y ticks=false]
                  \addplot[very thick, black] table[x=x, y=y, meta=label, col sep=comma] {Data/submarine/I2_im_LLG.csv};
                  \addplot[very thick, gray] table[x=x, y=y, meta=label, col sep=comma] {Data/submarine/RI2_im_LLG.csv};
                  % \legend{BF, OPT};
                  \end{axis}
              \end{tikzpicture}
        \end{subfigure}

        \vspace{0.2cm}

    \begin{center}
        \begin{subfigure}[!htb]{0.7\textwidth}
                \begin{tikzpicture}[trim axis right,trim axis left]
                    \pgfplotsset{width=12.5cm, height=6cm}
                    \begin{axis}[grid=major, xlabel={$X_c$}, ylabel={$f$}, /pgf/number format/.cd, legend style={at={(0.98,0.2)},anchor=south east,legend columns=1, draw=none, inner sep=0pt,fill=gray!10}, xtick={0,10,...,100}, ytick={0.2,0.21,...,0.24}, yticklabel style={/pgf/number format/fixed, /pgf/number format/precision=5}, axis line style = very thick]
                   \addplot[very thick, black] table[x=x, y=y, meta=label, col sep=comma] {Data/submarine/ff_LLG.csv};
                   \addplot[very thick, gray] table[x=x, y=y, meta=label, col sep=comma] {Data/submarine/Rff_LLG.csv};
                    \legend{OPT, ROPT};
                    \end{axis}
                \end{tikzpicture}
        \end{subfigure}
    \end{center}
    \caption{Influence of the currents on the objective function for the double line to ground fault and a submarine cable. OPT: solution to the optimization problem, ROPT: solution to the optimization problem restricted to only injecting reactive power.}
    \label{fig:LLGx1_s}
  \end{figure}
The plots for the submarine cable indicate that while the distribution of currents changes substantially when considering the restriction in the active current, the objective functions tend to take similar values. 

For instance, in the balanced fault, the best strategy in the ROPT case is to inject a maximum imaginary positive sequence current. On the contrary, the imaginary positive sequence current does not reach the limits of one. Instead, some part of the current is dedicated to the real part. As in the $R/X$ case of study, the negative sequence currents remain null for the full sweep. One can check that the objective function for a small $R/X$ ratio seems to coincide with the function when $X_c$ takes a large value. Again, the OPT scenario is slightly better than the ROPT one. 

In the line to ground fault the objective functions is not far apart from the results from Figure \ref{fig:LGx1_c}. However, this time we have not obtained a discontinuous profile in the evolution of the currents. This shows that maybe in this case there is only a single minimum, or also, that the optimal values follow the same trajectory due to the initialization. Oddly enough, in the OPT situation the imaginary positive sequence current takes rather small values. The real negative sequence current becomes predominant. It is shocking that despite the enormous differences in the distribution of currents, the objective functions are not far apart one from the other. 

The results for the line to line fault together with the ones from the double line to ground fault seem to be the most dubious. First, in the line to line fault the objective functions are somewhat larger than what it could be expected from Figure \ref{fig:LLx1_c}. In any case, while the real positive sequence and the imaginary negative sequence currents take extremely similar values, the differences are acute for the remaining two currents. The OPT case prioritizes the real negative sequence current whereas the ROPT opts for the imaginary positive sequence current. 

No more intuitively sound seem to be the double line to ground fault values. The objective function becomes considerably smaller than in the case of the $R/X$ analysis, and again, the distribution of currents reminds of the one for the line to line fault. This could be expected. However, the extreme differences are hardly justifiable. 
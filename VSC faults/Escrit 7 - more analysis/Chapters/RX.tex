\section{Variable resistance/inductance ratio}
In this section we try to answer to the question of what happens when the $R/X$ ratio varies. This way we experiment with cases where the resistive part is considerably larger than in the aforementioned anlysis. The fault impedance $\underline{Z}_f$ has been set at $0.1$, which is probably more realistic than $0.0 + 0.1j$. 

\subsection{Balanced fault}
Figure \ref{fig:3x1_c} depicts the optimal currents for the balanced fault.

\begin{figure}[!htb]\centering \footnotesize
    \begin{subfigure}[!htb]{.4\textwidth}
      \centering
          \begin{tikzpicture}[trim axis right,trim axis left]
              \pgfplotsset{width=7cm, height=5.5cm}
              \begin{axis}[grid=major, xlabel={$R/X$}, ylabel={${I}^+_{re}$}, /pgf/number format/.cd, legend style={at={(0.98,0.15)},anchor=south east,legend columns=1, draw=none, inner sep=0pt,fill=gray!10}, xtick={0,1,...,5}, ytick={0,0.1,...,0.4}, axis line style = very thick]
              \addplot[very thick, black] table[x=x, y=y, meta=label, col sep=comma] {Data/RX/I1_re_3x.csv};
              \addplot[very thick, gray] table[x=x, y=y, meta=label, col sep=comma] {Data/RX/RI1_re_3x.csv};
              % \legend{BF, OPT};
              \end{axis}
          \end{tikzpicture}
    \end{subfigure}
    \hspace{1.5cm}
    \begin{subfigure}[!htb]{.4\textwidth}
        \centering
            \begin{tikzpicture}[trim axis right,trim axis left]
                \pgfplotsset{width=7cm, height=5.5cm}
                \begin{axis}[grid=major, xlabel={$R/X$}, ylabel={${I}^+_{im}$}, /pgf/number format/.cd, legend style={at={(0.98,0.15)},anchor=south east,legend columns=1, draw=none, inner sep=0pt,fill=gray!10}, xtick={0,1,...,5}, axis line style = very thick]
                \addplot[very thick, black] table[x=x, y=y, meta=label, col sep=comma] {Data/RX/I1_im_3x.csv};
                \addplot[very thick, gray] table[x=x, y=y, meta=label, col sep=comma] {Data/RX/RI1_im_3x.csv};
                % \legend{OPT};
                \end{axis}
            \end{tikzpicture}
      \end{subfigure}

      \vspace{0.5cm}

      \begin{subfigure}[!htb]{.4\textwidth}
        \centering
            \begin{tikzpicture}[trim axis right,trim axis left]
                \pgfplotsset{width=7cm, height=5.5cm}
                \begin{axis}[grid=major, xlabel={$R/X$}, ylabel={${I}^-_{re}$}, /pgf/number format/.cd, legend style={at={(0.98,0.15)},anchor=south east,legend columns=1, draw=none, inner sep=0pt,fill=gray!10}, xtick={0,1,...,5}, ymin = -0.5, ymax=0.5, axis line style = very thick]
                \addplot[very thick, black] table[x=x, y=y, meta=label, col sep=comma] {Data/RX/I2_re_3x.csv};
                \addplot[very thick, gray] table[x=x, y=y, meta=label, col sep=comma] {Data/RX/RI2_re_3x.csv};
                % \legend{BF, OPT};
                \end{axis}
            \end{tikzpicture}
      \end{subfigure}
      \hspace{1.5cm}
      \begin{subfigure}[!htb]{.4\textwidth}
          \centering
              \begin{tikzpicture}[trim axis right,trim axis left]
                  \pgfplotsset{width=7cm, height=5.5cm}
                  \begin{axis}[grid=major, xlabel={$R/X$}, ylabel={${I}^-_{im}$}, /pgf/number format/.cd, legend style={at={(0.98,0.15)},anchor=south east,legend columns=1, draw=none, inner sep=0pt,fill=gray!10}, xtick={0,1,...,5}, ymin = -0.5, ymax=0.5, axis line style = very thick]
                  \addplot[very thick, black] table[x=x, y=y, meta=label, col sep=comma] {Data/RX/I2_im_3x.csv};
                  \addplot[very thick, gray] table[x=x, y=y, meta=label, col sep=comma] {Data/RX/RI2_im_3x.csv};
                  % \legend{BF, OPT};
                  \end{axis}
              \end{tikzpicture}
        \end{subfigure}

        \vspace{0.2cm}

    \begin{center}
        \begin{subfigure}[!htb]{0.7\textwidth}
                \begin{tikzpicture}[trim axis right,trim axis left]
                    \pgfplotsset{width=12.5cm, height=5.5cm}
                    \begin{axis}[grid=major, xlabel={$R/X$}, ylabel={$f$}, /pgf/number format/.cd, legend style={at={(0.98,0.2)},anchor=south east,legend columns=1, draw=none, inner sep=0pt,fill=gray!10}, xtick={0,0.5,...,5}, ymin = -0.0, ymax=0.25, ytick={0,0.05,...,0.25}, axis line style = very thick, yticklabel style={/pgf/number format/fixed, /pgf/number format/precision=2}]
                   \addplot[very thick, black] table[x=x, y=y, meta=label, col sep=comma] {Data/RX/ff_3x.csv};
                   \addplot[very thick, gray] table[x=x, y=y, meta=label, col sep=comma] {Data/RX/Rff_3x.csv};
                    \legend{OPT, ROPT};
                    \end{axis}
                \end{tikzpicture}
        \end{subfigure}
    \end{center}
    \caption{Influence of the currents on the objective function for the balanced fault and a changing $R/X$ ratio. OPT: solution to the optimization problem, ROPT: solution to the optimization problem restricted to only injecting reactive power.}
    \label{fig:3x1_c}
  \end{figure}

\subsection{Line to ground fault}
Figure \ref{fig:LGx1_c} depicts the optimal currents for the line to ground fault.
\begin{figure}[!htb]\centering \footnotesize
    \begin{subfigure}[!htb]{.4\textwidth}
      \centering
          \begin{tikzpicture}[trim axis right,trim axis left]
              \pgfplotsset{width=7cm, height=5.9cm}
              \begin{axis}[grid=major, xlabel={$R/X$}, ylabel={${I}^+_{re}$}, /pgf/number format/.cd, legend style={at={(0.98,0.15)},anchor=south east,legend columns=1, draw=none, inner sep=0pt,fill=gray!10}, xtick={0,1,...,5}, axis line style = very thick, ytick={-0.1,-0.05,...,0.2}, yticklabel style={/pgf/number format/fixed, /pgf/number format/precision=2}]
              \addplot[very thick, black] table[x=x, y=y, meta=label, col sep=comma] {Data/RX/I1_re_LG.csv};
              \addplot[very thick, gray] table[x=x, y=y, meta=label, col sep=comma] {Data/RX/RI1_re_LG.csv};
              % \legend{BF, OPT};
              \end{axis}
          \end{tikzpicture}
    \end{subfigure}
    \hspace{1.5cm}
    \begin{subfigure}[!htb]{.4\textwidth}
        \centering
            \begin{tikzpicture}[trim axis right,trim axis left]
                \pgfplotsset{width=7cm, height=5.9cm}
                \begin{axis}[grid=major, xlabel={$R/X$}, ylabel={${I}^+_{im}$}, /pgf/number format/.cd, legend style={at={(0.98,0.15)},anchor=south east,legend columns=1, draw=none, inner sep=0pt,fill=gray!10}, xtick={0,1,...,5}, axis line style = very thick]
                \addplot[very thick, black] table[x=x, y=y, meta=label, col sep=comma] {Data/RX/I1_im_LG.csv};
                \addplot[very thick, gray] table[x=x, y=y, meta=label, col sep=comma] {Data/RX/RI1_im_LG.csv};
                % \legend{OPT};
                \end{axis}
            \end{tikzpicture}
      \end{subfigure}

      \vspace{0.5cm}

      \begin{subfigure}[!htb]{.4\textwidth}
        \centering
            \begin{tikzpicture}[trim axis right,trim axis left]
                \pgfplotsset{width=7cm, height=5.9cm}
                \begin{axis}[grid=major, xlabel={$R/X$}, ylabel={${I}^-_{re}$}, /pgf/number format/.cd, legend style={at={(0.98,0.15)},anchor=south east,legend columns=1, draw=none, inner sep=0pt,fill=gray!10}, xtick={0,1,...,5}, ymin = -0.1, ymax=0.8, axis line style = very thick]
                \addplot[very thick, black] table[x=x, y=y, meta=label, col sep=comma] {Data/RX/I2_re_LG.csv};
                \addplot[very thick, gray] table[x=x, y=y, meta=label, col sep=comma] {Data/RX/RI2_re_LG.csv};
                % \legend{BF, OPT};
                \end{axis}
            \end{tikzpicture}
      \end{subfigure}
      \hspace{1.5cm}
      \begin{subfigure}[!htb]{.4\textwidth}
          \centering
              \begin{tikzpicture}[trim axis right,trim axis left]
                  \pgfplotsset{width=7cm, height=5.9cm}
                  \begin{axis}[grid=major, xlabel={$R/X$}, ylabel={${I}^-_{im}$}, /pgf/number format/.cd, legend style={at={(0.98,0.15)},anchor=south east,legend columns=1, draw=none, inner sep=0pt,fill=gray!10}, xtick={0,1,...,5}, ymin = -0.6, ymax=0.1, axis line style = very thick]
                  \addplot[very thick, black] table[x=x, y=y, meta=label, col sep=comma] {Data/RX/I2_im_LG.csv};
                  \addplot[very thick, gray] table[x=x, y=y, meta=label, col sep=comma] {Data/RX/RI2_im_LG.csv};
                  % \legend{BF, OPT};
                  \end{axis}
              \end{tikzpicture}
        \end{subfigure}

        \vspace{0.2cm}

    \begin{center}
        \begin{subfigure}[!htb]{0.7\textwidth}
                \begin{tikzpicture}[trim axis right,trim axis left]
                    \pgfplotsset{width=12.5cm, height=5.9cm}
                    \begin{axis}[grid=major, xlabel={$R/X$}, ylabel={$f$}, /pgf/number format/.cd, legend style={at={(0.98,0.2)},anchor=south east,legend columns=1, draw=none, inner sep=0pt,fill=gray!10}, xtick={0,0.5,...,5}, ytick={0,0.025,...,0.15}, axis line style = very thick, yticklabel style={/pgf/number format/fixed, /pgf/number format/precision=3}]
                   \addplot[very thick, black] table[x=x, y=y, meta=label, col sep=comma] {Data/RX/ff_LG.csv};
                   \addplot[very thick, gray] table[x=x, y=y, meta=label, col sep=comma] {Data/RX/Rff_LG.csv};
                    \legend{OPT, ROPT};
                    \end{axis}
                \end{tikzpicture}
        \end{subfigure}
    \end{center}
    \caption{Influence of the currents on the objective function for the line to ground fault and a changing $R/X$ ratio. OPT: solution to the optimization problem, ROPT: solution to the optimization problem restricted to only injecting reactive power.}
    \label{fig:LGx1_c}
  \end{figure}

\subsection{Line to line fault}
Figure \ref{fig:LLx1_c} depicts the optimal currents for the line to line fault.

\begin{figure}[!htb]\centering \footnotesize
    \begin{subfigure}[!htb]{.4\textwidth}
      \centering
          \begin{tikzpicture}[trim axis right,trim axis left]
              \pgfplotsset{width=7cm, height=6.0cm}
              \begin{axis}[grid=major, xlabel={$R/X$}, ylabel={${I}^+_{re}$}, /pgf/number format/.cd, legend style={at={(0.98,0.15)},anchor=south east,legend columns=1, draw=none, inner sep=0pt,fill=gray!10}, xtick={0,1,...,5}, axis line style = very thick, ytick={0,0.05,...,0.3}, scaled y ticks=false, yticklabel style={/pgf/number format/fixed, /pgf/number format/precision=2}]
              \addplot[very thick, black] table[x=x, y=y, meta=label, col sep=comma] {Data/RX/I1_re_LL.csv};
              \addplot[very thick, gray] table[x=x, y=y, meta=label, col sep=comma] {Data/RX/RI1_re_LL.csv};
              % \legend{BF, OPT};
              \end{axis}
          \end{tikzpicture}
    \end{subfigure}
    \hspace{1.5cm}
    \begin{subfigure}[!htb]{.4\textwidth}
        \centering
            \begin{tikzpicture}[trim axis right,trim axis left]
                \pgfplotsset{width=7cm, height=6.0cm}
                \begin{axis}[grid=major, xlabel={$R/X$}, ylabel={${I}^+_{im}$}, /pgf/number format/.cd, legend style={at={(0.98,0.15)},anchor=south east,legend columns=1, draw=none, inner sep=0pt,fill=gray!10}, xtick={0,1,...,5}, ytick={-1,-0.75,...,0}, axis line style = very thick]
                \addplot[very thick, black] table[x=x, y=y, meta=label, col sep=comma] {Data/RX/I1_im_LL.csv};
                \addplot[very thick, gray] table[x=x, y=y, meta=label, col sep=comma] {Data/RX/RI1_im_LL.csv};
                % \legend{OPT};
                \end{axis}
            \end{tikzpicture}
      \end{subfigure}

      \vspace{0.5cm}

      \begin{subfigure}[!htb]{.4\textwidth}
        \centering
            \begin{tikzpicture}[trim axis right,trim axis left]
                \pgfplotsset{width=7cm, height=6.0cm}
                \begin{axis}[grid=major, xlabel={$R/X$}, ylabel={${I}^-_{re}$}, /pgf/number format/.cd, legend style={at={(0.98,0.15)},anchor=south east,legend columns=1, draw=none, inner sep=0pt,fill=gray!10}, xtick={0,1,...,5}, ytick={-1,-0.75,...,0}, axis line style = very thick]
                \addplot[very thick, black] table[x=x, y=y, meta=label, col sep=comma] {Data/RX/I2_re_LL.csv};
                \addplot[very thick, gray] table[x=x, y=y, meta=label, col sep=comma] {Data/RX/RI2_re_LL.csv};
                % \legend{BF, OPT};
                \end{axis}
            \end{tikzpicture}
      \end{subfigure}
      \hspace{1.5cm}
      \begin{subfigure}[!htb]{.4\textwidth}
          \centering
              \begin{tikzpicture}[trim axis right,trim axis left]
                  \pgfplotsset{width=7cm, height=6.0cm}
                  \begin{axis}[grid=major, xlabel={$R/X$}, ylabel={${I}^-_{im}$}, /pgf/number format/.cd, legend style={at={(0.98,0.15)},anchor=south east,legend columns=1, draw=none, inner sep=0pt,fill=gray!10}, xtick={0,1,...,5}, ytick={0.1,0.2,...,0.6}, axis line style = very thick]
                  \addplot[very thick, black] table[x=x, y=y, meta=label, col sep=comma] {Data/RX/I2_im_LL.csv};
                  \addplot[very thick, gray] table[x=x, y=y, meta=label, col sep=comma] {Data/RX/RI2_im_LL.csv};
                  % \legend{BF, OPT};
                  \end{axis}
              \end{tikzpicture}
        \end{subfigure}

        \vspace{0.2cm}

    \begin{center}
        \begin{subfigure}[!htb]{0.7\textwidth}
                \begin{tikzpicture}[trim axis right,trim axis left]
                    \pgfplotsset{width=12.5cm, height=6cm}
                    \begin{axis}[grid=major, xlabel={$R/X$}, ylabel={$f$}, /pgf/number format/.cd, legend style={at={(0.98,0.8)},anchor=south east,legend columns=1, draw=none, inner sep=0pt,fill=gray!10}, xtick={0,0.5,...,5}, ytick={0.38,0.39,...,0.45}, axis line style = very thick, yticklabel style={/pgf/number format/fixed, /pgf/number format/precision=2}]
                   \addplot[very thick, black] table[x=x, y=y, meta=label, col sep=comma] {Data/RX/ff_LL.csv};
                   \addplot[very thick, gray] table[x=x, y=y, meta=label, col sep=comma] {Data/RX/Rff_LL.csv};
                    \legend{OPT, ROPT};
                    \end{axis}
                \end{tikzpicture}
        \end{subfigure}
    \end{center}
    \caption{Influence of the currents on the objective function for the line to line fault and a changing $R/X$ ratio. OPT: solution to the optimization problem, ROPT: solution to the optimization problem restricted to only injecting reactive power.}
    \label{fig:LLx1_c}
  \end{figure}

\subsection{Double line to ground fault}
Figure \ref{fig:LLGx1_c} depicts the optimal currents for the balanced fault.

\begin{figure}[!htb]\centering \footnotesize
    \begin{subfigure}[!htb]{.4\textwidth}
      \centering
          \begin{tikzpicture}[trim axis right,trim axis left]
              \pgfplotsset{width=7cm, height=6.0cm}
              \begin{axis}[grid=major, xlabel={$R/X$}, ylabel={${I}^+_{re}$}, /pgf/number format/.cd, legend style={at={(0.98,0.15)},anchor=south east,legend columns=1, draw=none, inner sep=0pt,fill=gray!10}, xtick={0,1,...,5}, ytick={0,0.01,...,0.05}, axis line style = very thick, yticklabel style={/pgf/number format/fixed, /pgf/number format/precision=5}, scaled y ticks=false]
              \addplot[very thick, black] table[x=x, y=y, meta=label, col sep=comma] {Data/RX/I1_re_LLG.csv};
              \addplot[very thick, gray] table[x=x, y=y, meta=label, col sep=comma] {Data/RX/RI1_re_LLG.csv};
              % \legend{BF, OPT};
              \end{axis}
          \end{tikzpicture}
    \end{subfigure}
    \hspace{1.5cm}
    \begin{subfigure}[!htb]{.4\textwidth}
        \centering
            \begin{tikzpicture}[trim axis right,trim axis left]
                \pgfplotsset{width=7cm, height=6.0cm}
                \begin{axis}[grid=major, xlabel={$R/X$}, ylabel={${I}^+_{im}$}, /pgf/number format/.cd, legend style={at={(0.98,0.15)},anchor=south east,legend columns=1, draw=none, inner sep=0pt,fill=gray!10}, xtick={0,1,...,5}, ytick={-0.5775,-0.5770,...,-0.5750}, yticklabel style={/pgf/number format/fixed, /pgf/number format/precision=5}, axis line style = very thick]
                \addplot[very thick, black] table[x=x, y=y, meta=label, col sep=comma] {Data/RX/I1_im_LLG.csv};
                \addplot[very thick, gray] table[x=x, y=y, meta=label, col sep=comma] {Data/RX/RI1_im_LLG.csv};
                % \legend{OPT};
                \end{axis}
            \end{tikzpicture}
      \end{subfigure}

      \vspace{0.5cm}

      \begin{subfigure}[!htb]{.4\textwidth}
        \centering
            \begin{tikzpicture}[trim axis right,trim axis left]
                \pgfplotsset{width=7cm, height=6.0cm}
                \begin{axis}[grid=major, xlabel={$R/X$}, ylabel={${I}^-_{re}$}, /pgf/number format/.cd, legend style={at={(0.98,0.15)},anchor=south east,legend columns=1, draw=none, inner sep=0pt,fill=gray!10}, xtick={0,1,...,5}, ytick={-0.05,-0.04,...,0}, axis line style = very thick, yticklabel style={/pgf/number format/fixed, /pgf/number format/precision=5}, scaled y ticks=false]
                \addplot[very thick, black] table[x=x, y=y, meta=label, col sep=comma] {Data/RX/I2_re_LLG.csv};
                \addplot[very thick, gray] table[x=x, y=y, meta=label, col sep=comma] {Data/RX/RI2_re_LLG.csv};
                % \legend{BF, OPT};
                \end{axis}
            \end{tikzpicture}
      \end{subfigure}
      \hspace{1.5cm}
      \begin{subfigure}[!htb]{.4\textwidth}
          \centering
              \begin{tikzpicture}[trim axis right,trim axis left]
                  \pgfplotsset{width=7cm, height=6.0cm}
                  \begin{axis}[grid=major, xlabel={$R/X$}, ylabel={${I}^-_{im}$}, /pgf/number format/.cd, legend style={at={(0.98,0.15)},anchor=south east,legend columns=1, draw=none, inner sep=0pt,fill=gray!10}, xtick={0,1,...,5}, ytick={0.5750,0.5755,...,0.5780}, axis line style = very thick, yticklabel style={/pgf/number format/fixed, /pgf/number format/precision=4}]
                  \addplot[very thick, black] table[x=x, y=y, meta=label, col sep=comma] {Data/RX/I2_im_LLG.csv};
                  \addplot[very thick, gray] table[x=x, y=y, meta=label, col sep=comma] {Data/RX/RI2_im_LLG.csv};
                  % \legend{BF, OPT};
                  \end{axis}
              \end{tikzpicture}
        \end{subfigure}

        \vspace{0.2cm}

    \begin{center}
        \begin{subfigure}[!htb]{0.7\textwidth}
                \begin{tikzpicture}[trim axis right,trim axis left]
                    \pgfplotsset{width=12.5cm, height=6cm}
                    \begin{axis}[grid=major, xlabel={$R/X$}, ylabel={$f$}, /pgf/number format/.cd, legend style={at={(0.98,0.5)},anchor=south east,legend columns=1, draw=none, inner sep=0pt,fill=gray!10}, xtick={0,0.5,...,5}, yticklabel style={/pgf/number format/fixed, /pgf/number format/precision=5}, ytick={0.8840,0.8841,...,0.8844}, axis line style = very thick]
                   \addplot[very thick, black] table[x=x, y=y, meta=label, col sep=comma] {Data/RX/ff_LLG.csv};
                   \addplot[very thick, gray] table[x=x, y=y, meta=label, col sep=comma] {Data/RX/Rff_LLG.csv};
                    \legend{OPT, ROPT};
                    \end{axis}
                \end{tikzpicture}
        \end{subfigure}
    \end{center}
    \caption{Influence of the currents on the objective function for the double line to ground fault and a changing $R/X$ ratio. OPT: solution to the optimization problem, ROPT: solution to the optimization problem restricted to only injecting reactive power.}
    \label{fig:LLGx1_c}
  \end{figure}
About the discussion of the plots, for the balanced fault it becomes clear that injecting a large imaginary positive sequence current (in absolute value) is the preferred option to minimize the objective function. In the OPT case some of this current is traded for some real positive sequence current that allows the objective function to become slightly lower when compared to the ROPT case. Since the negative sequence equivalent circuit is decoupled from the positive sequence one, the negative sequence currents are null in all cases. Besides, the larger $R$ becomes with respect to $X$, the larger the real positive current has to be as well. This phenomena makes complete sense when considering that we want to amplify the voltage drop in the positive sequence. However, even when $R$ is about five times larger than $X$, the imaginary positive sequence current is dominant.

As explained in the initial discussion where no sweep was taking place, the line to ground case may have multiple minimums. The plots in Figure \ref{fig:LGx1_c} indicate that this could also be the case here, as the evolution of currents does not follow a smooth trajectory. Rather, it is not continuously differentiable across all its path. This phenomena happens for all currents at the same $R/X$ values but is not noticeable from the objective function plot. Such continuity on the objective function could be a great indicator of the presence of various equal minimums for a single $R/X$ value.

The line to line fault ends up presenting a larger $f$ value than in the two abovementioned cases. That is, it becomes a more severe fault. Note from the objective function that the OPT is superior to the ROPT case for all values of the $R/X$ ratio. However, the distribution of currents hugely varies for small $R/X$ ratios. When $R$ tends to be way smaller than $X$, the negative sequence current is prioritized in the OPT case. When $R$ becomes larger, this diminishes and then the imaginary positive and negative sequence currents are the major ones. Contrarily to the other faults, the bigger $R$, the better. 

In the double line to ground fault the differences across all the sweep values is minimal. The objective functions remain at almost always the same values, and even though the OPT is again the winner, the difference with the ROPT situation does not become substantial. The imaginary positive and negative sequence currents tend to take the same values but with the opposite sign, whereas the real positive sequence currents is merely the reflection of the real negative sequence current. We can deduce that since the double line to ground fault involves a solid connection between phases, it makes this turns out to be the most severe situation.
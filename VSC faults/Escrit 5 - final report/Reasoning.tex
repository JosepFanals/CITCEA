%!TEX program = lualatex
\documentclass[10pt]{article}

\usepackage[left=30mm, right=30mm, top=30mm, bottom=30mm]{geometry}
%\usepackage[scale=4]{draftwatermark}

\usepackage[backend=biber, sorting=none]{biblatex} % per la bibliografia
\setlength\bibitemsep{\baselineskip} % per tenir més espai entre bibliografia
\addbibresource{bib.bib} % carregar el fitxer de bibliografia
\usepackage{bm}
\usepackage{amsmath}
\usepackage[activate={true, noncompatibility}]{microtype}
\usepackage{tikz}
\usepackage{circuitikz}
% \usepackage[american voltages, american currents,siunitx]{circuitikz}
\usepackage{amssymb}
\usepackage{diagbox}
\usepackage{pgfplots}
\pgfplotsset{compat=1.17}
\usetikzlibrary{arrows.meta}
% \usepackage[american voltages, american currents,siunitx]{circuitikz}
\usepackage{amssymb}  % per simbolitzar reals
%\usepackage{subfigure} 
\usepackage{subcaption}

\begin{document}

\begin{center}
    \Large
    \textbf{Positive and negative sequence currents to improve voltages during unbalanced faults}
         
    \vspace{0.4cm}
    \textbf{Josep Fanals}
       
       
    \vspace{0.4cm}
    \textbf{03/2021}
\end{center}

\section{Introduction}
Grid faults constitute a group of unfortunate events that cause severe perturbations in the grid. The voltages can take values below the established minimum, or on the contrary, exceed the maximum in non-faulted phases. The currents are also susceptible to vary considerably. Traditional power systems based on synchronous generators could encounter currents surpassing the nominal values, and therefore, the fault could be clearly detected. However, the increasing integration of renewables \cite{anees2012grid} supposes a change of paradigm, in which currents can be controlled but are limited so as not to damage the Isolated-Gate Bipolar Transistors (IGBT) found in the Voltage Source Converter (VSC) \cite{abdou2013improving}. 

Transmission System Operators (TSO) are responsible for imposing requirements related to the operation under voltage sags to generators and converters \cite{tsili2009review, iov2007mapping}. Such requirements are gathered in the respective grid codes. There seems to be no clear consensus on how to restore the voltage. In this sense, even if for instance the Low Voltage Ride Through (LVRT) profiles present similarities \cite{conroy2007low}, analysis aimed at determining analytically the optimal injection of positive and negative sequence currents are not numerous. As far as we are aware, only Camacho et al. offer an optimal solution regarding the injection of active and reactive powers \cite{camacho2017positive}. 

Consequently, this work focuses on finding the most convenient positive and negative sequence currents (and not powers) to raise the voltage at the point of common coupling. First, a simple model is discussed, and then, the results are shown together with the corresponding discussion. Despite the elegancy found in closed-form expressions, we obtain the optimal operating point by either computational brute force or an iterative process with the help of the SciPy library. The expressions regarding the several types of fault studied as well as its corresponent diagrams are shown in the appendix. 

\section{Definition of the system}
The system we are considering is formed by an ideal grid (only positive sequence voltage is present) coupled to a VSC. The model is depicted in Figure \ref{fig:sys_p}. The VSC will be controlled in a way that leads to an improvement in the voltage at the PCC.

\begin{figure}[!htb] \centering
\begin{circuitikz}[european]
\thicklines

\draw (0,0) to [sV, v_=$\underline{V}_{th}$] (0,2);
\draw (-3,2) to [R, l=$\underline{Z}_{th}$] (0,2);
\draw (-0.25,0) to [short] (0.25,0);
\node at (-6,1.3) {PCC};
\node at (-6,2.7) {$\underline{V}_c$};
\draw (-3,2) to [short] (-3.2,1);
\draw (-3.2,1) to [short] (-2.8,1);
\draw[-{Latex[length=3mm]}] (-2.8,1) to [short] (-3,0);
\draw (-6,2) to [R, l=$\underline{Z}_a$] (-3,2);
\draw[line width=0.65mm] (-6,2.5) to [short] (-6,1.5);
\draw[line width=0.65mm] (-3,2.5) to [short] (-3,1.5);
\draw (-9,2) to [R, l=$\underline{Z}_c$, i=$\underline{I}$] (-6,2);
\draw (-10.0,2) to [sdcac] (-9.0,2);


\end{circuitikz}
\caption{Single-phase representation of the simple system under a fault}
\label{fig:sys_p}
\end{figure}
This study considers the balanced fault type but also the unbalanced ones, which include the line to ground, the double line to ground and the line to line fault. The model in Figure \ref{fig:sys_p} attempts to describe simple yet sufficient system to test the influence of the injected currents by the VSC on the voltage at the PCC. The system could be complicated by adding parallel capacitances on both sides of $\underline{Z}_a$ to model a hypothetical submarine cable. However, we prioritize keeping the system simple. 

We are concerned with improving the voltage $\underline{V}_c$ as a function of the current $\underline{I}$ injected by the converter. Formally speaking there is no clear definition on what improving the voltage means as it depends on rather pre-stablished preferences. For instance, one could try to maximize the positive sequence voltage, minimize the negative sequence voltage or even maximize the difference between both. These three strategies have been covered in \cite{camacho2017positive}. A more flexible approach is based on defining the objective function as
\begin{equation}
    f(\underline{I}^+, \underline{I}^-) = \lambda^+|(|\underline{V}^+_c(\underline{I}^+, \underline{I}^-)| - 1)| + \lambda^-|(|\underline{V}^-_c(\underline{I}^+, \underline{I}^-) - 0|)|,
    \label{eq:1}
\end{equation}
where the weighting factors $\lambda^+$, $\lambda^- \in \mathbb{R}$. By adjusting these factors one can follow the three aforementioned strategies. Note that the goal is to obtain a positive sequence voltage as close as possible to one (in per unit) while simultaneously approaching zero in the negative sequence voltage. Even though the problem is described as a function of the positive and negative sequence currents, it could also be stated as a function of the original $abc$ currents without loss of generality. The associated expressions are gathered in the appendix as well.

Minimizing $f$ does not come with complete freedom. That is, the currents are constrained so as not to exceed the IGBT limits. It becomes more convenient to express the constraints in the $abc$ frame:
\begin{equation}
    g(\underline{I}_a, \underline{I}_b, \underline{I}_c) = \begin{cases}
        |\underline{I}_a|\leq I_{max},\\
        |\underline{I}_b|\leq I_{max},\\
        |\underline{I}_c|\leq I_{max}.\\
    \end{cases}
\label{eq:2}
\end{equation}
For now we are not concerned with the voltages imposed to the semiconductors due to the fact that the filter $\underline{Z}_c$ is most likely to take small values. In practical situations current limitations are the most serious constraint. Therefore, we define the optimization problem as follows:
\begin{subequations}
\begin{alignat}{2}
&\!\min_{\underline{I}^+,\underline{I}^-}        &\qquad& f(\underline{I}^+,\underline{I}^-)\label{eq:optProb}\\
&\text{subject to} &      & g(\underline{I}_a, \underline{I}_b, \underline{I}_c) ,\label{eq:constraint1}
\end{alignat}
\end{subequations}
where the currents in the $abc$ frame can be related to the currents expressed in the symmetrical components form by means of Fortescue's transformation, and viceversa \cite{fortescue1918method}. As a direct consequence of that, the analysis can be performed in the $abc$ frame while expressing the voltages as a function of the positive, negative and homopolar components. Another valid procedure is to work with the symmetrical components and relate the currents to the $abc$ ones. Both ways to confront the problem are equally valid. The results obtained in this study have been generated and validated with both paths.

Current grid codes define the low voltage ride through (LVRT) limit curve which indicates the relation between the duration of a fault and the voltage frontier that indicates if, for instance, a wind turbine ought to disconnect. Such LVRT curves present slight variations from country to country \cite{tsili2009review} but they all share the same pattern: in case of a severe fault, the disconnection should not take place if it has a low duration; on the contrary, less noticeable faults imply a larger connection time. Grid codes usually impose a curtailment of active power enforce the generation of reactive power in order to collaborate on raising the voltage \cite{altin2010overview,serban2016voltage}.  % review grid codes and explain their priorization

Nevertheless, studies dealing with determining the optimal currents to inject are not precisely numerous. At most, \cite{camacho2012flexible} express the voltages in terms of active and reactive power but do not consider constraints. \cite{camacho2017positive} formulate the optimization problem and arrive to a closed-form expression, even though it becomes iterative. The analysis is performed contemplating powers rather than currents. Because of that, this work focuses on computing the optimal currents to increase the positive sequence voltage as close to one as possible and achieving a negative sequence voltage that approaches zero. In essence, the results that follow come from solving Equations \ref{eq:optProb} and \ref{eq:constraint1}.

% The point of common coupling (PCC) is where the VSC together with its filter are connected. There are two restrictions to take into account in a VSC, one related to the maximum current and another to the voltage. The current limitation is likely the most relevant when operating under faults. Since the current is limited and the filter used to connect the VSC to the PCC takes rather low values, we can expect the voltage drop to not be substantial. Because of that, and taking into consideration the voltage sag at the grid side, the voltage limit is hardly ever surpassed. Thus, in the analysis that follows, we only impose the current restriction. As future work, we could also add the voltage limitations as a constraint, although probably the conclusions will not vary from the ones extracted here.

% Note that Figure \ref{fig:sys_p} is general, in the sense that it does not specify the type of fault. Besides, there will be a fault impedance, denoted by $\underline{Z}_f$. We believe every type of fault deserves to be studied separately. We are going to employ the symmetrical components, which are meant to simplify the analysis. In each fault, the voltage $\underline{V}_c$ will be decomposed in positive and negative sequence voltage and expressed as a function of the voltage at the grid together with the injected positive and negative sequence currents. It makes sense to model the VSC and its filter as a current source for this purpose. 

\section{Analysis}
Four types of fault are considered in this study. One is the balanced fault, which yields a simple yet valuable system to analyze the distribution of optimal currents. The remaining faults are unbalanced: the line to ground, the line to line and the double line to ground cases. 

In all situations, we show plots representing the objective function depending on the real and imaginary parts of the positive and negative sequence currents. They are obtained in a brute force manner, that is, we generate multiple combinations of currents and store those points if they do not respect the constraints. The solution obtained by solving the optimization problem as such with the SciPy library is also displayed, together with optimal point supposing no active currents can be injected. The latter tries to emulate the outcome of following the grid codes to improve voltages. Note that grid codes also consider the injection of active power, but this serves the purpose of maintaining a stable system in terms of frequency, something not contemplated in this analysis. 

For this and all the upcoming faults we have set the values shown in Table \ref{tab:val}. 

\begin{table}[!htb] \centering
    \begin{tabular}{cc}
       \hline
       Magnitude & Value (pu) \\
       \hline
       $\underline{Z}_f$ & $0.00 + 0.10j$ \\ 
       $\underline{Z}_a$ & $0.01 + 0.10j$ \\
       $\underline{Z}_{th}$ & $0.01 + 0.05j$ \\
       $I_{max}$ & $1.00$ \\
       $\lambda^+$ & $1.00$ \\
       $\lambda^-$ & $1.00$ \\
       $|\underline{V}^+_{th}|$ & $1.00$ \\
       $|\underline{V}^-_{th}|$ & $0.00$ \\
       \hline
    \end{tabular}
    \caption{Values for the system under study}
    \label{tab:val}
\end{table}
We have considered the impedances to be mainly inductive. However, adding a resistive part not only makes them more realistic, but it also may shed some light on if the truly optimal decision is to inject only reactive currents. One could anticipate that the larger the resistive part becomes, the greater the active current should be to cause a considerable voltage drop (positive or negative) so that the voltage is improved. Thus, for the given impedances in Table \ref{tab:val}, we foresee the fact that reactive currents will turn out to be substantially larger than active currents. For now, the same weighting is arbitrarily attributed to the positive sequence minimization subfunction as well as to the negative sequence one. 
% here display the plots for lambda1 and lambda2 = 1, but in the annex we can show the graphs for lambda1 = 0 and lambda2 = 1; and lambda1 = 1 and lambda2 = 0.

\subsection{Balanced fault}
Balanced faults are commonly referred to as the most severe type of fault, as currents take the largest values \cite{kothari2003modern}. Its representation in symmetrical components indicates a decoupling between sequences. As a result of that, a considerable positive sequence current should be injected to increase the positive sequence voltage. It should be mostly inductive. On the other hand, no negative sequence current has to be injected to keep a null negative sequence voltage (assuming that the grid presents no negative sequence voltage, as it is the case here). There are no homopolar currents in this and all the subsequent faults. 

Figure \ref{fig:3x1} shows on the one hand the results with the brute force methodology. Each of the four currents ($\underline{I}^{+}_{re}$, $\underline{I}^{+}_{im}$, $\underline{I}^{-}_{re}$, $\underline{I}^{-}_{im}$) takes 20 different values, so there would be a total of $20^4$ points in the plot. However, some points are suppressed because they exceed the current limitations. It is clear that the minimum is achieved when the negative sequence currents, both real and imaginary, tend to zero. The imaginary positive sequence current takes extreme negative values - close to $I_{max}$ while the real part remains small, yet it is not negligible. 

% On the other hand, the point corresponding to the solution of the optimization problem illustrates what was already deduced from the brute force computations. Injecting a negative imaginary positive sequence current causes a positive voltage drop so that the voltage we wish to improve can be far apart from the faulted one. Taking into account that in this case $X>>R$, the real positive sequence current becomes considerably smaller than the imaginary part. 

% graph of brute force for the 4 currents: I1re, I1im, I2re, I2im. At the same plot show the point obtained with the optimization and the one we would get with the grid codes (only reactive current, but positive or negative??) Indicate that the objective function is a bit better with my optimization than following the grid code. 
% prepare plots template first
% plot the point where only reactive current and 0 active current. Set the constraint in the code. 

\begin{figure}\centering \footnotesize
  \begin{subfigure}[!htb]{.4\textwidth}
    \centering
        \begin{tikzpicture}[trim axis right,trim axis left]
            \pgfplotsset{width=7cm, height=6cm}
            \begin{axis}[grid=major, xlabel={$\underline{I}^+_{re}$}, ylabel={$f$}, /pgf/number format/.cd, legend style={at={(0.98,0.15)},anchor=south east,legend columns=1, draw=none, inner sep=0pt,fill=gray!10}, xtick={-1,-0.5,...,1}, ytick={0.1,0.2,...,1}, scatter/classes={a={mark=o,draw=black, mark size=1pt}, b={mark=x,draw=red, mark size=2pt}, c={mark=square,draw=orange, mark size=1.5pt}},  scatter src=explicit symbolic, axis line style = very thick, legend style={at={(1.03,-0.03)},anchor=north west}]
            \addplot[thick, scatter, only marks, each nth point = 1] table[x=x, y=y, meta=label, col sep=comma] {Data/I1_re_3x.csv};
            \addplot[thick, scatter, only marks] table[x=x, y=y, meta=label, col sep=comma] {Data/I1_re_3x_2.csv};
            \addplot[thick, scatter, only marks] table[x=x, y=y, meta=label, col sep=comma] {Data/I1_re_3x_3.csv};
            % \legend{BF, OPT};
            \end{axis}
        \end{tikzpicture}
  \end{subfigure}
  \hspace{1cm}
\begin{subfigure}[!htb]{.4\textwidth}
    \centering
        \begin{tikzpicture}[trim axis right,trim axis left]
            \pgfplotsset{width=7cm, height=6cm}
            \begin{axis}[grid=major, xlabel={$\underline{I}^+_{im}$}, ylabel={$f$}, /pgf/number format/.cd, legend style={at={(0.98,0.15)},anchor=south east,legend columns=1, draw=none, inner sep=0pt,fill=gray!10},xtick={-1,-0.5,...,1}, ytick={0.1,0.2,...,1}, scatter/classes={a={mark=o,draw=black, mark size=1pt}, b={mark=x,draw=red, mark size=2pt}, c={mark=square,draw=orange, mark size=1.5pt}},  scatter src=explicit symbolic, axis line style = very thick, legend style={at={(0.97,0.03)},anchor=south east}]
            \addplot[thick, scatter, only marks, each nth point = 1] table[x=x, y=y, meta=label, col sep=comma] {Data/I1_im_3x.csv};
            \addplot[thick, scatter, only marks] table[x=x, y=y, meta=label, col sep=comma] {Data/I1_im_3x_2.csv};
            \addplot[thick, scatter, only marks] table[x=x, y=y, meta=label, col sep=comma] {Data/I1_im_3x_3.csv};
            \legend{BF, OPT, ROPT};
            \end{axis}
        \end{tikzpicture}
  \end{subfigure}
  \vspace{0.5cm}
\begin{subfigure}[!htb]{.4\textwidth}
    \centering
        \begin{tikzpicture}[trim axis right,trim axis left]
            \pgfplotsset{width=7cm, height=6cm}
            \begin{axis}[grid=major, xlabel={$\underline{I}^-_{re}$}, ylabel={$f$}, /pgf/number format/.cd, legend style={at={(0.98,0.15)},anchor=south east,legend columns=1, draw=none, inner sep=0pt,fill=gray!10}, xtick={-1,-0.5,...,1}, ytick={0.1,0.2,...,1}, scatter/classes={a={mark=o,draw=black, mark size=1pt}, b={mark=x,draw=red, mark size=2pt}, c={mark=square,draw=orange, mark size=1.5pt}},  scatter src=explicit symbolic, axis line style = very thick, legend style={at={(1.03,-0.03)},anchor=north west}]
            \addplot[thick, scatter, only marks, each nth point = 1] table[x=x, y=y, meta=label, col sep=comma] {Data/I2_re_3x.csv};
            \addplot[thick, scatter, only marks] table[x=x, y=y, meta=label, col sep=comma] {Data/I2_re_3x_3.csv};
            \addplot[thick, scatter, only marks] table[x=x, y=y, meta=label, col sep=comma] {Data/I2_re_3x_2.csv};
            \end{axis}
        \end{tikzpicture}
  \end{subfigure}
  \hspace{1cm}
\begin{subfigure}[!htb]{.4\textwidth}
    \centering
        \begin{tikzpicture}[trim axis right,trim axis left]
            \pgfplotsset{width=7cm, height=6cm}
            \begin{axis}[grid=major, xlabel={$\underline{I}^-_{im}$}, ylabel={$f$}, /pgf/number format/.cd, legend style={at={(0.98,0.15)},anchor=south east,legend columns=1, draw=none, inner sep=0pt,fill=gray!10}, xtick={-1,-0.5,...,1}, ytick={0.1,0.2,...,1}, scatter/classes={a={mark=o,draw=black, mark size=1pt}, b={mark=x,draw=red, mark size=2pt}, c={mark=square,draw=orange, mark size=1.5pt}},  scatter src=explicit symbolic, axis line style = very thick, legend style={at={(1.03,-0.03)},anchor=north west}]
            \addplot[thick, scatter, only marks, each nth point = 1] table[x=x, y=y, meta=label, col sep=comma] {Data/I2_im_3x.csv};
            \addplot[thick, scatter, only marks] table[x=x, y=y, meta=label, col sep=comma] {Data/I2_im_3x_3.csv};
            \addplot[thick, scatter, only marks] table[x=x, y=y, meta=label, col sep=comma] {Data/I2_im_3x_2.csv};
            \end{axis}
        \end{tikzpicture}
  \end{subfigure}
  \caption{Influence of the currents on the objective function for the balanced fault. BF: brute force, OPT: solution to the optimization problem, ROPT: solution to the optimization problem restricted to only injecting reactive power.}
  \label{fig:3x1}
\end{figure}
On the other hand, the point corresponding to the solution of the optimization problem illustrates what was already more or less deduced from the brute force computations. Nonetheless, solving the optimization problem yields a more favorable result. The objective function is slightly smaller and the optimal point can be located in a zone where not many points coming from the brute force are present. Such irregular distribution of points is due to the fact that around the optimal points many combinations of currents do not meet the constraints. In any case, injecting a negative imaginary positive sequence current causes a positive voltage drop so that the voltage we wish to improve can be far apart from the faulted one. Taking into account that in this case $X>>R$, the real positive sequence current becomes considerably smaller than the imaginary part. 


% \begin{figure}\centering
%   \begin{subfigure}[!htb]{.4\textwidth}
%     \centering
%         \begin{tikzpicture}[trim axis right,trim axis left]
%             \pgfplotsset{width=7cm, height=6cm}
%             \begin{axis}[grid=major, xlabel={$\underline{I}^+_{re}$}, ylabel={$f$}, /pgf/number format/.cd, legend style={at={(0.98,0.15)},anchor=south east,legend columns=1, draw=none, inner sep=0pt,fill=gray!10}, xtick={-1,-0.5,...,1}, ytick={0.1,0.2,...,1}]
%             \addplot[very thick, draw opacity = 0, each nth point=1, scatter, only marks] table[x=x, y=y, col sep=comma] {Data/I1_re_3x.csv};
%             \addplot[very thick, each nth point=1, scatter, only marks, mark=x,mark options={fill=red,scale=4}] table[x=x, y=y, col sep=comma] {Data/I1_re_3x_2.csv};
%             \end{axis}
%         \end{tikzpicture}
%   \end{subfigure}
%   \hspace{1cm}
% \begin{subfigure}[!htb]{.4\textwidth}
%     \centering
%         \begin{tikzpicture}[trim axis right,trim axis left]
%             \pgfplotsset{width=7cm, height=6cm}
%             \begin{axis}[grid=major, xlabel={$\underline{I}^+_{im}$}, ylabel={$f$}, /pgf/number format/.cd, legend style={at={(0.98,0.15)},anchor=south east,legend columns=1, draw=none, inner sep=0pt},xtick={-1,-0.5,...,1}, ytick={0.1,0.2,...,1}]
%             \addplot[very thick, draw opacity=0, each nth point=1, scatter, only marks] table[x=x, y=y, col sep=comma] {Data/I1_im_3x.csv};
%             \addplot[very thick, each nth point=1, scatter, only marks, mark=x,mark options={fill=red,scale=4}] table[x=x, y=y, col sep=comma] {Data/I1_im_3x_2.csv};
%             \end{axis}
%         \end{tikzpicture}
%   \end{subfigure}

%   \vspace{0.5cm}

% \begin{subfigure}[!htb]{.4\textwidth}
%     \centering
%         \begin{tikzpicture}[trim axis right,trim axis left]
%             \pgfplotsset{width=7cm, height=6cm}
%             \begin{axis}[grid=major, xlabel={$\underline{I}^-_{re}$}, ylabel={$f$}, /pgf/number format/.cd, legend style={at={(0.98,0.15)},anchor=south east,legend columns=1, draw=none, inner sep=0pt,fill=gray!10}, xtick={-1,-0.5,...,1}, ytick={0.1,0.2,...,1}]
%             \addplot[very thick, draw opacity=0, each nth point=1, scatter, only marks] table[x=x, y=y, col sep=comma] {Data/I2_re_3x.csv};
%             \addplot[very thick, each nth point=1, scatter, only marks, mark=x,mark options={fill=red,scale=4}] table[x=x, y=y, col sep=comma] {Data/I2_re_3x_2.csv};
%             \end{axis}
%         \end{tikzpicture}
%   \end{subfigure}
%   \hspace{1cm}
% \begin{subfigure}[!htb]{.4\textwidth}
%     \centering
%         \begin{tikzpicture}[trim axis right,trim axis left]
%             \pgfplotsset{width=7cm, height=6cm}
%             \begin{axis}[grid=major, xlabel={$\underline{I}^-_{im}$}, ylabel={$f$}, /pgf/number format/.cd, legend style={at={(0.98,0.15)},anchor=south east,legend columns=1, draw=none, inner sep=0pt,fill=gray!10}, xtick={-1,-0.5,...,1}, ytick={0.1,0.2,...,1}]
%             \addplot[very thick, draw opacity=0, each nth point=1, scatter, only marks] table[x=x, y=y, col sep=comma] {Data/I2_im_3x.csv};
%             \addplot[very thick, each nth point=1, scatter, only marks, mark=x,mark options={fill=red,scale=4}, red] table[x=x, y=y, col sep=comma] {Data/I2_im_3x_2.csv};
%             \end{axis}
%         \end{tikzpicture}
%   \end{subfigure}
%   \caption{Objective function depending on injected currents for the balanced case}
%   \label{fig:3x1}
% \end{figure}

\subsection{Line to ground fault}

\begin{figure}\centering \footnotesize
  \begin{subfigure}[!htb]{.4\textwidth}
    \centering
        \begin{tikzpicture}[trim axis right,trim axis left]
            \pgfplotsset{width=7cm, height=6cm}
            \begin{axis}[grid=major, xlabel={$\underline{I}^+_{re}$}, ylabel={$f$}, /pgf/number format/.cd, legend style={at={(0.98,0.15)},anchor=south east,legend columns=1, draw=none, inner sep=0pt,fill=gray!10}, xtick={-1,-0.5,...,1}, ytick={0.1,0.2,...,1}, scatter/classes={a={mark=o,draw=black, mark size=1pt}, b={mark=x,draw=red, mark size=2pt}, c={mark=square,draw=orange, mark size=1.5pt}},  scatter src=explicit symbolic, axis line style = very thick, legend style={at={(1.03,-0.03)},anchor=north west}]
            \addplot[thick, scatter, only marks, each nth point = 1] table[x=x, y=y, meta=label, col sep=comma] {Data/I1_re_LG.csv};
            \addplot[thick, scatter, only marks] table[x=x, y=y, meta=label, col sep=comma] {Data/I1_re_LG_2.csv};
            \addplot[thick, scatter, only marks] table[x=x, y=y, meta=label, col sep=comma] {Data/I1_re_LG_3.csv};
            % \legend{BF, OPT};
            \end{axis}
        \end{tikzpicture}
  \end{subfigure}
  \hspace{1cm}
\begin{subfigure}[!htb]{.4\textwidth}
    \centering
        \begin{tikzpicture}[trim axis right,trim axis left]
            \pgfplotsset{width=7cm, height=6cm}
            \begin{axis}[grid=major, xlabel={$\underline{I}^+_{im}$}, ylabel={$f$}, /pgf/number format/.cd, legend style={at={(0.98,0.15)},anchor=south east,legend columns=1, draw=none, inner sep=0pt,fill=gray!10},xtick={-1,-0.5,...,1}, ytick={0.1,0.2,...,1}, scatter/classes={a={mark=o,draw=black, mark size=1pt}, b={mark=x,draw=red, mark size=2pt}, c={mark=square,draw=orange, mark size=1.5pt}},  scatter src=explicit symbolic, axis line style = very thick, legend style={at={(0.97,0.03)},anchor=south east}]
            \addplot[thick, scatter, only marks, each nth point = 1] table[x=x, y=y, meta=label, col sep=comma] {Data/I1_im_LG.csv};
            \addplot[thick, scatter, only marks] table[x=x, y=y, meta=label, col sep=comma] {Data/I1_im_LG_2.csv};
            \addplot[thick, scatter, only marks] table[x=x, y=y, meta=label, col sep=comma] {Data/I1_im_LG_3.csv};
            \legend{BF, OPT, ROPT};
            \end{axis}
        \end{tikzpicture}
  \end{subfigure}
  \vspace{0.5cm}
\begin{subfigure}[!htb]{.4\textwidth}
    \centering
        \begin{tikzpicture}[trim axis right,trim axis left]
            \pgfplotsset{width=7cm, height=6cm}
            \begin{axis}[grid=major, xlabel={$\underline{I}^-_{re}$}, ylabel={$f$}, /pgf/number format/.cd, legend style={at={(0.98,0.15)},anchor=south east,legend columns=1, draw=none, inner sep=0pt,fill=gray!10}, xtick={-1,-0.5,...,1}, ytick={0.1,0.2,...,1}, scatter/classes={a={mark=o,draw=black, mark size=1pt}, b={mark=x,draw=red, mark size=2pt}, c={mark=square,draw=orange, mark size=1.5pt}},  scatter src=explicit symbolic, axis line style = very thick, legend style={at={(1.03,-0.03)},anchor=north west}]
            \addplot[thick, scatter, only marks, each nth point = 1] table[x=x, y=y, meta=label, col sep=comma] {Data/I2_re_LG.csv};
            \addplot[thick, scatter, only marks] table[x=x, y=y, meta=label, col sep=comma] {Data/I2_re_LG_3.csv};
            \addplot[thick, scatter, only marks] table[x=x, y=y, meta=label, col sep=comma] {Data/I2_re_LG_2.csv};
            \end{axis}
        \end{tikzpicture}
  \end{subfigure}
  \hspace{1cm}
\begin{subfigure}[!htb]{.4\textwidth}
    \centering
        \begin{tikzpicture}[trim axis right,trim axis left]
            \pgfplotsset{width=7cm, height=6cm}
            \begin{axis}[grid=major, xlabel={$\underline{I}^-_{im}$}, ylabel={$f$}, /pgf/number format/.cd, legend style={at={(0.98,0.15)},anchor=south east,legend columns=1, draw=none, inner sep=0pt,fill=gray!10}, xtick={-1,-0.5,...,1}, ytick={0.1,0.2,...,1}, scatter/classes={a={mark=o,draw=black, mark size=1pt}, b={mark=x,draw=red, mark size=2pt}, c={mark=square,draw=orange, mark size=1.5pt}},  scatter src=explicit symbolic, axis line style = very thick, legend style={at={(1.03,-0.03)},anchor=north west}]
            \addplot[thick, scatter, only marks, each nth point = 1] table[x=x, y=y, meta=label, col sep=comma] {Data/I2_im_LG.csv};
            \addplot[thick, scatter, only marks] table[x=x, y=y, meta=label, col sep=comma] {Data/I2_im_LG_3.csv};
            \addplot[thick, scatter, only marks] table[x=x, y=y, meta=label, col sep=comma] {Data/I2_im_LG_2.csv};
            \end{axis}
        \end{tikzpicture}
  \end{subfigure}
  \caption{Influence of the currents on the objective function for the line to ground fault. BF: brute force, OPT: solution to the optimization problem, ROPT: solution to the optimization problem restricted to only injecting reactive power.}
  \label{fig:3x1}
\end{figure}



\subsection{Line to line fault}

\begin{figure}\centering \footnotesize
  \begin{subfigure}[!htb]{.4\textwidth}
    \centering
        \begin{tikzpicture}[trim axis right,trim axis left]
            \pgfplotsset{width=7cm, height=6cm}
            \begin{axis}[grid=major, xlabel={$\underline{I}^+_{re}$}, ylabel={$f$}, /pgf/number format/.cd, legend style={at={(0.98,0.15)},anchor=south east,legend columns=1, draw=none, inner sep=0pt,fill=gray!10}, xtick={-1,-0.5,...,1}, ytick={0.1,0.2,...,1}, scatter/classes={a={mark=o,draw=black, mark size=1pt}, b={mark=x,draw=red, mark size=2pt}, c={mark=square,draw=orange, mark size=1.5pt}},  scatter src=explicit symbolic, axis line style = very thick, legend style={at={(1.03,-0.03)},anchor=north west}]
            \addplot[thick, scatter, only marks, each nth point = 1] table[x=x, y=y, meta=label, col sep=comma] {Data/I1_re_LL.csv};
            \addplot[thick, scatter, only marks] table[x=x, y=y, meta=label, col sep=comma] {Data/I1_re_LL_2.csv};
            \addplot[thick, scatter, only marks] table[x=x, y=y, meta=label, col sep=comma] {Data/I1_re_LL_3.csv};
            % \legend{BF, OPT};
            \end{axis}
        \end{tikzpicture}
  \end{subfigure}
  \hspace{1cm}
\begin{subfigure}[!htb]{.4\textwidth}
    \centering
        \begin{tikzpicture}[trim axis right,trim axis left]
            \pgfplotsset{width=7cm, height=6cm}
            \begin{axis}[grid=major, xlabel={$\underline{I}^+_{im}$}, ylabel={$f$}, /pgf/number format/.cd, legend style={at={(0.98,0.15)},anchor=south east,legend columns=1, draw=none, inner sep=0pt,fill=gray!10},xtick={-1,-0.5,...,1}, ytick={0.1,0.2,...,1}, scatter/classes={a={mark=o,draw=black, mark size=1pt}, b={mark=x,draw=red, mark size=2pt}, c={mark=square,draw=orange, mark size=1.5pt}},  scatter src=explicit symbolic, axis line style = very thick, legend style={at={(0.97,0.03)},anchor=south east}]
            \addplot[thick, scatter, only marks, each nth point = 1] table[x=x, y=y, meta=label, col sep=comma] {Data/I1_im_LL.csv};
            \addplot[thick, scatter, only marks] table[x=x, y=y, meta=label, col sep=comma] {Data/I1_im_LL_2.csv};
            \addplot[thick, scatter, only marks] table[x=x, y=y, meta=label, col sep=comma] {Data/I1_im_LL_3.csv};
            \legend{BF, OPT, ROPT};
            \end{axis}
        \end{tikzpicture}
  \end{subfigure}
  \vspace{0.5cm}
\begin{subfigure}[!htb]{.4\textwidth}
    \centering
        \begin{tikzpicture}[trim axis right,trim axis left]
            \pgfplotsset{width=7cm, height=6cm}
            \begin{axis}[grid=major, xlabel={$\underline{I}^-_{re}$}, ylabel={$f$}, /pgf/number format/.cd, legend style={at={(0.98,0.15)},anchor=south east,legend columns=1, draw=none, inner sep=0pt,fill=gray!10}, xtick={-1,-0.5,...,1}, ytick={0.1,0.2,...,1}, scatter/classes={a={mark=o,draw=black, mark size=1pt}, b={mark=x,draw=red, mark size=2pt}, c={mark=square,draw=orange, mark size=1.5pt}},  scatter src=explicit symbolic, axis line style = very thick, legend style={at={(1.03,-0.03)},anchor=north west}]
            \addplot[thick, scatter, only marks, each nth point = 1] table[x=x, y=y, meta=label, col sep=comma] {Data/I2_re_LL.csv};
            \addplot[thick, scatter, only marks] table[x=x, y=y, meta=label, col sep=comma] {Data/I2_re_LL_3.csv};
            \addplot[thick, scatter, only marks] table[x=x, y=y, meta=label, col sep=comma] {Data/I2_re_LL_2.csv};
            \end{axis}
        \end{tikzpicture}
  \end{subfigure}
  \hspace{1cm}
\begin{subfigure}[!htb]{.4\textwidth}
    \centering
        \begin{tikzpicture}[trim axis right,trim axis left]
            \pgfplotsset{width=7cm, height=6cm}
            \begin{axis}[grid=major, xlabel={$\underline{I}^-_{im}$}, ylabel={$f$}, /pgf/number format/.cd, legend style={at={(0.98,0.15)},anchor=south east,legend columns=1, draw=none, inner sep=0pt,fill=gray!10}, xtick={-1,-0.5,...,1}, ytick={0.1,0.2,...,1}, scatter/classes={a={mark=o,draw=black, mark size=1pt}, b={mark=x,draw=red, mark size=2pt}, c={mark=square,draw=orange, mark size=1.5pt}},  scatter src=explicit symbolic, axis line style = very thick, legend style={at={(1.03,-0.03)},anchor=north west}]
            \addplot[thick, scatter, only marks, each nth point = 1] table[x=x, y=y, meta=label, col sep=comma] {Data/I2_im_LL.csv};
            \addplot[thick, scatter, only marks] table[x=x, y=y, meta=label, col sep=comma] {Data/I2_im_LL_3.csv};
            \addplot[thick, scatter, only marks] table[x=x, y=y, meta=label, col sep=comma] {Data/I2_im_LL_2.csv};
            \end{axis}
        \end{tikzpicture}
  \end{subfigure}
\end{figure}


\subsection{Double line to ground fault}

\begin{figure}\centering \footnotesize
  \begin{subfigure}[!htb]{.4\textwidth}
    \centering
        \begin{tikzpicture}[trim axis right,trim axis left]
            \pgfplotsset{width=7cm, height=6cm}
            \begin{axis}[grid=major, xlabel={$\underline{I}^+_{re}$}, ylabel={$f$}, /pgf/number format/.cd, legend style={at={(0.98,0.15)},anchor=south east,legend columns=1, draw=none, inner sep=0pt,fill=gray!10}, xtick={-1,-0.5,...,1}, ytick={0.1,0.2,...,1}, scatter/classes={a={mark=o,draw=black, mark size=1pt}, b={mark=x,draw=red, mark size=2pt}, c={mark=square,draw=orange, mark size=1.5pt}},  scatter src=explicit symbolic, axis line style = very thick, legend style={at={(1.03,-0.03)},anchor=north west}]
            \addplot[thick, scatter, only marks, each nth point = 1] table[x=x, y=y, meta=label, col sep=comma] {Data/I1_re_LLG.csv};
            \addplot[thick, scatter, only marks] table[x=x, y=y, meta=label, col sep=comma] {Data/I1_re_LLG_2.csv};
            \addplot[thick, scatter, only marks] table[x=x, y=y, meta=label, col sep=comma] {Data/I1_re_LLG_3.csv};
            % \legend{BF, OPT};
            \end{axis}
        \end{tikzpicture}
  \end{subfigure}
  \hspace{1cm}
\begin{subfigure}[!htb]{.4\textwidth}
    \centering
        \begin{tikzpicture}[trim axis right,trim axis left]
            \pgfplotsset{width=7cm, height=6cm}
            \begin{axis}[grid=major, xlabel={$\underline{I}^+_{im}$}, ylabel={$f$}, /pgf/number format/.cd, legend style={at={(0.98,0.15)},anchor=south east,legend columns=1, draw=none, inner sep=0pt,fill=gray!10},xtick={-1,-0.5,...,1}, ytick={0.1,0.2,...,1}, scatter/classes={a={mark=o,draw=black, mark size=1pt}, b={mark=x,draw=red, mark size=2pt}, c={mark=square,draw=orange, mark size=1.5pt}},  scatter src=explicit symbolic, axis line style = very thick, legend style={at={(0.97,0.03)},anchor=south east}]
            \addplot[thick, scatter, only marks, each nth point = 1] table[x=x, y=y, meta=label, col sep=comma] {Data/I1_im_LLG.csv};
            \addplot[thick, scatter, only marks] table[x=x, y=y, meta=label, col sep=comma] {Data/I1_im_LLG_2.csv};
            \addplot[thick, scatter, only marks] table[x=x, y=y, meta=label, col sep=comma] {Data/I1_im_LLG_3.csv};
            \legend{BF, OPT, ROPT};
            \end{axis}
        \end{tikzpicture}
  \end{subfigure}
  \vspace{0.5cm}
\begin{subfigure}[!htb]{.4\textwidth}
    \centering
        \begin{tikzpicture}[trim axis right,trim axis left]
            \pgfplotsset{width=7cm, height=6cm}
            \begin{axis}[grid=major, xlabel={$\underline{I}^-_{re}$}, ylabel={$f$}, /pgf/number format/.cd, legend style={at={(0.98,0.15)},anchor=south east,legend columns=1, draw=none, inner sep=0pt,fill=gray!10}, xtick={-1,-0.5,...,1}, ytick={0.1,0.2,...,1}, scatter/classes={a={mark=o,draw=black, mark size=1pt}, b={mark=x,draw=red, mark size=2pt}, c={mark=square,draw=orange, mark size=1.5pt}},  scatter src=explicit symbolic, axis line style = very thick, legend style={at={(1.03,-0.03)},anchor=north west}]
            \addplot[thick, scatter, only marks, each nth point = 1] table[x=x, y=y, meta=label, col sep=comma] {Data/I2_re_LLG.csv};
            \addplot[thick, scatter, only marks] table[x=x, y=y, meta=label, col sep=comma] {Data/I2_re_LLG_3.csv};
            \addplot[thick, scatter, only marks] table[x=x, y=y, meta=label, col sep=comma] {Data/I2_re_LLG_2.csv};
            \end{axis}
        \end{tikzpicture}
  \end{subfigure}
  \hspace{1cm}
\begin{subfigure}[!htb]{.4\textwidth}
    \centering
        \begin{tikzpicture}[trim axis right,trim axis left]
            \pgfplotsset{width=7cm, height=6cm}
            \begin{axis}[grid=major, xlabel={$\underline{I}^-_{im}$}, ylabel={$f$}, /pgf/number format/.cd, legend style={at={(0.98,0.15)},anchor=south east,legend columns=1, draw=none, inner sep=0pt,fill=gray!10}, xtick={-1,-0.5,...,1}, ytick={0.1,0.2,...,1}, scatter/classes={a={mark=o,draw=black, mark size=1pt}, b={mark=x,draw=red, mark size=2pt}, c={mark=square,draw=orange, mark size=1.5pt}},  scatter src=explicit symbolic, axis line style = very thick, legend style={at={(1.03,-0.03)},anchor=north west}]
            \addplot[thick, scatter, only marks, each nth point = 1] table[x=x, y=y, meta=label, col sep=comma] {Data/I2_im_LLG.csv};
            \addplot[thick, scatter, only marks] table[x=x, y=y, meta=label, col sep=comma] {Data/I2_im_LLG_3.csv};
            \addplot[thick, scatter, only marks] table[x=x, y=y, meta=label, col sep=comma] {Data/I2_im_LLG_2.csv};
            \end{axis}
        \end{tikzpicture}
  \end{subfigure}
\end{figure}





\newpage
\appendix
\section{$abc$ circuits}

\begin{figure}[!htb] \centering
\begin{circuitikz}[european]
\thicklines

\draw (0,0) to [american controlled current source,  *-] (2,0);
\draw (0,-2) to [american controlled current source, *-] (2,-2);
\draw (0,-4) to [american controlled current source, *-] (2,-4);
\draw (2,0) to [R, l=$\underline{Z}_a$, -] (4,0);
\draw (2,-2) to [R, l=$\underline{Z}_a$, -] (4,-2);
\draw (2,-4) to [R, l=$\underline{Z}_a$, -] (4,-4);
\draw (4,0) to [short] (6,0);
\draw (4,-2) to [short] (6,-2);
\draw (4,-4) to [short] (6,-4);
\draw (0,0) to [short] (0,-4);

\draw (6,0) to [R, l=$\underline{Z}_{th}$, -] (8,0);
\draw (6,-2) to [R, l=$\underline{Z}_{th}$, -] (8,-2);
\draw (6,-4) to [R, l=$\underline{Z}_{th}$, -] (8,-4);
\draw (10,0) to [sV, v_=$\underline{V}^{a}_{th}$, *-] (8,0);
\draw (10,-2) to [sV, v_=$\underline{V}^{b}_{th}$, *-] (8,-2);
\draw (10,-4) to [sV, v_=$\underline{V}^{c}_{th}$, *-] (8,-4);
\draw (10, 0) to [short] (10, -4);
\draw (10,-2) to [short] (11.5,-2);
\draw (11.5,-2) to [short] (11.5, -4.5);
\draw (11.5,-4.5) -- (11.5,-5.00) node[ground]{};


\draw (1.5,-0.4) node[]{$\underline{I}^a$};
\draw (2,0.3) to [open,v=$\underline{V}^a_c$] (0,0.3);
\draw (1.5,-2.4) node[]{$\underline{I}^b$};
\draw (2,-1.7) to [open,v=$\underline{V}^b_c$] (0,-1.7);
\draw (1.5,-4.4) node[]{$\underline{I}^c$};
\draw (2,-3.7) to [open,v=$\underline{V}^c_c$] (0,-3.7);

\draw (4.0,0) to [short, *-] (4.0,-4);
\draw (5.0,-2) to [short, *-] (5.0,-4);

\draw (4.0,-4) to [R, l=$\underline{Z}_f$, -*] (4.0,-6);
\draw (5.0,-4) to [R, l=$\underline{Z}_f$, -*] (5.0,-6);
\draw (6.0,-4) to [R, l=$\underline{Z}_f$, *-*] (6.0,-6);
\draw (4,-6) to [short] (6,-6);
 
\end{circuitikz}
\caption{Balanced fault schematic}
\label{fig:3_trif}
\end{figure}


\begin{figure}[!htb] \centering
\begin{circuitikz}[european]
\thicklines

\draw (0,0) to [american controlled current source,  *-] (2,0);
\draw (0,-2) to [american controlled current source, *-] (2,-2);
\draw (0,-4) to [american controlled current source, *-] (2,-4);
\draw (2,0) to [R, l=$\underline{Z}_a$, -] (4,0);
\draw (2,-2) to [R, l=$\underline{Z}_a$, -] (4,-2);
\draw (2,-4) to [R, l=$\underline{Z}_a$, -] (4,-4);
\draw (4,0) to [short] (6,0);
\draw (4,-2) to [short] (6,-2);
\draw (4,-4) to [short] (6,-4);
\draw (0,0) to [short] (0,-4);

\draw (6,0) to [R, l=$\underline{Z}_{th}$, -] (8,0);
\draw (6,-2) to [R, l=$\underline{Z}_{th}$, -] (8,-2);
\draw (6,-4) to [R, l=$\underline{Z}_{th}$, -] (8,-4);
\draw (10,0) to [sV, v_=$\underline{V}^{a}_{th}$, *-] (8,0);
\draw (10,-2) to [sV, v_=$\underline{V}^{b}_{th}$, *-] (8,-2);
\draw (10,-4) to [sV, v_=$\underline{V}^{c}_{th}$, *-] (8,-4);
\draw (10, 0) to [short] (10, -4);
\draw (10,-2) to [short] (11.5,-2);
\draw (11.5,-2) to [short] (11.5, -4.5);
\draw (11.5,-4.5) -- (11.5,-5.00) node[ground]{};


\draw (1.5,-0.4) node[]{$\underline{I}^a$};
\draw (2,0.3) to [open,v=$\underline{V}^a_c$] (0,0.3);
\draw (1.5,-2.4) node[]{$\underline{I}^b$};
\draw (2,-1.7) to [open,v=$\underline{V}^b_c$] (0,-1.7);
\draw (1.5,-4.4) node[]{$\underline{I}^c$};
\draw (2,-3.7) to [open,v=$\underline{V}^c_c$] (0,-3.7);

\draw (5.0, 0) to [short, *-] (5.0, -4);
\draw (5.0,-4) to [R, l=$\underline{Z}_f$] (5.0,-5.5);
\draw (5.0,-5.5) -- (5.0,-5.5) node[ground]{};
 
\end{circuitikz}
\caption{Line to ground fault schematic}
\label{fig:3_lg}
\end{figure}


\begin{figure}[!htb] \centering
\begin{circuitikz}[european]
\thicklines

\draw (0,0) to [american controlled current source,  *-] (2,0);
\draw (0,-2) to [american controlled current source, *-] (2,-2);
\draw (0,-4) to [american controlled current source, *-] (2,-4);
\draw (2,0) to [R, l=$\underline{Z}_a$, -] (4,0);
\draw (2,-2) to [R, l=$\underline{Z}_a$, -] (4,-2);
\draw (2,-4) to [R, l=$\underline{Z}_a$, -] (4,-4);
\draw (4,0) to [short] (6,0);
\draw (4,-2) to [short] (6,-2);
\draw (4,-4) to [short] (6,-4);
\draw (0,0) to [short] (0,-4);

\draw (6,0) to [R, l=$\underline{Z}_{th}$, -] (8,0);
\draw (6,-2) to [R, l=$\underline{Z}_{th}$, -] (8,-2);
\draw (6,-4) to [R, l=$\underline{Z}_{th}$, -] (8,-4);
\draw (10,0) to [sV, v_=$\underline{V}^{a}_{th}$, *-] (8,0);
\draw (10,-2) to [sV, v_=$\underline{V}^{b}_{th}$, *-] (8,-2);
\draw (10,-4) to [sV, v_=$\underline{V}^{c}_{th}$, *-] (8,-4);
\draw (10, 0) to [short] (10, -4);
\draw (10,-2) to [short] (11.5,-2);
\draw (11.5,-2) to [short] (11.5, -4.5);
\draw (11.5,-4.5) -- (11.5,-5.00) node[ground]{};


\draw (1.5,-0.4) node[]{$\underline{I}^a$};
\draw (2,0.3) to [open,v=$\underline{V}^a_c$] (0,0.3);
\draw (1.5,-2.4) node[]{$\underline{I}^b$};
\draw (2,-1.7) to [open,v=$\underline{V}^b_c$] (0,-1.7);
\draw (1.5,-4.4) node[]{$\underline{I}^c$};
\draw (2,-3.7) to [open,v=$\underline{V}^c_c$] (0,-3.7);

\draw (4.0,-2) to [short, *-] (4.0, -5.0);
\draw (6.0,-4) to [short, *-] (6.0, -5.0);
\draw (4,-5.0) to [R, l=$\underline{Z}_f$] (6, -5.0);

 
\end{circuitikz}
\caption{Line to line fault schematic}
\label{fig:3_ll}
\end{figure}


\begin{figure}[!htb] \centering
\begin{circuitikz}[european]
\thicklines

\draw (0,0) to [american controlled current source,  *-] (2,0);
\draw (0,-2) to [american controlled current source, *-] (2,-2);
\draw (0,-4) to [american controlled current source, *-] (2,-4);
\draw (2,0) to [R, l=$\underline{Z}_a$, -] (4,0);
\draw (2,-2) to [R, l=$\underline{Z}_a$, -] (4,-2);
\draw (2,-4) to [R, l=$\underline{Z}_a$, -] (4,-4);
\draw (4,0) to [short] (6,0);
\draw (4,-2) to [short] (6,-2);
\draw (4,-4) to [short] (6,-4);
\draw (0,0) to [short] (0,-4);

\draw (6,0) to [R, l=$\underline{Z}_{th}$, -] (8,0);
\draw (6,-2) to [R, l=$\underline{Z}_{th}$, -] (8,-2);
\draw (6,-4) to [R, l=$\underline{Z}_{th}$, -] (8,-4);
\draw (10,0) to [sV, v_=$\underline{V}^{a}_{th}$, *-] (8,0);
\draw (10,-2) to [sV, v_=$\underline{V}^{b}_{th}$, *-] (8,-2);
\draw (10,-4) to [sV, v_=$\underline{V}^{c}_{th}$, *-] (8,-4);
\draw (10, 0) to [short] (10, -4);
\draw (10,-2) to [short] (11.5,-2);
\draw (11.5,-2) to [short] (11.5, -4.5);
\draw (11.5,-4.5) -- (11.5,-5.00) node[ground]{};


\draw (1.5,-0.4) node[]{$\underline{I}^a$};
\draw (2,0.3) to [open,v=$\underline{V}^a_c$] (0,0.3);
\draw (1.5,-2.4) node[]{$\underline{I}^b$};
\draw (2,-1.7) to [open,v=$\underline{V}^b_c$] (0,-1.7);
\draw (1.5,-4.4) node[]{$\underline{I}^c$};
\draw (2,-3.7) to [open,v=$\underline{V}^c_c$] (0,-3.7);

\draw (4.0,-2) to [short, *-] (4.0, -5.0);
\draw (6.0,-4) to [short, *-] (6.0, -5.0);
\draw (4,-5.0) to [short, -*] (5, -5.0);
\draw (5,-5.0) to [short, ] (6, -5.0);
\draw (5,-5) to [R, l=$\underline{Z}_f$] (5,-6.5);
\draw (5,-6.5) -- (5,-6.5) node[ground]{};

 
\end{circuitikz}
\caption{Double line to ground fault schematic}
\label{fig:3_llg}
\end{figure}

\clearpage
\newpage
\section{$+-0$ schemes}

\begin{figure}[!htb] \centering
\begin{circuitikz}[european]
\thicklines

\draw (0,6) to [sV, v_=$\underline{V}_{th}^+$] (0,8);
\draw (-3,8) to [R, l=$\underline{Z}_{th}$] (0,8);
\draw (-6,6) to [short] (0.0,6);
\draw (-6,8) to [R, l=$\underline{Z}_a$] (-3,8);
\draw (-6,6.5) to [american controlled current source, l_=$I^+$] (-6,7.5);
\draw (-6,6) to [short] (-6,6.5);
\draw (-6,7.5) to [short] (-6,8);
\draw (-6.3,8) to [open,v=$\underline{V}^+_c$] (-6.3,6);

\draw (0,3) to [short] (0,3.5);
\draw (0,3.5) to [short, *-*] (0,4.5);
\draw (0,4.5) to [short] (0,5);
\draw (-3,5) to [R, l=$\underline{Z}_{th}$] (0,5);
\draw (-6,3) to [short] (0.0,3);
\draw (-6,5) to [R, l=$\underline{Z}_a$] (-3,5);
\draw (-6,3.5) to [american controlled current source, l_=$I^-$] (-6,4.5);
\draw (-6,3) to [short] (-6,3.5);
\draw (-6,4.5) to [short] (-6,5);
\draw (-6.3,5) to [open,v=$\underline{V}^-_c$] (-6.3,3);

\draw (0,0) to [short] (0,0.5);
\draw (0,0.5) to [short, *-*] (0,1.5);
\draw (0,1.5) to [short] (0,2);
\draw (-3,2) to [R, l=$\underline{Z}_{th}$] (0,2);
\draw (-6,0) to [short] (0.0,0);
\draw (-6,2) to [R, l=$\underline{Z}_a$] (-3,2);
\draw (-6,0) to [short, -*] (-6,0.5);
\draw (-6,1.5) to [short, *-] (-6,2);
\draw (-6.3,2) to [open,v=$\underline{V}^0_c$] (-6.3,0);

\draw (-3,8) to [R, l=$\underline{Z}_{f}$] (-3,6);
\draw (-3,5) to [R, l=$\underline{Z}_{f}$] (-3,3);
\draw (-3,2) to [R, l=$\underline{Z}_{f}$] (-3,0);


\end{circuitikz}
\caption{Equivalent circuit for the balanced fault analysis}
\label{fig:sys_3x}
\end{figure}


\begin{figure}[!htb] \centering
\begin{circuitikz}[european]
\thicklines

\draw (0,6) to [sV, v_=$\underline{V}_{th}^+$] (0,8);
\draw (-3,8) to [R, l=$\underline{Z}_{th}$] (0,8);
\draw (-6,6) to [short] (0.0,6);
\draw (-6,8) to [R, l=$\underline{Z}_a$] (-3,8);
\draw (-6,6.5) to [american controlled current source, l_=$I^+$] (-6,7.5);
\draw (-6,6) to [short] (-6,6.5);
\draw (-6,7.5) to [short] (-6,8);
\draw (-6.3,8) to [open,v=$\underline{V}^+_c$] (-6.3,6);

\draw (0,3) to [short] (0,3.5);
\draw (0,3.5) to [short, *-*] (0,4.5);
\draw (0,4.5) to [short] (0,5);
\draw (-3,5) to [R, l=$\underline{Z}_{th}$] (0,5);
\draw (-6,3) to [short] (0.0,3);
\draw (-6,5) to [R, l=$\underline{Z}_a$] (-3,5);
\draw (-6,3.5) to [american controlled current source, l_=$I^-$] (-6,4.5);
\draw (-6,3) to [short] (-6,3.5);
\draw (-6,4.5) to [short] (-6,5);
\draw (-6.3,5) to [open,v=$\underline{V}^-_c$] (-6.3,3);

\draw (0,0) to [short] (0,0.5);
\draw (0,0.5) to [short, *-*] (0,1.5);
\draw (0,1.5) to [short] (0,2);
\draw (-3,2) to [R, l=$\underline{Z}_{th}$] (0,2);
\draw (-6,0) to [short] (0.0,0);
\draw (-6,2) to [R, l=$\underline{Z}_a$] (-3,2);
\draw (-6,0) to [short, -*] (-6,0.5);
\draw (-6,1.5) to [short, *-] (-6,2);
\draw (-6.3,2) to [open,v=$\underline{V}^0_c$] (-6.3,0);

\draw (-3,2) to [short, *-*] (-3,3);
\draw (-3,5) to [short, *-*] (-3,6);
\draw (-3,8) to [short, *-] (-3, 9);
\draw (-3,0) to [short, *-] (-3, -0.5);

\draw (-3,9) to [short] (2,9);
\draw (-3,-0.5) to [short] (2,-0.5);
\draw (2,9) to [R, l=$3\underline{Z}_f$] (2, -0.5);

\end{circuitikz}
\caption{Equivalent circuit for the line to ground fault analysis}
\label{fig:sys_LG}
\end{figure}

\begin{figure}[!htb] \centering
\begin{circuitikz}[european]
\thicklines

\draw (0,6) to [sV, v_=$\underline{V}_{th}^+$] (0,8);
\draw (-2.5,8) to [R, l=$\underline{Z}_{th}$] (0,8);
\draw (-5,6) to [short] (0.0,6);
\draw (-5,8) to [R, l=$\underline{Z}_a$] (-2.5,8);
\draw (-5,6.5) to [american controlled current source, l_=$I^+$] (-5,7.5);
\draw (-5,6) to [short] (-5,6.5);
\draw (-5,7.5) to [short] (-5,8);
\draw (-5.3,8) to [open,v=$\underline{V}^+_c$] (-5.3,6);

\draw (8,6) to [short] (8,6.5);
\draw (8,6.5) to [short, *-*] (8,7.5);
\draw (8,7.5) to [short] (8,8);
\draw (5.5,8) to [R, l=$\underline{Z}_{th}$] (8,8);
\draw (3,6) to [short] (8.0,6);
\draw (3,8) to [R, l=$\underline{Z}_a$] (5.5,8);
\draw (3,6.5) to [american controlled current source, l_=$I^-$] (3,7.5);
\draw (3,6) to [short] (3,6.5);
\draw (3,7.5) to [short] (3,8);
\draw (2.7,8) to [open,v=$\underline{V}^-_c$] (2.7,6);

\draw (-2.5, 8) to [short, *-] (-2.5, 9);
\draw (-2.5, 9) to [R, l=$\underline{Z}_{f}$] (5.5,9);
\draw (5.5,9) to [short, -*] (5.5, 8);

\draw (-2.5, 6) to [short, *-] (-2.5, 5.5);
\draw (-2.5, 5.5) to [short] (5.5, 5.5);
\draw (5.5, 5.5) to [short, -*] (5.5, 6);

\end{circuitikz}
\caption{Equivalent circuit for the line to line fault analysis}
\label{fig:sys_LL}
\end{figure}


\begin{figure}[!htb] \centering
\begin{circuitikz}[european]
\thicklines

\draw (0,8) to [sV, v_=$\underline{V}_{th}^+$] (0,10);
\draw (-3,10) to [R, l=$\underline{Z}_{th}$] (0,10);
\draw (-6,8) to [short] (0.0,8);
\draw (-6,10) to [R, l=$\underline{Z}_a$] (-3,10);
\draw (-6,8.5) to [american controlled current source, l_=$I^+$] (-6,9.5);
\draw (-6,8) to [short] (-6,8.5);
\draw (-6,9.5) to [short] (-6,10);
\draw (-6.3,10) to [open,v=$\underline{V}^+_c$] (-6.3,8);

\draw (0,4) to [short] (0,4.5);
\draw (0,4.5) to [short, *-*] (0,5.5);
\draw (0,5.5) to [short] (0,6);
\draw (-3,6) to [R, l=$\underline{Z}_{th}$] (0,6);
\draw (-6,4) to [short] (0.0,4);
\draw (-6,6) to [R, l=$\underline{Z}_a$] (-3,6);
\draw (-6,4.5) to [american controlled current source, l_=$I^-$] (-6,5.5);
\draw (-6,4) to [short] (-6,4.5);
\draw (-6,5.5) to [short] (-6,6);
\draw (-6.3,6) to [open,v=$\underline{V}^-_c$] (-6.3,4);

\draw (0,0) to [short] (0,0.5);
\draw (0,0.5) to [short, *-*] (0,1.5);
\draw (0,1.5) to [short] (0,2);
\draw (-3,2) to [R, l=$\underline{Z}_{th}$] (0,2);
\draw (-6,0) to [short] (0.0,0);
\draw (-6,2) to [R, l=$\underline{Z}_a$] (-3,2);
\draw (-6,0) to [short, -*] (-6,0.5);
\draw (-6,1.5) to [short, *-] (-6,2);
\draw (-6.3,2) to [open,v=$\underline{V}^0_c$] (-6.3,0);

\draw (-3,0) to [short, *-] (-3,-0.5);
\draw (-3,4) to [short, *-] (-3,3.5);
\draw (-3,8) to [short, *-] (-3,7.5);

\draw (-3,2) to [short, *-] (-3,3);
\draw (-3,6) to [short, *-] (-3,7);
\draw (-3,10) to [short, *-] (-3,11);

\draw (-3,3) to [short] (2,3);
\draw (-3,7) to [short] (2,7);
\draw (-3,11) to [short] (2,11);

\draw (-3,-0.5) to [short] (-8,-0.5);
\draw (-3,3.5) to [short] (-8,3.5);
\draw (-3,7.5) to [short] (-8,7.5);

\draw (-8,-0.5) to [short, -*] (-8, 3.5);
\draw (-8, 3.5) to [short] (-8, 7.5);

\draw (2,3) to [R, l_=$3\underline{Z}_f$, -*] (2,7);
\draw (2,11) to [short] (2,7);


\end{circuitikz}
\caption{Equivalent circuit for the double line to ground fault analysis}
\label{fig:sys_LLG}
\end{figure}



\clearpage
\newpage
\section{Expressions}

\subsection{Balanced fault}
\begin{equation}
    \begin{cases}
        \underline{V}^a_c = \dfrac{1}{\underline{Z}_f + \underline{Z}_{th}}[\underline{V}^a_{th}\underline{Z}_f + \underline{I}_a(\underline{Z}_a\underline{Z}_{th} + \underline{Z}_{th}\underline{Z}_f + \underline{Z}_f\underline{Z}_a)]\\
        \underline{V}^b_c = \dfrac{1}{\underline{Z}_f + \underline{Z}_{th}}[\underline{V}^b_{th}\underline{Z}_f + \underline{I}_b(\underline{Z}_a\underline{Z}_{th} + \underline{Z}_{th}\underline{Z}_f + \underline{Z}_f\underline{Z}_a)]\\
        \underline{V}^c_c = \dfrac{1}{\underline{Z}_f + \underline{Z}_{th}}[\underline{V}^c_{th}\underline{Z}_f + \underline{I}_c(\underline{Z}_a\underline{Z}_{th} + \underline{Z}_{th}\underline{Z}_f + \underline{Z}_f\underline{Z}_a)]
    \end{cases}
\end{equation}


\begin{equation}
    \begin{cases}
        \underline{V}^+_c= \dfrac{1}{\underline{Z}_f + \underline{Z}_{th}}[\underline{V}^+_{th}\underline{Z}_f + \underline{I}^+(\underline{Z}_a\underline{Z}_f + \underline{Z}_a\underline{Z}_{th} + \underline{Z}_f\underline{Z}_{th})] \\ 
 \underline{V}^-_c=\dfrac{1}{\underline{Z}_{f} + \underline{Z}_{th}}[\underline{I}^-(\underline{Z}_{th}\underline{Z}_f + \underline{Z}_a\underline{Z}_{th} + \underline{Z}_a\underline{Z}_f)] \\
 \underline{V}^0_c=0
    \end{cases}
\end{equation}


\subsection{Line to ground fault}
\begin{equation}
    \begin{cases}
        \underline{V}^a_c = \dfrac{1}{\underline{Z}_{th} + \underline{Z}_f}[\underline{I}_a(\underline{Z}_a\underline{Z}_{th} + \underline{Z}_a\underline{Z}_f + \underline{Z}_{th}\underline{Z}_f) + \underline{V}^a_{th}\underline{Z}_f] \\
        \underline{V}^b_c = \underline{V}^b_{th} + \underline{I}_b(\underline{Z}_a + \underline{Z}_{th}) \\
        \underline{V}^c_c =  \underline{V}^c_{th} + \underline{I}_c(\underline{Z}_a + \underline{Z}_{th})
    \end{cases}
\end{equation}

\begin{equation}
    \begin{cases}
        \underline{V}^+_c=\underline{I}^+(\underline{Z}_a + \underline{Z}_{th}) + \underline{V}^+_{th} - \dfrac{\underline{Z}_{th}}{3\underline{Z}_f + 3\underline{Z}_{th}}[\underline{I}^+\underline{Z}_{th} + \underline{I}^-\underline{Z}_{th} + \underline{V}^+_{th}]\\ 
        \underline{V}^-_c=\underline{I}^-(\underline{Z}_a + \underline{Z}_{th}) - \dfrac{\underline{Z}_{th}}{3\underline{Z}_f + 3\underline{Z}_{th}}[\underline{I}^+\underline{Z}_{th} + \underline{I}^-\underline{Z}_{th} + \underline{V}^+_{th}]\\
        \underline{V}^0_c=-\dfrac{\underline{Z}_{th}}{3\underline{Z}_f + 3\underline{Z}_{th}}[\underline{V}^+_{th} + \underline{I}^+\underline{Z}_{th} + \underline{I}^-\underline{Z}_{th}]
   \end{cases}
\end{equation}


\subsection{Line to line fault}
\begin{equation}
    \begin{cases}
        \underline{V}^a_c = \underline{I}_a (\underline{Z}_a + \underline{Z}_{th}) + \underline{V}^a_{th}  \\
        \underline{V}^b_c = \underline{I}_b\underline{Z}_a + \dfrac{1}{(\underline{Z}_{f} + \underline{Z}_{th}) (\underline{Z}_f + 2\underline{Z}_{th})}[\underline{I}_b (\underline{Z}_{th}\underline{Z}_f\underline{Z}_f + 2\underline{Z}_{th}\underline{Z}_{th}\underline{Z}_{f} + \underline{Z}_{th}\underline{Z}_{th}\underline{Z}_{th}) \\+ \underline{I}_c(\underline{Z}_{th}\underline{Z}_{th}\underline{Z}_f + \underline{Z}_{th}\underline{Z}_{th}\underline{Z}_{th}) + \underline{V}^b_{th} (\underline{Z}_f\underline{Z}_f + 2 \underline{Z}_f\underline{Z}_{th} + \underline{Z}_{th}\underline{Z}_{th}) + \underline{V}^c_{th}(\underline{Z}_{th}\underline{Z}_f + \underline{Z}_{th} \underline{Z}_{th})  ] \\
        \underline{V}^c_c = \underline{I}_c\underline{Z}_a + \dfrac{1}{\underline{Z}_f + 2\underline{Z}_{th}}[\underline{I}_c(\underline{Z}_{th}(\underline{Z}_f + \underline{Z}_{th})) + \underline{V}^c_{th} (\underline{Z}_f + \underline{Z}_{th}) + \underline{I}_b\underline{Z}_{th}\underline{Z}_{th} + \underline{V}^b_{th}\underline{Z}_{th}]
    \end{cases}
\end{equation}

\begin{equation}
    \begin{cases}
        \underline{V}^+_c= \underline{V}^+_{th} + \underline{I}^+(\underline{Z}_{a} + \underline{Z}_{th}) - \dfrac{\underline{Z}_{th}}{2\underline{Z}_{th}+\underline{Z}_f}[\underline{V}^+_{th} + \underline{I}^+\underline{Z}_{th}-\underline{I}^-\underline{Z}_{th}]  \\ 
        \underline{V}^-_c= \underline{V}^+_{th} + \underline{I}^+\underline{Z}_{th}+ \underline{I}^-\underline{Z}_a -\dfrac{\underline{Z}_{th}+\underline{Z}_f}{2\underline{Z}_{th}+\underline{Z}_f}[\underline{V}^+_{th} + \underline{I}^+\underline{Z}_{th} - \underline{I}^-\underline{Z}_{th}] \\
        \underline{V}^0_c=0
    \end{cases}
\end{equation} 


\subsection{Double line to ground fault}
\begin{equation}
    \begin{cases}
        \underline{V}^a_c = \underline{I}_a (\underline{Z}_a + \underline{Z}_{th}) + \underline{V}^a_{th}  \\
        \underline{V}^b_c = \underline{I}_b\underline{Z}_a + \dfrac{\underline{Z}_{th}\underline{Z}_f (\underline{I}_b+\underline{I}_c) + \underline{Z}_f(\underline{V}^b_{th} + \underline{V}^c_{th})}{2\underline{Z}_f + \underline{Z}_{th}} \\
        \underline{V}^c_c = \underline{I}_c\underline{Z}_a + \dfrac{\underline{Z}_{th}\underline{Z}_f (\underline{I}_b+\underline{I}_c) + \underline{Z}_f(\underline{V}^b_{th} + \underline{V}^c_{th})}{2\underline{Z}_f + \underline{Z}_{th}} 
    \end{cases}
\end{equation}

\begin{equation}
    \begin{cases}
        \underline{V}^+_c=\underline{I}^+\underline{Z}_a + \dfrac{\underline{Z}_{th} + 3\underline{Z}_f}{3\underline{Z}_{th} + 6\underline{Z}_f}[\underline{I}^+\underline{Z}_{th} + \underline{I}^-\underline{Z}_{th} + \underline{V}^+_{th}]    \\ 
        \underline{V}^-_c= \underline{I}^-\underline{Z}_a + \dfrac{\underline{Z}_{th} + 3\underline{Z}_f}{3\underline{Z}_{th} + 6\underline{Z}_f}[\underline{I}^+\underline{Z}_{th} + \underline{I}^-\underline{Z}_{th} + \underline{V}^+_{th}] \\
        \underline{V}^0_c=\dfrac{\underline{Z}_{th}}{3\underline{Z}_{th} + 6\underline{Z}_f}[\underline{I}^+\underline{Z}_{th} + \underline{I}^-\underline{Z}_{th} + \underline{V}^+_{th}]
    \end{cases}
\end{equation}






\newpage
\printbibliography


\end{document}

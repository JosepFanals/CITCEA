\documentclass[10pt]{article}

\usepackage[left=30mm, right=30mm, top=30mm, bottom=30mm]{geometry}
%\usepackage[scale=4]{draftwatermark}

\usepackage[backend=biber, sorting=none]{biblatex} % per la bibliografia
\setlength\bibitemsep{\baselineskip} % per tenir més espai entre bibliografia
\addbibresource{bib.bib} % carregar el fitxer de bibliografia
\usepackage{bm}
\usepackage{amsmath}
\usepackage[activate={true, noncompatibility}]{microtype}
\usepackage{tikz}
\usepackage{circuitikz}
% \usepackage[american voltages, american currents,siunitx]{circuitikz}
\usepackage{amssymb}
\usepackage{diagbox}
\usepackage{pgfplots}
\pgfplotsset{compat=1.17}
\usetikzlibrary{arrows.meta}
% \usepackage[american voltages, american currents,siunitx]{circuitikz}
\usepackage{amssymb}  % per simbolitzar reals
%\usepackage{subfigure} 
\usepackage{subcaption}

\begin{document}

\begin{center}
    \Large
    \textbf{Positive and negative sequence currents to improve voltages during unbalanced faults}
         
    \vspace{0.4cm}
    \textbf{Josep Fanals}
       
       
    \vspace{0.4cm}
    \textbf{02/2021}
\end{center}

\section{Introduction}
Grid faults constitute a group of unfortunate events that cause severe perturbations in the grid. The voltages can take values below the established minimum, or on the contrary, exceed the maximum in non-faulted phases. The currents are also susceptible to vary considerably. Traditional power systems based on synchronous generators could encounter currents surpassing the nominal values, and therefore, the fault could be clearly detected. However, the increasing integration of renewables \cite{anees2012grid} supposes a change of paradigm, in which currents can be controlled but are limited so as not to damage the Isolated-Gate Bipolar Transistors (IGBT) found in the Voltage Source Converter (VSC) \cite{abdou2013improving}. 

Transmission System Operators (TSO) are responsible for imposing requirements related to the operation under voltage sags to generators and converters \cite{tsili2009review, iov2007mapping}. Such requirements are gathered in the respective grid codes. There seems to be no clear consensus on how to restore the voltage. In this sense, even if for instance the Low Voltage Ride Through (LVRT) profiles present similarities \cite{conroy2007low}, analysis aimed at determining analytically the optimal injection of positive and negative sequence currents are not numerous. As far as we are aware, only Camacho et al. offer an optimal solution regarding the injection of active and reactive powers \cite{camacho2017positive}. 

Consequently, this work focuses on finding the most convenient positive and negative sequence currents (and not powers) to raise the voltage at the point of common coupling. First, a simple model is discussed, and then, the results are shown together with the corresponding discussion. Despite the elegancy found in closed-form expressions, we obtain the optimal operating point by either computational brute force or an iterative process with the help of the SciPy library. The expressions regarding the several types of fault studied as well as its corresponent diagrams are shown in the appendix. 

\section{Definition of the system}
The system we are considering is formed by an ideal grid (only positive sequence voltage is present) coupled to a VSC. The model is depicted in Figure \ref{fig:sys_p}. The VSC will be controlled in a way that leads to an improvement in the voltage at the PCC.

\begin{figure}[!htb] \centering
\begin{circuitikz}[european]
\thicklines

\draw (0,0) to [sV, v_=$\underline{V}_{th}$] (0,2);
\draw (-3,2) to [R, l=$\underline{Z}_{th}$] (0,2);
\draw (-0.25,0) to [short] (0.25,0);
\node at (-6,1.3) {PCC};
\node at (-6,2.7) {$\underline{V}_c$};
\draw (-3,2) to [short] (-3.2,1);
\draw (-3.2,1) to [short] (-2.8,1);
\draw[-{Latex[length=3mm]}] (-2.8,1) to [short] (-3,0);
\draw (-6,2) to [R, l=$\underline{Z}_a$] (-3,2);
\draw[line width=0.65mm] (-6,2.5) to [short] (-6,1.5);
\draw[line width=0.65mm] (-3,2.5) to [short] (-3,1.5);
\draw (-9,2) to [R, l=$\underline{Z}_c$, i=$\underline{I}$] (-6,2);
\draw (-10.0,2) to [sdcac] (-9.0,2);


\end{circuitikz}
\caption{Single-phase representation of the simple system under a fault}
\label{fig:sys_p}
\end{figure}
This study considers the balanced fault type but also the unbalanced ones, which include the line to ground, the double line to ground and the line to line fault. The model in Figure \ref{fig:sys_p} attempts to describe simple yet sufficient system to test the influence of the injected currents by the VSC on the voltage at the PCC. The system could be complicated by adding parallel capacitances on both sides of $\underline{Z}_a$ to model a hypothetical submarine cable. However, we prioritize keeping the system simple. 

We are concerned with improving the voltage $\underline{V}_c$ as a function of the current $\underline{I}$ injected by the converter. Formally speaking there is no clear definition on what improving the voltage means as it depends on rather pre-stablished preferences. For instance, one could try to maximize the positive sequence voltage, minimize the negative sequence voltage or even maximize the difference between both. These three strategies have been covered in \cite{camacho2017positive}. A more flexible approach is based on defining the objective function as
\begin{equation}
    f(\underline{I}^+, \underline{I}^-) = \lambda^+|(|\underline{V}^+_c(\underline{I}^+, \underline{I}^-)| - 1)| + \lambda^-|(|\underline{V}^-_c(\underline{I}^+, \underline{I}^-) - 0|)|,
    \label{eq:1}
\end{equation}
where the weighting factors $\lambda^+$, $\lambda^- \in \mathbb{R}$. By adjusting these factors one can follow the three aforementioned strategies. Note that the goal is to obtain a positive sequence voltage as close as possible to one (in per unit) while simultaneously approaching zero in the negative sequence voltage. Even though the problem is described as a function of the positive and negative sequence currents, it could also be stated as a function of the original $abc$ currents without loss of generality. The associated expressions are gathered in the appendix as well.

Minimizing $f$ does not come with complete freedom. That is, the currents are constrained so as not to exceed the IGBT limits. It becomes more convenient to express the constraints in the $abc$ frame:
\begin{equation}
    g(\underline{I}_a, \underline{I}_b, \underline{I}_c) = \begin{cases}
        |\underline{I}_a|\leq I_{max},\\
        |\underline{I}_b|\leq I_{max},\\
        |\underline{I}_c|\leq I_{max}.\\
    \end{cases}
\label{eq:2}
\end{equation}
For now we are not concerned with the voltages imposed to the semiconductors due to the fact that the filter $\underline{Z}_c$ is most likely to take small values. In practical situations current limitations are the most serious constraint. Therefore, we define the optimization problem as follows:
\begin{subequations}
\begin{alignat}{2}
&\!\min_{\underline{I}^+,\underline{I}^-}        &\qquad& f(\underline{I}^+,\underline{I}^-)\label{eq:optProb}\\
&\text{subject to} &      & g(\underline{I}_a, \underline{I}_b, \underline{I}_c) ,\label{eq:constraint1}
\end{alignat}
\end{subequations}
where the currents in the $abc$ frame can be related to the currents expressed in the symmetrical components form by means of Fortescue's transformation, and viceversa. 

Current grid codes... % review grid codes and explain their priorization

% The point of common coupling (PCC) is where the VSC together with its filter are connected. There are two restrictions to take into account in a VSC, one related to the maximum current and another to the voltage. The current limitation is likely the most relevant when operating under faults. Since the current is limited and the filter used to connect the VSC to the PCC takes rather low values, we can expect the voltage drop to not be substantial. Because of that, and taking into consideration the voltage sag at the grid side, the voltage limit is hardly ever surpassed. Thus, in the analysis that follows, we only impose the current restriction. As future work, we could also add the voltage limitations as a constraint, although probably the conclusions will not vary from the ones extracted here.

% Note that Figure \ref{fig:sys_p} is general, in the sense that it does not specify the type of fault. Besides, there will be a fault impedance, denoted by $\underline{Z}_f$. We believe every type of fault deserves to be studied separately. We are going to employ the symmetrical components, which are meant to simplify the analysis. In each fault, the voltage $\underline{V}_c$ will be decomposed in positive and negative sequence voltage and expressed as a function of the voltage at the grid together with the injected positive and negative sequence currents. It makes sense to model the VSC and its filter as a current source for this purpose. 

\section{Analysis}
% we study the four types of fault individually...

\subsection{Balanced fault}
% graph of brute force for the 4 currents: I1re, I1im, I2re, I2im. At the same plot show the point obtained with the optimization and the one we would get with the grid codes (only reactive current, but positive or negative??) Indicate that the objective function is a bit better with my optimization than following the grid code. 
% prepare plots template first

% \begin{figure}[!htb] \centering
%         \subcaptionbox{Caption 1}
%         {\begin{tikzpicture}[trim axis right,trim axis left]
%             {\pgfplotsset{width=10cm, height=4cm}
%             \begin{axis}[grid=major, xlabel={Time (ms)}, ylabel={$qd0$ voltages (pu)}, /pgf/number format/.cd, legend style={at={(0.95,0.15)},anchor=south east,legend columns=1, draw=none, inner sep=0pt,fill=gray!10}, xtick={0, 10,...,60}, ytick={0, 0.2,...,1.0}, xmin=0, xmax=60]
%             \addplot[very thick, violet,  each nth point=100] table[x=x, y=y, col sep=semicolon] {Data/pll1_q.csv};
%             \addplot[very thick, orange,  each nth point=100] table[x=x, y=y, col sep=semicolon] {Data/pll1_d.csv};
%             \addplot[very thick, cyan,  each nth point=100] table[x=x, y=y, col sep=semicolon] {Data/pll1_0.csv};
%             \legend{$q$, $d$, $0$}
%             \end{axis}
%         \end{tikzpicture}}
%         \begin{subfigure}
%         {\begin{tikzpicture}[trim axis right,trim axis left]
%             \pgfplotsset{width=10cm, height=4cm}
%             \begin{axis}[grid=major, xlabel={Time (ms)}, ylabel={Phase (rad)}, /pgf/number format/.cd, legend style={at={(0.98,0.15)},anchor=south east,legend columns=1, draw=none, inner sep=0pt,fill=gray!10}, xtick={0, 10,...,60}, ytick={0, 2,...,8}, xmin=0, xmax=60, ymin=0]
%             \addplot[very thick, black, each nth point=100] table[x=x, y=y, col sep=semicolon] {Data/pll1_ang.csv};
%             \legend{Angle}
%             \end{axis}
%         \end{tikzpicture}}
% \end{figure}



\begin{figure}
  \begin{subfigure}[t]{.4\textwidth}
    \centering
        \begin{tikzpicture}[trim axis right,trim axis left]
            \pgfplotsset{width=10cm, height=4cm}
            \begin{axis}[grid=major, xlabel={Time (ms)}, ylabel={Phase (rad)}, /pgf/number format/.cd, legend style={at={(0.98,0.15)},anchor=south east,legend columns=1, draw=none, inner sep=0pt,fill=gray!10}, xtick={0, 10,...,60}, ytick={0, 2,...,8}, xmin=0, xmax=60, ymin=0]
            \addplot[very thick, black, each nth point=100] table[x=x, y=y, col sep=semicolon] {Data/pll1_ang.csv};
            \legend{Angle}
            \end{axis}
        \end{tikzpicture}
    \caption{Caption1}
  \end{subfigure}
  \hfill
  \begin{subfigure}[t]{.4\textwidth}
    \centering
    \includegraphics[width=\linewidth]{example-image-b}
    \caption{\textbf{Vereinigung}: $A \cap B$: Element liegt in $A$ \textbf{und} in $B$.}
  \end{subfigure}

  \medskip

  \begin{subfigure}[t]{.4\textwidth}
    \centering
    \includegraphics[width=\linewidth]{example-image-c}
    \caption{\textbf{Differenz}: $A \setminus B$: Element liegt in $A$ \textbf{nicht} in $B$. (\textit{A ohne B})}
  \end{subfigure}
  \hfill
  \begin{subfigure}[t]{.4\textwidth}
    \centering
    \includegraphics[width=\linewidth]{example-image-a}
    \caption{\textbf{Symmetrische Differenz}: $A \Delta B$: Element liegt \textbf{entweder} in $A$ oder in $B$.}
  \end{subfigure}
\end{figure}

\subsection{Line to ground fault}
\subsection{Line to line fault}
\subsection{Double line to ground fault}

\newpage
\appendix
\section{$abc$ circuits}

\begin{figure}[!htb] \centering
\begin{circuitikz}[european]
\thicklines

\draw (0,0) to [american controlled current source,  *-] (2,0);
\draw (0,-2) to [american controlled current source, *-] (2,-2);
\draw (0,-4) to [american controlled current source, *-] (2,-4);
\draw (2,0) to [R, l=$\underline{Z}_a$, -] (4,0);
\draw (2,-2) to [R, l=$\underline{Z}_a$, -] (4,-2);
\draw (2,-4) to [R, l=$\underline{Z}_a$, -] (4,-4);
\draw (4,0) to [short] (6,0);
\draw (4,-2) to [short] (6,-2);
\draw (4,-4) to [short] (6,-4);
\draw (0,0) to [short] (0,-4);

\draw (6,0) to [R, l=$\underline{Z}_{th}$, -] (8,0);
\draw (6,-2) to [R, l=$\underline{Z}_{th}$, -] (8,-2);
\draw (6,-4) to [R, l=$\underline{Z}_{th}$, -] (8,-4);
\draw (10,0) to [sV, v_=$\underline{V}^{a}_{th}$, *-] (8,0);
\draw (10,-2) to [sV, v_=$\underline{V}^{b}_{th}$, *-] (8,-2);
\draw (10,-4) to [sV, v_=$\underline{V}^{c}_{th}$, *-] (8,-4);
\draw (10, 0) to [short] (10, -4);
\draw (10,-2) to [short] (11.5,-2);
\draw (11.5,-2) to [short] (11.5, -4.5);
\draw (11.5,-4.5) -- (11.5,-5.00) node[ground]{};


\draw (1.5,-0.4) node[]{$\underline{I}^a$};
\draw (2,0.3) to [open,v=$\underline{V}^a_c$] (0,0.3);
\draw (1.5,-2.4) node[]{$\underline{I}^b$};
\draw (2,-1.7) to [open,v=$\underline{V}^b_c$] (0,-1.7);
\draw (1.5,-4.4) node[]{$\underline{I}^c$};
\draw (2,-3.7) to [open,v=$\underline{V}^c_c$] (0,-3.7);

\draw (4.0,0) to [short, *-] (4.0,-4);
\draw (5.0,-2) to [short, *-] (5.0,-4);

\draw (4.0,-4) to [R, l=$\underline{Z}_f$, -*] (4.0,-6);
\draw (5.0,-4) to [R, l=$\underline{Z}_f$, -*] (5.0,-6);
\draw (6.0,-4) to [R, l=$\underline{Z}_f$, *-*] (6.0,-6);
\draw (4,-6) to [short] (6,-6);
 
\end{circuitikz}
\caption{Balanced fault schematic}
\label{fig:3_trif}
\end{figure}


\begin{figure}[!htb] \centering
\begin{circuitikz}[european]
\thicklines

\draw (0,0) to [american controlled current source,  *-] (2,0);
\draw (0,-2) to [american controlled current source, *-] (2,-2);
\draw (0,-4) to [american controlled current source, *-] (2,-4);
\draw (2,0) to [R, l=$\underline{Z}_a$, -] (4,0);
\draw (2,-2) to [R, l=$\underline{Z}_a$, -] (4,-2);
\draw (2,-4) to [R, l=$\underline{Z}_a$, -] (4,-4);
\draw (4,0) to [short] (6,0);
\draw (4,-2) to [short] (6,-2);
\draw (4,-4) to [short] (6,-4);
\draw (0,0) to [short] (0,-4);

\draw (6,0) to [R, l=$\underline{Z}_{th}$, -] (8,0);
\draw (6,-2) to [R, l=$\underline{Z}_{th}$, -] (8,-2);
\draw (6,-4) to [R, l=$\underline{Z}_{th}$, -] (8,-4);
\draw (10,0) to [sV, v_=$\underline{V}^{a}_{th}$, *-] (8,0);
\draw (10,-2) to [sV, v_=$\underline{V}^{b}_{th}$, *-] (8,-2);
\draw (10,-4) to [sV, v_=$\underline{V}^{c}_{th}$, *-] (8,-4);
\draw (10, 0) to [short] (10, -4);
\draw (10,-2) to [short] (11.5,-2);
\draw (11.5,-2) to [short] (11.5, -4.5);
\draw (11.5,-4.5) -- (11.5,-5.00) node[ground]{};


\draw (1.5,-0.4) node[]{$\underline{I}^a$};
\draw (2,0.3) to [open,v=$\underline{V}^a_c$] (0,0.3);
\draw (1.5,-2.4) node[]{$\underline{I}^b$};
\draw (2,-1.7) to [open,v=$\underline{V}^b_c$] (0,-1.7);
\draw (1.5,-4.4) node[]{$\underline{I}^c$};
\draw (2,-3.7) to [open,v=$\underline{V}^c_c$] (0,-3.7);

\draw (5.0, 0) to [short, *-] (5.0, -4);
\draw (5.0,-4) to [R, l=$\underline{Z}_f$] (5.0,-5.5);
\draw (5.0,-5.5) -- (5.0,-5.5) node[ground]{};
 
\end{circuitikz}
\caption{Line to ground fault schematic}
\label{fig:3_lg}
\end{figure}


\begin{figure}[!htb] \centering
\begin{circuitikz}[european]
\thicklines

\draw (0,0) to [american controlled current source,  *-] (2,0);
\draw (0,-2) to [american controlled current source, *-] (2,-2);
\draw (0,-4) to [american controlled current source, *-] (2,-4);
\draw (2,0) to [R, l=$\underline{Z}_a$, -] (4,0);
\draw (2,-2) to [R, l=$\underline{Z}_a$, -] (4,-2);
\draw (2,-4) to [R, l=$\underline{Z}_a$, -] (4,-4);
\draw (4,0) to [short] (6,0);
\draw (4,-2) to [short] (6,-2);
\draw (4,-4) to [short] (6,-4);
\draw (0,0) to [short] (0,-4);

\draw (6,0) to [R, l=$\underline{Z}_{th}$, -] (8,0);
\draw (6,-2) to [R, l=$\underline{Z}_{th}$, -] (8,-2);
\draw (6,-4) to [R, l=$\underline{Z}_{th}$, -] (8,-4);
\draw (10,0) to [sV, v_=$\underline{V}^{a}_{th}$, *-] (8,0);
\draw (10,-2) to [sV, v_=$\underline{V}^{b}_{th}$, *-] (8,-2);
\draw (10,-4) to [sV, v_=$\underline{V}^{c}_{th}$, *-] (8,-4);
\draw (10, 0) to [short] (10, -4);
\draw (10,-2) to [short] (11.5,-2);
\draw (11.5,-2) to [short] (11.5, -4.5);
\draw (11.5,-4.5) -- (11.5,-5.00) node[ground]{};


\draw (1.5,-0.4) node[]{$\underline{I}^a$};
\draw (2,0.3) to [open,v=$\underline{V}^a_c$] (0,0.3);
\draw (1.5,-2.4) node[]{$\underline{I}^b$};
\draw (2,-1.7) to [open,v=$\underline{V}^b_c$] (0,-1.7);
\draw (1.5,-4.4) node[]{$\underline{I}^c$};
\draw (2,-3.7) to [open,v=$\underline{V}^c_c$] (0,-3.7);

\draw (4.0,-2) to [short, *-] (4.0, -5.0);
\draw (6.0,-4) to [short, *-] (6.0, -5.0);
\draw (4,-5.0) to [R, l=$\underline{Z}_f$] (6, -5.0);

 
\end{circuitikz}
\caption{Line to line fault schematic}
\label{fig:3_ll}
\end{figure}


\begin{figure}[!htb] \centering
\begin{circuitikz}[european]
\thicklines

\draw (0,0) to [american controlled current source,  *-] (2,0);
\draw (0,-2) to [american controlled current source, *-] (2,-2);
\draw (0,-4) to [american controlled current source, *-] (2,-4);
\draw (2,0) to [R, l=$\underline{Z}_a$, -] (4,0);
\draw (2,-2) to [R, l=$\underline{Z}_a$, -] (4,-2);
\draw (2,-4) to [R, l=$\underline{Z}_a$, -] (4,-4);
\draw (4,0) to [short] (6,0);
\draw (4,-2) to [short] (6,-2);
\draw (4,-4) to [short] (6,-4);
\draw (0,0) to [short] (0,-4);

\draw (6,0) to [R, l=$\underline{Z}_{th}$, -] (8,0);
\draw (6,-2) to [R, l=$\underline{Z}_{th}$, -] (8,-2);
\draw (6,-4) to [R, l=$\underline{Z}_{th}$, -] (8,-4);
\draw (10,0) to [sV, v_=$\underline{V}^{a}_{th}$, *-] (8,0);
\draw (10,-2) to [sV, v_=$\underline{V}^{b}_{th}$, *-] (8,-2);
\draw (10,-4) to [sV, v_=$\underline{V}^{c}_{th}$, *-] (8,-4);
\draw (10, 0) to [short] (10, -4);
\draw (10,-2) to [short] (11.5,-2);
\draw (11.5,-2) to [short] (11.5, -4.5);
\draw (11.5,-4.5) -- (11.5,-5.00) node[ground]{};


\draw (1.5,-0.4) node[]{$\underline{I}^a$};
\draw (2,0.3) to [open,v=$\underline{V}^a_c$] (0,0.3);
\draw (1.5,-2.4) node[]{$\underline{I}^b$};
\draw (2,-1.7) to [open,v=$\underline{V}^b_c$] (0,-1.7);
\draw (1.5,-4.4) node[]{$\underline{I}^c$};
\draw (2,-3.7) to [open,v=$\underline{V}^c_c$] (0,-3.7);

\draw (4.0,-2) to [short, *-] (4.0, -5.0);
\draw (6.0,-4) to [short, *-] (6.0, -5.0);
\draw (4,-5.0) to [short, -*] (5, -5.0);
\draw (5,-5.0) to [short, ] (6, -5.0);
\draw (5,-5) to [R, l=$\underline{Z}_f$] (5,-6.5);
\draw (5,-6.5) -- (5,-6.5) node[ground]{};

 
\end{circuitikz}
\caption{Double line to ground fault schematic}
\label{fig:3_llg}
\end{figure}

\clearpage
\newpage
\section{$+-0$ schemes}

\begin{figure}[!htb] \centering
\begin{circuitikz}[european]
\thicklines

\draw (0,6) to [sV, v_=$\underline{V}_{th}^+$] (0,8);
\draw (-3,8) to [R, l=$\underline{Z}_{th}$] (0,8);
\draw (-6,6) to [short] (0.0,6);
\draw (-6,8) to [R, l=$\underline{Z}_a$] (-3,8);
\draw (-6,6.5) to [american controlled current source, l_=$I^+$] (-6,7.5);
\draw (-6,6) to [short] (-6,6.5);
\draw (-6,7.5) to [short] (-6,8);
\draw (-6.3,8) to [open,v=$\underline{V}^+_c$] (-6.3,6);

\draw (0,3) to [short] (0,3.5);
\draw (0,3.5) to [short, *-*] (0,4.5);
\draw (0,4.5) to [short] (0,5);
\draw (-3,5) to [R, l=$\underline{Z}_{th}$] (0,5);
\draw (-6,3) to [short] (0.0,3);
\draw (-6,5) to [R, l=$\underline{Z}_a$] (-3,5);
\draw (-6,3.5) to [american controlled current source, l_=$I^-$] (-6,4.5);
\draw (-6,3) to [short] (-6,3.5);
\draw (-6,4.5) to [short] (-6,5);
\draw (-6.3,5) to [open,v=$\underline{V}^-_c$] (-6.3,3);

\draw (0,0) to [short] (0,0.5);
\draw (0,0.5) to [short, *-*] (0,1.5);
\draw (0,1.5) to [short] (0,2);
\draw (-3,2) to [R, l=$\underline{Z}_{th}$] (0,2);
\draw (-6,0) to [short] (0.0,0);
\draw (-6,2) to [R, l=$\underline{Z}_a$] (-3,2);
\draw (-6,0) to [short, -*] (-6,0.5);
\draw (-6,1.5) to [short, *-] (-6,2);
\draw (-6.3,2) to [open,v=$\underline{V}^0_c$] (-6.3,0);

\draw (-3,8) to [R, l=$\underline{Z}_{f}$] (-3,6);
\draw (-3,5) to [R, l=$\underline{Z}_{f}$] (-3,3);
\draw (-3,2) to [R, l=$\underline{Z}_{f}$] (-3,0);


\end{circuitikz}
\caption{Equivalent circuit for the balanced fault analysis}
\label{fig:sys_3x}
\end{figure}


\begin{figure}[!htb] \centering
\begin{circuitikz}[european]
\thicklines

\draw (0,6) to [sV, v_=$\underline{V}_{th}^+$] (0,8);
\draw (-3,8) to [R, l=$\underline{Z}_{th}$] (0,8);
\draw (-6,6) to [short] (0.0,6);
\draw (-6,8) to [R, l=$\underline{Z}_a$] (-3,8);
\draw (-6,6.5) to [american controlled current source, l_=$I^+$] (-6,7.5);
\draw (-6,6) to [short] (-6,6.5);
\draw (-6,7.5) to [short] (-6,8);
\draw (-6.3,8) to [open,v=$\underline{V}^+_c$] (-6.3,6);

\draw (0,3) to [short] (0,3.5);
\draw (0,3.5) to [short, *-*] (0,4.5);
\draw (0,4.5) to [short] (0,5);
\draw (-3,5) to [R, l=$\underline{Z}_{th}$] (0,5);
\draw (-6,3) to [short] (0.0,3);
\draw (-6,5) to [R, l=$\underline{Z}_a$] (-3,5);
\draw (-6,3.5) to [american controlled current source, l_=$I^-$] (-6,4.5);
\draw (-6,3) to [short] (-6,3.5);
\draw (-6,4.5) to [short] (-6,5);
\draw (-6.3,5) to [open,v=$\underline{V}^-_c$] (-6.3,3);

\draw (0,0) to [short] (0,0.5);
\draw (0,0.5) to [short, *-*] (0,1.5);
\draw (0,1.5) to [short] (0,2);
\draw (-3,2) to [R, l=$\underline{Z}_{th}$] (0,2);
\draw (-6,0) to [short] (0.0,0);
\draw (-6,2) to [R, l=$\underline{Z}_a$] (-3,2);
\draw (-6,0) to [short, -*] (-6,0.5);
\draw (-6,1.5) to [short, *-] (-6,2);
\draw (-6.3,2) to [open,v=$\underline{V}^0_c$] (-6.3,0);

\draw (-3,2) to [short, *-*] (-3,3);
\draw (-3,5) to [short, *-*] (-3,6);
\draw (-3,8) to [short, *-] (-3, 9);
\draw (-3,0) to [short, *-] (-3, -0.5);

\draw (-3,9) to [short] (2,9);
\draw (-3,-0.5) to [short] (2,-0.5);
\draw (2,9) to [R, l=$3\underline{Z}_f$] (2, -0.5);

\end{circuitikz}
\caption{Equivalent circuit for the line to ground fault analysis}
\label{fig:sys_LG}
\end{figure}

\begin{figure}[!htb] \centering
\begin{circuitikz}[european]
\thicklines

\draw (0,6) to [sV, v_=$\underline{V}_{th}^+$] (0,8);
\draw (-2.5,8) to [R, l=$\underline{Z}_{th}$] (0,8);
\draw (-5,6) to [short] (0.0,6);
\draw (-5,8) to [R, l=$\underline{Z}_a$] (-2.5,8);
\draw (-5,6.5) to [american controlled current source, l_=$I^+$] (-5,7.5);
\draw (-5,6) to [short] (-5,6.5);
\draw (-5,7.5) to [short] (-5,8);
\draw (-5.3,8) to [open,v=$\underline{V}^+_c$] (-5.3,6);

\draw (8,6) to [short] (8,6.5);
\draw (8,6.5) to [short, *-*] (8,7.5);
\draw (8,7.5) to [short] (8,8);
\draw (5.5,8) to [R, l=$\underline{Z}_{th}$] (8,8);
\draw (3,6) to [short] (8.0,6);
\draw (3,8) to [R, l=$\underline{Z}_a$] (5.5,8);
\draw (3,6.5) to [american controlled current source, l_=$I^-$] (3,7.5);
\draw (3,6) to [short] (3,6.5);
\draw (3,7.5) to [short] (3,8);
\draw (2.7,8) to [open,v=$\underline{V}^-_c$] (2.7,6);

\draw (-2.5, 8) to [short, *-] (-2.5, 9);
\draw (-2.5, 9) to [R, l=$\underline{Z}_{f}$] (5.5,9);
\draw (5.5,9) to [short, -*] (5.5, 8);

\draw (-2.5, 6) to [short, *-] (-2.5, 5.5);
\draw (-2.5, 5.5) to [short] (5.5, 5.5);
\draw (5.5, 5.5) to [short, -*] (5.5, 6);

\end{circuitikz}
\caption{Equivalent circuit for the line to line fault analysis}
\label{fig:sys_LL}
\end{figure}


\begin{figure}[!htb] \centering
\begin{circuitikz}[european]
\thicklines

\draw (0,8) to [sV, v_=$\underline{V}_{th}^+$] (0,10);
\draw (-3,10) to [R, l=$\underline{Z}_{th}$] (0,10);
\draw (-6,8) to [short] (0.0,8);
\draw (-6,10) to [R, l=$\underline{Z}_a$] (-3,10);
\draw (-6,8.5) to [american controlled current source, l_=$I^+$] (-6,9.5);
\draw (-6,8) to [short] (-6,8.5);
\draw (-6,9.5) to [short] (-6,10);
\draw (-6.3,10) to [open,v=$\underline{V}^+_c$] (-6.3,8);

\draw (0,4) to [short] (0,4.5);
\draw (0,4.5) to [short, *-*] (0,5.5);
\draw (0,5.5) to [short] (0,6);
\draw (-3,6) to [R, l=$\underline{Z}_{th}$] (0,6);
\draw (-6,4) to [short] (0.0,4);
\draw (-6,6) to [R, l=$\underline{Z}_a$] (-3,6);
\draw (-6,4.5) to [american controlled current source, l_=$I^-$] (-6,5.5);
\draw (-6,4) to [short] (-6,4.5);
\draw (-6,5.5) to [short] (-6,6);
\draw (-6.3,6) to [open,v=$\underline{V}^-_c$] (-6.3,4);

\draw (0,0) to [short] (0,0.5);
\draw (0,0.5) to [short, *-*] (0,1.5);
\draw (0,1.5) to [short] (0,2);
\draw (-3,2) to [R, l=$\underline{Z}_{th}$] (0,2);
\draw (-6,0) to [short] (0.0,0);
\draw (-6,2) to [R, l=$\underline{Z}_a$] (-3,2);
\draw (-6,0) to [short, -*] (-6,0.5);
\draw (-6,1.5) to [short, *-] (-6,2);
\draw (-6.3,2) to [open,v=$\underline{V}^0_c$] (-6.3,0);

\draw (-3,0) to [short, *-] (-3,-0.5);
\draw (-3,4) to [short, *-] (-3,3.5);
\draw (-3,8) to [short, *-] (-3,7.5);

\draw (-3,2) to [short, *-] (-3,3);
\draw (-3,6) to [short, *-] (-3,7);
\draw (-3,10) to [short, *-] (-3,11);

\draw (-3,3) to [short] (2,3);
\draw (-3,7) to [short] (2,7);
\draw (-3,11) to [short] (2,11);

\draw (-3,-0.5) to [short] (-8,-0.5);
\draw (-3,3.5) to [short] (-8,3.5);
\draw (-3,7.5) to [short] (-8,7.5);

\draw (-8,-0.5) to [short, -*] (-8, 3.5);
\draw (-8, 3.5) to [short] (-8, 7.5);

\draw (2,3) to [R, l_=$3\underline{Z}_f$, -*] (2,7);
\draw (2,11) to [short] (2,7);


\end{circuitikz}
\caption{Equivalent circuit for the double line to ground fault analysis}
\label{fig:sys_LLG}
\end{figure}



\clearpage
\newpage
\section{Expressions}

\subsection{Balanced fault}
\begin{equation}
    \begin{cases}
        \underline{V}^a_c = \dfrac{1}{\underline{Z}_f + \underline{Z}_{th}}[\underline{V}^a_{th}\underline{Z}_f + \underline{I}_a(\underline{Z}_a\underline{Z}_{th} + \underline{Z}_{th}\underline{Z}_f + \underline{Z}_f\underline{Z}_a)]\\
        \underline{V}^b_c = \dfrac{1}{\underline{Z}_f + \underline{Z}_{th}}[\underline{V}^b_{th}\underline{Z}_f + \underline{I}_b(\underline{Z}_a\underline{Z}_{th} + \underline{Z}_{th}\underline{Z}_f + \underline{Z}_f\underline{Z}_a)]\\
        \underline{V}^c_c = \dfrac{1}{\underline{Z}_f + \underline{Z}_{th}}[\underline{V}^c_{th}\underline{Z}_f + \underline{I}_c(\underline{Z}_a\underline{Z}_{th} + \underline{Z}_{th}\underline{Z}_f + \underline{Z}_f\underline{Z}_a)]
    \end{cases}
\end{equation}


\begin{equation}
    \begin{cases}
        \underline{V}^+_c= \dfrac{1}{\underline{Z}_f + \underline{Z}_{th}}[\underline{V}^+_{th}\underline{Z}_f + \underline{I}^+(\underline{Z}_a\underline{Z}_f + \underline{Z}_a\underline{Z}_{th} + \underline{Z}_f\underline{Z}_{th})] \\ 
 \underline{V}^-_c=\dfrac{1}{\underline{Z}_{f} + \underline{Z}_{th}}[\underline{I}^-(\underline{Z}_{th}\underline{Z}_f + \underline{Z}_a\underline{Z}_{th} + \underline{Z}_a\underline{Z}_f)] \\
 \underline{V}^0_c=0
    \end{cases}
\end{equation}


\subsection{Line to ground fault}
\begin{equation}
    \begin{cases}
        \underline{V}^a_c = \dfrac{1}{\underline{Z}_{th} + \underline{Z}_f}[\underline{I}_a(\underline{Z}_a\underline{Z}_{th} + \underline{Z}_a\underline{Z}_f + \underline{Z}_{th}\underline{Z}_f) + \underline{V}^a_{th}\underline{Z}_f] \\
        \underline{V}^b_c = \underline{V}^b_{th} + \underline{I}_b(\underline{Z}_a + \underline{Z}_{th}) \\
        \underline{V}^c_c =  \underline{V}^c_{th} + \underline{I}_c(\underline{Z}_a + \underline{Z}_{th})
    \end{cases}
\end{equation}

\begin{equation}
    \begin{cases}
        \underline{V}^+_c=\underline{I}^+(\underline{Z}_a + \underline{Z}_{th}) + \underline{V}^+_{th} - \dfrac{\underline{Z}_{th}}{3\underline{Z}_f + 3\underline{Z}_{th}}[\underline{I}^+\underline{Z}_{th} + \underline{I}^-\underline{Z}_{th} + \underline{V}^+_{th}]\\ 
        \underline{V}^-_c=\underline{I}^-(\underline{Z}_a + \underline{Z}_{th}) - \dfrac{\underline{Z}_{th}}{3\underline{Z}_f + 3\underline{Z}_{th}}[\underline{I}^+\underline{Z}_{th} + \underline{I}^-\underline{Z}_{th} + \underline{V}^+_{th}]\\
        \underline{V}^0_c=-\dfrac{\underline{Z}_{th}}{3\underline{Z}_f + 3\underline{Z}_{th}}[\underline{V}^+_{th} + \underline{I}^+\underline{Z}_{th} + \underline{I}^-\underline{Z}_{th}]
   \end{cases}
\end{equation}


\subsection{Line to line fault}
\begin{equation}
    \begin{cases}
        \underline{V}^a_c = \underline{I}_a (\underline{Z}_a + \underline{Z}_{th}) + \underline{V}^a_{th}  \\
        \underline{V}^b_c = \underline{I}_b\underline{Z}_a + \dfrac{1}{(\underline{Z}_{f} + \underline{Z}_{th}) (\underline{Z}_f + 2\underline{Z}_{th})}[\underline{I}_b (\underline{Z}_{th}\underline{Z}_f\underline{Z}_f + 2\underline{Z}_{th}\underline{Z}_{th}\underline{Z}_{f} + \underline{Z}_{th}\underline{Z}_{th}\underline{Z}_{th}) \\+ \underline{I}_c(\underline{Z}_{th}\underline{Z}_{th}\underline{Z}_f + \underline{Z}_{th}\underline{Z}_{th}\underline{Z}_{th}) + \underline{V}^b_{th} (\underline{Z}_f\underline{Z}_f + 2 \underline{Z}_f\underline{Z}_{th} + \underline{Z}_{th}\underline{Z}_{th}) + \underline{V}^c_{th}(\underline{Z}_{th}\underline{Z}_f + \underline{Z}_{th} \underline{Z}_{th})  ] \\
        \underline{V}^c_c = \underline{I}_c\underline{Z}_a + \dfrac{1}{\underline{Z}_f + 2\underline{Z}_{th}}[\underline{I}_c(\underline{Z}_{th}(\underline{Z}_f + \underline{Z}_{th})) + \underline{V}^c_{th} (\underline{Z}_f + \underline{Z}_{th}) + \underline{I}_b\underline{Z}_{th}\underline{Z}_{th} + \underline{V}^b_{th}\underline{Z}_{th}]
    \end{cases}
\end{equation}

\begin{equation}
    \begin{cases}
        \underline{V}^+_c= \underline{V}^+_{th} + \underline{I}^+(\underline{Z}_{a} + \underline{Z}_{th}) - \dfrac{\underline{Z}_{th}}{2\underline{Z}_{th}+\underline{Z}_f}[\underline{V}^+_{th} + \underline{I}^+\underline{Z}_{th}-\underline{I}^-\underline{Z}_{th}]  \\ 
        \underline{V}^-_c= \underline{V}^+_{th} + \underline{I}^+\underline{Z}_{th}+ \underline{I}^-\underline{Z}_a -\dfrac{\underline{Z}_{th}+\underline{Z}_f}{2\underline{Z}_{th}+\underline{Z}_f}[\underline{V}^+_{th} + \underline{I}^+\underline{Z}_{th} - \underline{I}^-\underline{Z}_{th}] \\
        \underline{V}^0_c=0
    \end{cases}
\end{equation} 


\subsection{Double line to ground fault}
\begin{equation}
    \begin{cases}
        \underline{V}^a_c = \underline{I}_a (\underline{Z}_a + \underline{Z}_{th}) + \underline{V}^a_{th}  \\
        \underline{V}^b_c = \underline{I}_b\underline{Z}_a + \dfrac{\underline{Z}_{th}\underline{Z}_f (\underline{I}_b+\underline{I}_c) + \underline{Z}_f(\underline{V}^b_{th} + \underline{V}^c_{th})}{2\underline{Z}_f + \underline{Z}_{th}} \\
        \underline{V}^c_c = \underline{I}_c\underline{Z}_a + \dfrac{\underline{Z}_{th}\underline{Z}_f (\underline{I}_b+\underline{I}_c) + \underline{Z}_f(\underline{V}^b_{th} + \underline{V}^c_{th})}{2\underline{Z}_f + \underline{Z}_{th}} 
    \end{cases}
\end{equation}

\begin{equation}
    \begin{cases}
        \underline{V}^+_c=\underline{I}^+\underline{Z}_a + \dfrac{\underline{Z}_{th} + 3\underline{Z}_f}{3\underline{Z}_{th} + 6\underline{Z}_f}[\underline{I}^+\underline{Z}_{th} + \underline{I}^-\underline{Z}_{th} + \underline{V}^+_{th}]    \\ 
        \underline{V}^-_c= \underline{I}^-\underline{Z}_a + \dfrac{\underline{Z}_{th} + 3\underline{Z}_f}{3\underline{Z}_{th} + 6\underline{Z}_f}[\underline{I}^+\underline{Z}_{th} + \underline{I}^-\underline{Z}_{th} + \underline{V}^+_{th}] \\
        \underline{V}^0_c=\dfrac{\underline{Z}_{th}}{3\underline{Z}_{th} + 6\underline{Z}_f}[\underline{I}^+\underline{Z}_{th} + \underline{I}^-\underline{Z}_{th} + \underline{V}^+_{th}]
    \end{cases}
\end{equation}






\newpage
\printbibliography


\end{document}

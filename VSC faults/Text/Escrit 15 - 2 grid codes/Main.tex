%!TEX program = lualatex
\documentclass[10pt]{article}

\usepackage[left=30mm, right=30mm, top=30mm, bottom=30mm]{geometry}
%\usepackage[scale=4]{draftwatermark}

% \usepackage[backend=bibtex, sorting=none]{biblatex} % per la bibliografia
\usepackage[backend=biber, sorting=none]{biblatex}
% \bibliographystyle{ieeetr}
% \setlength\bibitemsep{\baselineskip} % per tenir més espai entre bibliografia
\addbibresource{bib.bib} % carregar el fitxer de bibliografia
\usepackage{bm}
\usepackage{amsmath}
\usepackage[activate={true, noncompatibility}]{microtype}
\usepackage{tikz}
\usepackage{circuitikz}
% \usepackage[american voltages, american currents,siunitx]{circuitikz}
\usepackage{amssymb}
\usepackage{diagbox}
\usepackage{pgfplots}
\pgfplotsset{compat=1.17}
\usetikzlibrary{arrows.meta}
% \usepackage[american voltages, american currents,siunitx]{circuitikz}
\usepackage{amssymb}  % per simbolitzar reals
%\usepackage{subfigure} 
\usepackage{subcaption}
\usepgfplotslibrary{patchplots,colormaps}
\usetikzlibrary{intersections}
\usepgfplotslibrary{fillbetween}
\usepackage{import} % per a les imatges d'Inkscape
\usepackage{xifthen} % per a les imatges d'Inkscape
\usepackage{pdfpages} % per a les imatges d'Inkscape
\usepackage{transparent} % per a les imatges d'Inkscape
\newcommand{\incfig}[1]{% per incloure figura d'Inkscape
    \def\svgwidth{\columnwidth}
    \import{./Data/Drawings}{#1.pdf_tex}
}
\usepackage{svg} % pel logo en format .svg



\begin{document}



\section{Positive current prioritization}
Given a generic disposition of positive and negative voltages as shown in Fig. \ref{fig:p1}, we are concerned with determining the maximum negative sequence current capacity.

\begin{figure}[!htb]\centering
    \incfig{v1}
    \caption{General representation of positive and negative sequence currents and voltages}
    \label{fig:p1}
\end{figure}
Currents are delayed or advanced $\frac{\pi}{2}$ in order to only inject reactive power. As limitations are imposed in the phases (natural reference frame), currents are transformed as follows
\begin{equation}
    \begin{pmatrix}
        \underline{I}_a \\
        \underline{I}_b \\
        \underline{I}_c \\
    \end{pmatrix}
    = \begin{pmatrix}
        1 & 1 & 1 \\
        1 & \underline{a}^2 & \underline{a} \\
        1 & \underline{a} & \underline{a}^2 \\
    \end{pmatrix}
    \begin{pmatrix}
        0 \\
        \underline{I}^+ \\
        \underline{I}^-
    \end{pmatrix}  
\end{equation}
where $\underline{a}=e^{j\frac{2\pi}{3}}$ and the zero sequence current is forced to be null. Then, phase currents are expressed as
\begin{equation}
    \begin{cases}
    \underline{I}_a = \underline{I}^+ + \underline{I}^- \\
    \underline{I}_b = \underline{a}^2 \underline{I}^+ + \underline{a}\underline{I}^- \\
    \underline{I}_c = \underline{a}\underline{I}^+ + \underline{a}^2\underline{I}^- \\
    \end{cases}
\end{equation}
It is now convenient to write the previous expressions without the operator $a$ explicitly shown:
\begin{equation}
    \begin{cases}
    \underline{I}_a = \underline{I}^+_a + \underline{I}^-_a \\
    \underline{I}_b = \underline{I}^+_b + \underline{I}^-_b \\
    \underline{I}_c = \underline{I}^+_c + \underline{I}^-_c \\
    \end{cases}
\end{equation}
where $\underline{I}^+_c = \underline{a}\underline{I}^+$ and $\underline{I}^-_c = \underline{a}^2\underline{I}^-$, and analogously for other phases. All three cases have to be computed, yet the smallest negative sequence current (i.e., \texttt{min}$({I}^-_a, {I}^-_b, {I}^-_c)$) acts as the most restrictive option. 

Next, all phase currents are set to its maximum value, denoted by $I_{\text{max}}$. Thus, squaring them:
\begin{equation}
    \begin{cases}
        I_{\text{max}}^2 = (I^+_{a,re} + I^-_{a,re})^2 + (I^+_{a,im} + t_a I^-_{a,re})^2 \\
        I_{\text{max}}^2 = (I^+_{b,re} + I^-_{b,re})^2 + (I^+_{b,im} + t_b I^-_{b,re})^2 \\
        I_{\text{max}}^2 = (I^+_{c,re} + I^-_{c,re})^2 + (I^+_{c,im} + t_c I^-_{c,re})^2 \\
    \end{cases}
    \label{eq:ii1}
\end{equation}
which has to be solved for $I^-_{a,re}, I^-_{b,re}$ and $I^-_{c,re}$, where $t_a$, $t_b$ and $t_c$ represent the proportion between the imaginary and the real part of the particular negative sequence components. In other words, if phasor $\underline{I}^-$ is at an angle $\alpha$, as indicated in Fig. \ref{fig:p1}, then:
\begin{equation}
    \begin{cases}
        t_a \equiv \tan(\alpha) = \frac{I^-_{a,im}}{I^-_{a,re}} \\
        t_b \equiv \tan(\alpha + \frac{2\pi}{3}) = \frac{I^-_{b,im}}{I^-_{b,re}} \\
        t_c \equiv \tan(\alpha - \frac{2\pi}{3}) = \frac{I^-_{c,im}}{I^-_{c,re}} \\
    \end{cases}
\end{equation}
The final step is to solve each one of the three quadratic equations that are derived from \eqref{eq:ii1}, select the appropriate solution according to the already known direction, and compute the absolute value of these negative sequence currents 
\begin{equation}
    \begin{cases}
        I^-_{a} = \sqrt{(I^-_{a,re})^2 + (t_a I^-_{a,re})^2} \\
        I^-_{b} = \sqrt{(I^-_{b,re})^2 + (t_b I^-_{b,re})^2} \\
        I^-_{c} = \sqrt{(I^-_{c,re})^2 + (t_c I^-_{c,re})^2} \\
    \end{cases}
\end{equation}


\section{Negative current prioritization}
\newpage
% \section{Optimization overview and constant impedance analysis}
The system we are considering is formed by an ideal grid (only positive sequence voltage is present) coupled to a VSC. The model is depicted in Figure \ref{fig:sys_p}. The VSC will be controlled in a way that leads to an improvement in the voltage at the PCC.

\begin{figure}[!htb] \centering
\begin{circuitikz}[european]
\thicklines

\draw (0,0) to [sV, v_=$\underline{V}_{th}$] (0,2);
\draw (-3,2) to [R, l=$\underline{Z}_{th}$] (0,2);
\draw (-0.25,0) to [short] (0.25,0);
\node at (-6,1.3) {PCC};
\node at (-6,2.7) {$\underline{V}_c$};
\draw (-3,2) to [short] (-3.2,1);
\draw (-3.2,1) to [short] (-2.8,1);
\draw[-{Latex[length=3mm]}] (-2.8,1) to [short] (-3,0);
\draw (-6,2) to [R, l=$\underline{Z}_a$] (-3,2);
\draw[line width=0.65mm] (-6,2.5) to [short] (-6,1.5);
\draw[line width=0.65mm] (-3,2.5) to [short] (-3,1.5);
\draw (-9,2) to [R, l=$\underline{Z}_c$, i=$\underline{I}$] (-6,2);
\draw (-10.0,2) to [sdcac] (-9.0,2);


\end{circuitikz}
\caption{Single-phase representation of the simple system under a fault}
\label{fig:sys_p}
\end{figure}
This study considers the balanced fault type but also the unbalanced ones, which include the line to ground, the double line to ground and the line to line fault. The model in Figure \ref{fig:sys_p} attempts to describe simple yet sufficient system to test the influence of the injected currents by the VSC on the voltage at the PCC. The system could be complicated by adding parallel capacitances on both sides of $\underline{Z}_a$ to model a hypothetical submarine cable. However, we prioritize keeping the system simple. 

We are concerned with improving the voltage $\underline{V}_c$ as a function of the current $\underline{I}$ injected by the converter. Formally speaking there is no clear definition on what improving the voltage means as it depends on rather pre-stablished preferences. For instance, one could try to maximize the positive sequence voltage, minimize the negative sequence voltage or even maximize the difference between both. These three strategies have been covered in \cite{camacho2017positive}. A more flexible approach is based on defining the objective function as
\begin{equation}
    f(\underline{I}^+, \underline{I}^-) = \lambda^+|(|\underline{V}^+_c(\underline{I}^+, \underline{I}^-)| - 1)| + \lambda^-|(|\underline{V}^-_c(\underline{I}^+, \underline{I}^-)| - 0)|,
    \label{eq:1}
\end{equation}
where the weighting factors $\lambda^+$, $\lambda^- \in \mathbb{R}$. By adjusting these factors one can follow the three aforementioned strategies. Note that the goal is to obtain a positive sequence voltage as close as possible to one (in per unit) while simultaneously approaching zero in the negative sequence voltage. Even though the problem is described as a function of the positive and negative sequence currents, it could also be stated as a function of the original $abc$ currents without loss of generality. The associated expressions are gathered in the appendix as well.

Minimizing $f$ does not come with complete freedom. That is, the currents are constrained so as not to exceed the IGBT limits. It becomes more convenient to express the constraints in the $abc$ frame:
\begin{equation}
    g(\underline{I}_a, \underline{I}_b, \underline{I}_c) = \begin{cases}
        |\underline{I}_a|\leq I_{max},\\
        |\underline{I}_b|\leq I_{max},\\
        |\underline{I}_c|\leq I_{max}.\\
    \end{cases}
\label{eq:2}
\end{equation}
For now we are not concerned with the voltages imposed to the semiconductors due to the fact that the filter $\underline{Z}_c$ is most likely to take small values. In practical situations current limitations are the most serious constraint. Therefore, we define the optimization problem as follows:
\begin{subequations}
\begin{alignat}{2}
&\!\min_{\underline{I}^+,\underline{I}^-}        &\qquad& f(\underline{I}^+,\underline{I}^-)\label{eq:optProb}\\
&\text{subject to} &      & g(\underline{I}_a, \underline{I}_b, \underline{I}_c) ,\label{eq:constraint1}
\end{alignat}
\end{subequations}
where the currents in the $abc$ frame can be related to the currents expressed in the symmetrical components form by means of Fortescue's transformation, and vice versa \cite{fortescue1918method}. As a direct consequence of that, the analysis can be performed in the $abc$ frame while expressing the voltages as a function of the positive, negative and homopolar components. Another valid procedure is to work with the symmetrical components and relate the currents to the $abc$ ones. Both ways to confront the problem are equally valid. The results obtained in this study have been generated and validated with both paths.

Current grid codes define the low voltage ride through (LVRT) limit curve which indicates the relation between the duration of a fault and the voltage frontier at which for instance a wind turbine ought to disconnect. Such LVRT curves present slight variations from country to country \cite{tsili2009review} but they all share the same pattern: in case of a severe fault, the disconnection should not take place if it has a low duration; on the contrary, less noticeable faults imply a larger disconnection time. Grid codes usually impose a curtailment of active power and enforce the generation of reactive power in order to collaborate on raising the voltage \cite{altin2010overview,serban2016voltage}.  % review grid codes and explain their priorization

Nevertheless, studies dealing with determining the optimal currents to inject are not precisely numerous. At most, some authors express the voltages in terms of active and reactive power but do not consider constraints \cite{camacho2012flexible}. Others formulate the optimization problem and arrive to a closed-form expression, even though it becomes iterative \cite{camacho2017positive}. The analysis is performed contemplating powers rather than currents. Because of that, this work focuses on computing the optimal currents to increase the positive sequence voltage as close to one as possible and achieving a negative sequence voltage that approaches zero. In essence, the results that follow come from solving Equations \ref{eq:optProb} and \ref{eq:constraint1}.

% The point of common coupling (PCC) is where the VSC together with its filter are connected. There are two restrictions to take into account in a VSC, one related to the maximum current and another to the voltage. The current limitation is likely the most relevant when operating under faults. Since the current is limited and the filter used to connect the VSC to the PCC takes rather low values, we can expect the voltage drop to not be substantial. Because of that, and taking into consideration the voltage sag at the grid side, the voltage limit is hardly ever surpassed. Thus, in the analysis that follows, we only impose the current restriction. As future work, we could also add the voltage limitations as a constraint, although probably the conclusions will not vary from the ones extracted here.

% Note that Figure \ref{fig:sys_p} is general, in the sense that it does not specify the type of fault. Besides, there will be a fault impedance, denoted by $\underline{Z}_f$. We believe every type of fault deserves to be studied separately. We are going to employ the symmetrical components, which are meant to simplify the analysis. In each fault, the voltage $\underline{V}_c$ will be decomposed in positive and negative sequence voltage and expressed as a function of the voltage at the grid together with the injected positive and negative sequence currents. It makes sense to model the VSC and its filter as a current source for this purpose. 

Four types of faults are considered in this study. One is the balanced fault, which yields a simple yet valuable system to analyze the distribution of optimal currents. The remaining faults are unbalanced: the line to ground, the line to line and the double line to ground cases. 

In all situations, we show plots representing the objective function depending on the real and imaginary parts of the positive and negative sequence currents. They are obtained in a brute force manner, that is, we generate multiple combinations of currents and store those points if they do not respect the constraints. The solution obtained by solving the optimization problem as such with the SciPy library is also displayed, together with optimal point supposing no active currents can be injected. The latter tries to emulate the outcome of following the grid codes to improve voltages. Note that grid codes also consider the injection of active power, but this serves the purpose of maintaining a stable system in terms of frequency, something not contemplated in this analysis. 

For this and all the upcoming faults we have set the values shown in Table \ref{tab:val}. 

\begin{table}[!htb] \centering
    \begin{tabular}{cc}
       \hline
       Magnitude & Value (pu) \\
       \hline
       $\underline{Z}_f$ & $0.00 + 0.10j$ \\ 
       $\underline{Z}_a$ & $0.01 + 0.10j$ \\
       $\underline{Z}_{th}$ & $0.01 + 0.05j$ \\
       $I_{max}$ & $1.00$ \\
       $\lambda^+$ & $1.00$ \\
       $\lambda^-$ & $1.00$ \\
       $|\underline{V}^+_{th}|$ & $1.00$ \\
       $|\underline{V}^-_{th}|$ & $0.00$ \\
       \hline
    \end{tabular}
    \caption{Values for the system under study}
    \label{tab:val}
\end{table}
We have considered the impedances to be mainly inductive. However, adding a resistive part not only makes them more realistic, but it also may shed some light on if the truly optimal decision is to inject only reactive currents. One could anticipate that the larger the resistive part becomes, the greater the active current should be to cause a considerable voltage drop (positive or negative) so that the voltage is improved. Thus, for the given impedances in Table \ref{tab:val}, we foresee the fact that reactive currents will turn out to be substantially larger than active currents. Besides, the same weighting is arbitrarily attributed to the positive sequence minimization subfunction as well as to the negative sequence one. 
% here display the plots for lambda1 and lambda2 = 1, but in the annex we can show the graphs for lambda1 = 0 and lambda2 = 1; and lambda1 = 1 and lambda2 = 0.

\subsection{Balanced fault}
Balanced faults are commonly referred to as the most severe type of fault, as currents take the largest values \cite{kothari2003modern}. Its representation in symmetrical components indicates a decoupling between sequences. As a result of that, a considerable positive sequence current should be injected to increase the positive sequence voltage. It should be mostly inductive. On the other hand, no negative sequence current has to be injected to keep a null negative sequence voltage (assuming that the grid presents no negative sequence voltage, as it is the case here). There are no homopolar currents in this and all the subsequent faults. 

Figure \ref{fig:3x1} shows on the one hand the results with the brute force methodology. We have employed over $20^4$ combinations to construct the plots. A more regular shape would be obtained in case we worked with smaller intervals. However, some points are suppressed because they exceed the current limitations. It is clear that the minimum is achieved when the negative sequence currents, both real and imaginary, tend to zero. The imaginary positive sequence current takes extreme negative values - close to $I_{max}$ - while the real part remains small.

% On the other hand, the point corresponding to the solution of the optimization problem illustrates what was already deduced from the brute force computations. Injecting a negative imaginary positive sequence current causes a positive voltage drop so that the voltage we wish to improve can be far apart from the faulted one. Taking into account that in this case $X>>R$, the real positive sequence current becomes considerably smaller than the imaginary part. 

% graph of brute force for the 4 currents: I1re, I1im, I2re, I2im. At the same plot show the point obtained with the optimization and the one we would get with the grid codes (only reactive current, but positive or negative??) Indicate that the objective function is a bit better with my optimization than following the grid code. 
% prepare plots template first
% plot the point where only reactive current and 0 active current. Set the constraint in the code. 

\begin{figure}[!htb]\centering \footnotesize
  \begin{subfigure}[!htb]{.4\textwidth}
    \centering
        \begin{tikzpicture}[trim axis right,trim axis left]
            \pgfplotsset{width=7cm, height=6cm}
            \begin{axis}[grid=major, xlabel={${I}^+_{re}$}, ylabel={$f$}, /pgf/number format/.cd, legend style={at={(0.98,0.15)},anchor=south east,legend columns=1, draw=none, inner sep=0pt,fill=gray!10}, xtick={-1,-0.5,...,1}, ytick={0.1,0.2,...,1}, scatter/classes={a={mark=o,draw=black, mark size=1pt}, b={mark=x,draw=red, mark size=2pt}, c={mark=square,draw=orange, mark size=1.5pt}},  scatter src=explicit symbolic, axis line style = very thick, legend style={at={(1.03,-0.03)},anchor=north west}]
            \addplot[thick, scatter, only marks, each nth point = 100] table[x=x, y=y, meta=label, col sep=comma] {Data/I1_re_3x.csv};
            \addplot[thick, scatter, only marks] table[x=x, y=y, meta=label, col sep=comma] {Data/I1_re_3x_2.csv};
            \addplot[thick, scatter, only marks] table[x=x, y=y, meta=label, col sep=comma] {Data/I1_re_3x_3.csv};
            % \legend{BF, OPT};
            \end{axis}
        \end{tikzpicture}
  \end{subfigure}
  \hspace{1cm}
\begin{subfigure}[!htb]{.4\textwidth}
    \centering
        \begin{tikzpicture}[trim axis right,trim axis left]
            \pgfplotsset{width=7cm, height=6cm}
            \begin{axis}[grid=major, xlabel={${I}^+_{im}$}, ylabel={$f$}, /pgf/number format/.cd, legend style={at={(0.98,0.15)},anchor=south east,legend columns=1, draw=none, inner sep=0pt,fill=gray!10},xtick={-1,-0.5,...,1}, ytick={0.1,0.2,...,1}, scatter/classes={a={mark=o,draw=black, mark size=1pt}, b={mark=x,draw=red, mark size=2pt}, c={mark=square,draw=orange, mark size=1.5pt}},  scatter src=explicit symbolic, axis line style = very thick, legend style={at={(0.97,0.03)},anchor=south east}]
            \addplot[thick, scatter, only marks, each nth point = 100] table[x=x, y=y, meta=label, col sep=comma] {Data/I1_im_3x.csv};
            \addplot[thick, scatter, only marks] table[x=x, y=y, meta=label, col sep=comma] {Data/I1_im_3x_2.csv};
            \addplot[thick, scatter, only marks] table[x=x, y=y, meta=label, col sep=comma] {Data/I1_im_3x_3.csv};
            \legend{BF, OPT, ROPT};
            \end{axis}
        \end{tikzpicture}
  \end{subfigure}
  \vspace{0.5cm}
\begin{subfigure}[!htb]{.4\textwidth}
    \centering
        \begin{tikzpicture}[trim axis right,trim axis left]
            \pgfplotsset{width=7cm, height=6cm}
            \begin{axis}[grid=major, xlabel={${I}^-_{re}$}, ylabel={$f$}, /pgf/number format/.cd, legend style={at={(0.98,0.15)},anchor=south east,legend columns=1, draw=none, inner sep=0pt,fill=gray!10}, xtick={-1,-0.5,...,1}, ytick={0.1,0.2,...,1}, scatter/classes={a={mark=o,draw=black, mark size=1pt}, b={mark=x,draw=red, mark size=2pt}, c={mark=square,draw=orange, mark size=1.5pt}},  scatter src=explicit symbolic, axis line style = very thick, legend style={at={(1.03,-0.03)},anchor=north west}]
            \addplot[thick, scatter, only marks, each nth point = 100] table[x=x, y=y, meta=label, col sep=comma] {Data/I2_re_3x.csv};
            \addplot[thick, scatter, only marks] table[x=x, y=y, meta=label, col sep=comma] {Data/I2_re_3x_3.csv};
            \addplot[thick, scatter, only marks] table[x=x, y=y, meta=label, col sep=comma] {Data/I2_re_3x_2.csv};
            \end{axis}
        \end{tikzpicture}
  \end{subfigure}
  \hspace{1cm}
\begin{subfigure}[!htb]{.4\textwidth}
    \centering
        \begin{tikzpicture}[trim axis right,trim axis left]
            \pgfplotsset{width=7cm, height=6cm}
            \begin{axis}[grid=major, xlabel={${I}^-_{im}$}, ylabel={$f$}, /pgf/number format/.cd, legend style={at={(0.98,0.15)},anchor=south east,legend columns=1, draw=none, inner sep=0pt,fill=gray!10}, xtick={-1,-0.5,...,1}, ytick={0.1,0.2,...,1}, scatter/classes={a={mark=o,draw=black, mark size=1pt}, b={mark=x,draw=red, mark size=2pt}, c={mark=square,draw=orange, mark size=1.5pt}},  scatter src=explicit symbolic, axis line style = very thick, legend style={at={(1.03,-0.03)},anchor=north west}]
            \addplot[thick, scatter, only marks, each nth point = 100] table[x=x, y=y, meta=label, col sep=comma] {Data/I2_im_3x.csv};
            \addplot[thick, scatter, only marks] table[x=x, y=y, meta=label, col sep=comma] {Data/I2_im_3x_3.csv};
            \addplot[thick, scatter, only marks] table[x=x, y=y, meta=label, col sep=comma] {Data/I2_im_3x_2.csv};
            \end{axis}
        \end{tikzpicture}
  \end{subfigure}
  \caption{Influence of the currents on the objective function for the balanced fault when $\lambda^+=1$ and $\lambda^-=1$. BF: brute force, OPT: solution to the optimization problem, ROPT: solution to the optimization problem restricted to only injecting reactive power.}
  \label{fig:3x1}
\end{figure}
On the other hand, the point corresponding to the solution of the optimization problem illustrates what was already more or less deduced from the brute force computations. Nonetheless, solving the optimization problem yields a more favorable result. The objective function is slightly smaller and the optimal point can be located in a zone where not many points coming from the brute force are present. Such irregular distribution of points is due to the fact that around the optimal points many combinations of currents do not meet the constraints. In any case, injecting a negative imaginary positive sequence current causes a positive voltage drop so that the voltage we wish to improve can be far apart from the faulted one. Taking into account that in this case $X>>R$, the real positive sequence current becomes considerably smaller than the imaginary part. 

The brute force calculation in Figure \ref{fig:3x1} was computed considering that $\lambda^+=1$ and $\lambda^-=1$ as well. However, the optimality can also be studied for various $\lambda$ values. This generic parameter would fit in the objective function as
\begin{equation}
    f(\underline{I}^+, \underline{I}^-) = \lambda|(|\underline{V}^+_c(\underline{I}^+, \underline{I}^-)| - 1)| + (1-\lambda)|(|\underline{V}^-_c(\underline{I}^+, \underline{I}^-)| - 0)|,
    \label{eq:1}
\end{equation}
where $\lambda=[0,1]$. Therefore, bigger values of $\lambda$ would imply that we prioritize the positive sequence voltage while small values will tend to give more importance to the negative sequence voltage. Figure \ref{fig:full_3x} shows the voltage profiles for a sweep of $\lambda$ values. 

\pgfplotsset{
colormap={whitered}{color(0cm)=(white); color(1cm)=(orange!75!red)}
}

\begin{figure}[!htb]\centering \footnotesize
\begin{tikzpicture}
\begin{axis}[%
    colormap name=whitered,
    width=12cm,
    height=8.5cm,
    view={45}{30},
    enlargelimits=false,
    grid=major,
    domain=-1:4,
    y domain=-1:4,
    samples=26,
    ztick={0.0,0.15,...,1},
    zmin=-0.05,
    zmax=0.9,
    xlabel=$\lambda^+ \equiv \lambda$,
    ylabel=$\lambda^- \equiv 1 -\lambda$,
    zlabel={$|V_c|$},
    axis line style = thick,
    x dir=reverse,
    legend style={at={(1.06,0.5)},anchor=south west,legend columns=1, draw=none, inner sep=0pt,fill=gray!10},
    colorbar,
    colorbar style={
        at={(1.06,0.03)},
        anchor=south west,
        height=0.30*\pgfkeysvalueof{/pgfplots/parent axis height},
        title={$f$}
    }
]

\addplot3 [domain=-0:1,samples=31, samples y=0, very thick, smooth, densely dashed, black]  table[x=x, y=y, z=z, col sep=comma] {Data/constant/V1_3x.csv};
\addplot3 [domain=-0:1,samples=31, samples y=0, very thick, smooth, densely dotted, black] table[x=x, y=y, z=z, col sep=comma] {Data/constant/V2_3x.csv};
\addplot3 [domain=-0:1,samples=31, samples y=0, very thick, smooth, densely dashed, gray]  table[x=x, y=y, z=z, col sep=comma] {Data/constant/RV1_3x.csv};
\addplot3 [domain=-0:1,samples=31, samples y=0, very thick, smooth, densely dotted, gray] table[x=x, y=y, z=z, col sep=comma] {Data/constant/RV2_3x.csv};
\addplot3 [scatter, only marks, ycomb, each nth point = 2] table[x=x, y=y, z=z, col sep=comma, forget plot] {Data/constant/ff_3x.csv};
\addplot3 [scatter, only marks, ycomb, each nth point = 2] table[x=x, y=y, z=z, col sep=comma, forget plot] {Data/constant/Rff_3x.csv};
\addplot3 [gray, no markers, line width=1pt] table[x=x, y=y, z=z, col sep=comma, forget plot] {Data/constant/terra_3x.csv};

% \node at (0.5,0.1,0.3) [pin=165:$P(x_1)$] {};
% \node at (0.5,0.1,0.2) [pin=85:$P(x_2)$] {};
% \node at (0.5,0.5,0.1) [pin=165:$P(x_3)$] {};

\legend{$V^+_c$ OPT, $V^-_c$ OPT, $V^+_c$ ROPT, $V^-_c$ ROPT};

\end{axis}
\end{tikzpicture}
\caption{Sequence voltages together with the objective function for the balanced fault}
\label{fig:full_3x}
\end{figure}

\subsection{Line to ground fault}
The line to ground fault has the particularity of presenting a distribution of dots similar for both real and imaginary currents for both sequences. As shown in Figure \ref{fig:LGx1}, the minimum in the real currents plot takes place around the zero. Despite that, in reality, the real positive sequence current becomes slightly larger than zero, whilst for the negative sequence it takes a small but not negligible negative value. Observing Figure \ref{fig:sys_LG} it becomes clear that the real part of the positive sequence currents ought to be greater than zero whereas the negative sequence current should take negative values; these combinations cause the maximization of the positive sequence voltage and the minimization of the negative sequence voltage. 

\begin{figure}[!htb]\centering \footnotesize
  \begin{subfigure}[!htb]{.4\textwidth}
    \centering
        \begin{tikzpicture}[trim axis right,trim axis left]
            \pgfplotsset{width=7cm, height=6cm}
            \begin{axis}[grid=major, xlabel={${I}^+_{re}$}, ylabel={$f$}, /pgf/number format/.cd, legend style={at={(0.98,0.15)},anchor=south east,legend columns=1, draw=none, inner sep=0pt,fill=gray!10}, xtick={-1,-0.5,...,1}, ytick={0.1,0.2,...,1}, scatter/classes={a={mark=o,draw=black, mark size=1pt}, b={mark=x,draw=red, mark size=2pt}, c={mark=square,draw=orange, mark size=1.5pt}},  scatter src=explicit symbolic, axis line style = very thick, legend style={at={(1.03,-0.03)},anchor=north west}]
            \addplot[thick, scatter, only marks, each nth point = 100] table[x=x, y=y, meta=label, col sep=comma] {Data/I1_re_LG.csv};
            \addplot[thick, scatter, only marks] table[x=x, y=y, meta=label, col sep=comma] {Data/I1_re_LG_2.csv};
            \addplot[thick, scatter, only marks] table[x=x, y=y, meta=label, col sep=comma] {Data/I1_re_LG_3.csv};
            % \legend{BF, OPT};
            \end{axis}
        \end{tikzpicture}
  \end{subfigure}
  \hspace{1cm}
\begin{subfigure}[!htb]{.4\textwidth}
    \centering
        \begin{tikzpicture}[trim axis right,trim axis left]
            \pgfplotsset{width=7cm, height=6cm}
            \begin{axis}[grid=major, xlabel={${I}^+_{im}$}, ylabel={$f$}, /pgf/number format/.cd, legend style={at={(0.98,0.15)},anchor=south east,legend columns=1, draw=none, inner sep=0pt,fill=gray!10},xtick={-1,-0.5,...,1}, ytick={0.1,0.2,...,1}, scatter/classes={a={mark=o,draw=black, mark size=1pt}, b={mark=x,draw=red, mark size=2pt}, c={mark=square,draw=orange, mark size=1.5pt}},  scatter src=explicit symbolic, axis line style = very thick, legend style={at={(0.97,0.03)},anchor=south east}]
            \addplot[thick, scatter, only marks, each nth point = 100] table[x=x, y=y, meta=label, col sep=comma] {Data/I1_im_LG.csv};
            \addplot[thick, scatter, only marks] table[x=x, y=y, meta=label, col sep=comma] {Data/I1_im_LG_2.csv};
            \addplot[thick, scatter, only marks] table[x=x, y=y, meta=label, col sep=comma] {Data/I1_im_LG_3.csv};
            \legend{BF, OPT, ROPT};
            \end{axis}
        \end{tikzpicture}
  \end{subfigure}
  \vspace{0.5cm}
\begin{subfigure}[!htb]{.4\textwidth}
    \centering
        \begin{tikzpicture}[trim axis right,trim axis left]
            \pgfplotsset{width=7cm, height=6cm}
            \begin{axis}[grid=major, xlabel={${I}^-_{re}$}, ylabel={$f$}, /pgf/number format/.cd, legend style={at={(0.98,0.15)},anchor=south east,legend columns=1, draw=none, inner sep=0pt,fill=gray!10}, xtick={-1,-0.5,...,1}, ytick={0.1,0.2,...,1}, scatter/classes={a={mark=o,draw=black, mark size=1pt}, b={mark=x,draw=red, mark size=2pt}, c={mark=square,draw=orange, mark size=1.5pt}},  scatter src=explicit symbolic, axis line style = very thick, legend style={at={(1.03,-0.03)},anchor=north west}]
            \addplot[thick, scatter, only marks, each nth point = 100] table[x=x, y=y, meta=label, col sep=comma] {Data/I2_re_LG.csv};
            \addplot[thick, scatter, only marks] table[x=x, y=y, meta=label, col sep=comma] {Data/I2_re_LG_3.csv};
            \addplot[thick, scatter, only marks] table[x=x, y=y, meta=label, col sep=comma] {Data/I2_re_LG_2.csv};
            \end{axis}
        \end{tikzpicture}
  \end{subfigure}
  \hspace{1cm}
\begin{subfigure}[!htb]{.4\textwidth}
    \centering
        \begin{tikzpicture}[trim axis right,trim axis left]
            \pgfplotsset{width=7cm, height=6cm}
            \begin{axis}[grid=major, xlabel={${I}^-_{im}$}, ylabel={$f$}, /pgf/number format/.cd, legend style={at={(0.98,0.15)},anchor=south east,legend columns=1, draw=none, inner sep=0pt,fill=gray!10}, xtick={-1,-0.5,...,1}, ytick={0.1,0.2,...,1}, scatter/classes={a={mark=o,draw=black, mark size=1pt}, b={mark=x,draw=red, mark size=2pt}, c={mark=square,draw=orange, mark size=1.5pt}},  scatter src=explicit symbolic, axis line style = very thick, legend style={at={(1.03,-0.03)},anchor=north west}]
            \addplot[thick, scatter, only marks, each nth point = 100] table[x=x, y=y, meta=label, col sep=comma] {Data/I2_im_LG.csv};
            \addplot[thick, scatter, only marks] table[x=x, y=y, meta=label, col sep=comma] {Data/I2_im_LG_3.csv};
            \addplot[thick, scatter, only marks] table[x=x, y=y, meta=label, col sep=comma] {Data/I2_im_LG_2.csv};
            \end{axis}
        \end{tikzpicture}
  \end{subfigure}
  \caption{Influence of the currents on the objective function for the line to ground fault. BF: brute force, OPT: solution to the optimization problem, ROPT: solution to the optimization problem restricted to only injecting reactive power.}
  \label{fig:LGx1}
\end{figure}
The imaginary part of the currents presents a similar distribution in both cases. Notice that contrarily to the balanced fault, where one vertex of the distribution coincided with the point at which $f$ was at its minimum, there is a full edge in which the function becomes minimum. This phenomena suggests that maybe there is more than one minimum. For instance, carrying the analysis in the $abc$ frame caused the optimal currents $[\underline{I}^+, \underline{I}^-]$ to become $[0.1715-j0.4955, -0.0804-j0.5003]$. Leaving aside the differences in current, the objective function became the same up to a precision of $10^{-10}$. We have tested that this is indeed not an abnormality. Initializing differently the currents passed to the SciPy \texttt{minimize()} function results in variations in the imaginary currents across the zone where $f$ becomes minimum. Besides, the variations between the OPT and the ROPT cases are almost imperceptible. 

Figure \ref{fig:full_LG} shows the distribution of voltages and the objective function across multiple values of $\lambda$.

\begin{figure}[!htb]\centering \footnotesize
\begin{tikzpicture}
\begin{axis}[%
    colormap name=whitered,
    width=12cm,
    height=8.5cm,
    view={45}{30},
    enlargelimits=false,
    grid=major,
    domain=-1:4,
    y domain=-1:4,
    samples=26,
    ztick={0.0,0.15,...,1},
    zmin=0.00,
    zmax=1.0,
    xlabel=$\lambda^+ \equiv \lambda$,
    ylabel=$\lambda^- \equiv 1 -\lambda$,
    zlabel={$|V_c|$},
    axis line style = thick,
    x dir=reverse,
    legend style={at={(1.06,0.5)},anchor=south west,legend columns=1, draw=none, inner sep=0pt,fill=gray!10},
    colorbar,
    colorbar style={
        at={(1.06,0.03)},
        anchor=south west,
        height=0.30*\pgfkeysvalueof{/pgfplots/parent axis height},
        title={$f$}
    }
]

\addplot3 [domain=-0:1,samples=31, samples y=0, very thick, smooth, densely dashed, black]  table[x=x, y=y, z=z, col sep=comma] {Data/constant/V1_LG.csv};
\addplot3 [domain=-0:1,samples=31, samples y=0, very thick, smooth, densely dotted, black] table[x=x, y=y, z=z, col sep=comma] {Data/constant/V2_LG.csv};
\addplot3 [domain=-0:1,samples=31, samples y=0, very thick, smooth, densely dashed, gray]  table[x=x, y=y, z=z, col sep=comma] {Data/constant/RV1_LG.csv};
\addplot3 [domain=-0:1,samples=31, samples y=0, very thick, smooth, densely dotted, gray] table[x=x, y=y, z=z, col sep=comma] {Data/constant/RV2_LG.csv};
\addplot3 [scatter, only marks, ycomb, each nth point = 2] table[x=x, y=y, z=z, col sep=comma, forget plot] {Data/constant/ff_LG.csv};
\addplot3 [scatter, only marks, ycomb, each nth point = 2] table[x=x, y=y, z=z, col sep=comma, forget plot] {Data/constant/Rff_LG.csv};
\addplot3 [gray, no markers, line width=1pt] table[x=x, y=y, z=z, col sep=comma, forget plot] {Data/constant/terra_LG.csv};

% \node at (0.5,0.1,0.3) [pin=165:$P(x_1)$] {};
% \node at (0.5,0.1,0.2) [pin=85:$P(x_2)$] {};
% \node at (0.5,0.5,0.1) [pin=165:$P(x_3)$] {};

\legend{$V^+_c$ OPT, $V^-_c$ OPT, $V^+_c$ ROPT, $V^-_c$ ROPT};

\end{axis}
\end{tikzpicture}
\caption{Sequence voltages together with the objective function for the line to ground fault}
\label{fig:full_LG}
\end{figure}

\subsection{Line to line fault}
The line to line fault can be considered to be a more severe fault compared to the line to ground case, since the function to minimize presents larger values. This can be understood when looking at the representations in Figures \ref{fig:3_lg} and \ref{fig:3_ll}, where the fault impedance is between two phases and not phase and ground. Therefore, the voltage drop becomes larger. Figure \ref{fig:LLx1} shows that even if the real part of the currents remains near the zero, visually speaking the imaginary parts take symmetrical values. The distribution of dots for the imaginary part of the negative sequence current is the main difference with respect to the line to ground fault. 

\begin{figure}[!htb]\centering \footnotesize
  \begin{subfigure}[!htb]{.4\textwidth}
    \centering
        \begin{tikzpicture}[trim axis right,trim axis left]
            \pgfplotsset{width=7cm, height=6cm}
            \begin{axis}[grid=major, xlabel={${I}^+_{re}$}, ylabel={$f$}, /pgf/number format/.cd, legend style={at={(0.98,0.15)},anchor=south east,legend columns=1, draw=none, inner sep=0pt,fill=gray!10}, xtick={-1,-0.5,...,1}, ytick={0.1,0.2,...,1}, scatter/classes={a={mark=o,draw=black, mark size=1pt}, b={mark=x,draw=red, mark size=2pt}, c={mark=square,draw=orange, mark size=1.5pt}},  scatter src=explicit symbolic, axis line style = very thick, legend style={at={(1.03,-0.03)},anchor=north west}]
            \addplot[thick, scatter, only marks, each nth point = 100] table[x=x, y=y, meta=label, col sep=comma] {Data/I1_re_LL.csv};
            \addplot[thick, scatter, only marks] table[x=x, y=y, meta=label, col sep=comma] {Data/I1_re_LL_2.csv};
            \addplot[thick, scatter, only marks] table[x=x, y=y, meta=label, col sep=comma] {Data/I1_re_LL_3.csv};
            % \legend{BF, OPT};
            \end{axis}
        \end{tikzpicture}
  \end{subfigure}
  \hspace{1cm}
\begin{subfigure}[!htb]{.4\textwidth}
    \centering
        \begin{tikzpicture}[trim axis right,trim axis left]
            \pgfplotsset{width=7cm, height=6cm}
            \begin{axis}[grid=major, xlabel={${I}^+_{im}$}, ylabel={$f$}, /pgf/number format/.cd, legend style={at={(0.98,0.15)},anchor=south east,legend columns=1, draw=none, inner sep=0pt,fill=gray!10},xtick={-1,-0.5,...,1}, ytick={0.1,0.2,...,1}, scatter/classes={a={mark=o,draw=black, mark size=1pt}, b={mark=x,draw=red, mark size=2pt}, c={mark=square,draw=orange, mark size=1.5pt}},  scatter src=explicit symbolic, axis line style = very thick, legend style={at={(0.97,0.03)},anchor=south east}]
            \addplot[thick, scatter, only marks, each nth point = 100] table[x=x, y=y, meta=label, col sep=comma] {Data/I1_im_LL.csv};
            \addplot[thick, scatter, only marks] table[x=x, y=y, meta=label, col sep=comma] {Data/I1_im_LL_2.csv};
            \addplot[thick, scatter, only marks] table[x=x, y=y, meta=label, col sep=comma] {Data/I1_im_LL_3.csv};
            \legend{BF, OPT, ROPT};
            \end{axis}
        \end{tikzpicture}
  \end{subfigure}
  \vspace{0.5cm}
\begin{subfigure}[!htb]{.4\textwidth}
    \centering
        \begin{tikzpicture}[trim axis right,trim axis left]
            \pgfplotsset{width=7cm, height=6cm}
            \begin{axis}[grid=major, xlabel={${I}^-_{re}$}, ylabel={$f$}, /pgf/number format/.cd, legend style={at={(0.98,0.15)},anchor=south east,legend columns=1, draw=none, inner sep=0pt,fill=gray!10}, xtick={-1,-0.5,...,1}, ytick={0.1,0.2,...,1}, scatter/classes={a={mark=o,draw=black, mark size=1pt}, b={mark=x,draw=red, mark size=2pt}, c={mark=square,draw=orange, mark size=1.5pt}},  scatter src=explicit symbolic, axis line style = very thick, legend style={at={(1.03,-0.03)},anchor=north west}]
            \addplot[thick, scatter, only marks, each nth point = 100] table[x=x, y=y, meta=label, col sep=comma] {Data/I2_re_LL.csv};
            \addplot[thick, scatter, only marks] table[x=x, y=y, meta=label, col sep=comma] {Data/I2_re_LL_3.csv};
            \addplot[thick, scatter, only marks] table[x=x, y=y, meta=label, col sep=comma] {Data/I2_re_LL_2.csv};
            \end{axis}
        \end{tikzpicture}
  \end{subfigure}
  \hspace{1cm}
\begin{subfigure}[!htb]{.4\textwidth}
    \centering
        \begin{tikzpicture}[trim axis right,trim axis left]
            \pgfplotsset{width=7cm, height=6cm}
            \begin{axis}[grid=major, xlabel={${I}^-_{im}$}, ylabel={$f$}, /pgf/number format/.cd, legend style={at={(0.98,0.15)},anchor=south east,legend columns=1, draw=none, inner sep=0pt,fill=gray!10}, xtick={-1,-0.5,...,1}, ytick={0.1,0.2,...,1}, scatter/classes={a={mark=o,draw=black, mark size=1pt}, b={mark=x,draw=red, mark size=2pt}, c={mark=square,draw=orange, mark size=1.5pt}},  scatter src=explicit symbolic, axis line style = very thick, legend style={at={(1.03,-0.03)},anchor=north west}]
            \addplot[thick, scatter, only marks, each nth point = 100] table[x=x, y=y, meta=label, col sep=comma] {Data/I2_im_LL.csv};
            \addplot[thick, scatter, only marks] table[x=x, y=y, meta=label, col sep=comma] {Data/I2_im_LL_3.csv};
            \addplot[thick, scatter, only marks] table[x=x, y=y, meta=label, col sep=comma] {Data/I2_im_LL_2.csv};
            \end{axis}
        \end{tikzpicture}
  \end{subfigure}
  \caption{Influence of the currents on the objective function for the line to line fault. BF: brute force, OPT: solution to the optimization problem, ROPT: solution to the optimization problem restricted to only injecting reactive power.}
  \label{fig:LLx1}
\end{figure}
However, this time it has been checked that there is only one point that minimizes the function, independently of the initialization vector. Not less surprising, the imaginary positive and negative sequence currents are practically the same with a change of sign. We can intuitively make sense of it as the positive sequence voltage at the PCC tends to one while the negative one approaches zero. Thus, the current flowing towards the $\underline{Z}_f$ impedance is quite large and a big part of it is due to the injected positive sequence current. To achieve a small negative sequence voltage, as can be understood from Figure \ref{fig:sys_LL}, the negative sequence current has to counteract the positive sequence current. Because of that, the optimal positive and negative sequence currents share almost the same magnitude and opposite signs.

Figure \ref{fig:full_LL} depicts the plot where the positive and negative sequence voltages as well as the objective function are plotted.

\begin{figure}[!htb]\centering \footnotesize
\begin{tikzpicture}
\begin{axis}[%
    colormap name=whitered,
    width=12cm,
    height=8.5cm,
    view={45}{30},
    enlargelimits=false,
    grid=major,
    domain=-1:4,
    y domain=-1:4,
    samples=26,
    ztick={0.0,0.15,...,1},
    zmin=0.00,
    zmax=1.0,
    xlabel=$\lambda^+ \equiv \lambda$,
    ylabel=$\lambda^- \equiv 1 -\lambda$,
    zlabel={$|V_c|$},
    axis line style = thick,
    x dir=reverse,
    legend style={at={(1.06,0.5)},anchor=south west,legend columns=1, draw=none, inner sep=0pt,fill=gray!10},
    colorbar,
    colorbar style={
        at={(1.06,0.03)},
        anchor=south west,
        height=0.30*\pgfkeysvalueof{/pgfplots/parent axis height},
        title={$f$}
    }
]

\addplot3 [domain=-0:1,samples=31, samples y=0, very thick, smooth, densely dashed, black]  table[x=x, y=y, z=z, col sep=comma] {Data/constant/V1_LL.csv};
\addplot3 [domain=-0:1,samples=31, samples y=0, very thick, smooth, densely dotted, black] table[x=x, y=y, z=z, col sep=comma] {Data/constant/V2_LL.csv};
\addplot3 [domain=-0:1,samples=31, samples y=0, very thick, smooth, densely dashed, gray]  table[x=x, y=y, z=z, col sep=comma] {Data/constant/RV1_LL.csv};
\addplot3 [domain=-0:1,samples=31, samples y=0, very thick, smooth, densely dotted, gray] table[x=x, y=y, z=z, col sep=comma] {Data/constant/RV2_LL.csv};
\addplot3 [scatter, only marks, ycomb, each nth point = 2] table[x=x, y=y, z=z, col sep=comma, forget plot] {Data/constant/ff_LL.csv};
\addplot3 [scatter, only marks, ycomb, each nth point = 2] table[x=x, y=y, z=z, col sep=comma, forget plot] {Data/constant/Rff_LL.csv};
\addplot3 [gray, no markers, line width=1pt] table[x=x, y=y, z=z, col sep=comma, forget plot] {Data/constant/terra_LL.csv};

% \node at (0.5,0.1,0.3) [pin=165:$P(x_1)$] {};
% \node at (0.5,0.1,0.2) [pin=85:$P(x_2)$] {};
% \node at (0.5,0.5,0.1) [pin=165:$P(x_3)$] {};

\legend{$V^+_c$ OPT, $V^-_c$ OPT, $V^+_c$ ROPT, $V^-_c$ ROPT};

\end{axis}
\end{tikzpicture}
\caption{Sequence voltages together with the objective function for the line to line fault}
\label{fig:full_LL}
\end{figure}

\subsection{Double line to ground fault}
The double line to ground fault is likely to be the most severe in terms of voltage. It turns out to have the more distant voltages from the references, as shown in \cite{taul2020modeling}. By nature, this fault is defined by a solid line to line connection and also a link between the two faulted phases and ground through an impedance. Figure \ref{fig:LLGx1} reveals that minimum of $f$ falls considerably far from the ideal zero. 

The double line to ground fault has many similarities with the line to line fault. First of all, the distribution of dots for the brute force computations is similar. Secondly, the optimal currents do not differ much from the ones obtained in Figure \ref{fig:LLx1}. All this could be deduced from its sequence representation. The same logic applies: the negative sequence current mirrors the positive sequence.

\begin{figure}[!htb]\centering \footnotesize
  \begin{subfigure}[!htb]{.4\textwidth}
    \centering
        \begin{tikzpicture}[trim axis right,trim axis left]
            \pgfplotsset{width=7cm, height=6cm}
            \begin{axis}[grid=major, xlabel={${I}^+_{re}$}, ylabel={$f$}, /pgf/number format/.cd, legend style={at={(0.98,0.15)},anchor=south east,legend columns=1, draw=none, inner sep=0pt,fill=gray!10}, xtick={-1,-0.5,...,1}, ytick={0.9,0.95,...,1.1}, scatter/classes={a={mark=o,draw=black, mark size=1pt}, b={mark=x,draw=red, mark size=2pt}, c={mark=square,draw=orange, mark size=1.5pt}},  scatter src=explicit symbolic, axis line style = very thick, legend style={at={(1.03,-0.03)},anchor=north west}]
            \addplot[thick, scatter, only marks, each nth point = 100] table[x=x, y=y, meta=label, col sep=comma] {Data/I1_re_LLG.csv};
            \addplot[thick, scatter, only marks] table[x=x, y=y, meta=label, col sep=comma] {Data/I1_re_LLG_2.csv};
            \addplot[thick, scatter, only marks] table[x=x, y=y, meta=label, col sep=comma] {Data/I1_re_LLG_3.csv};
            % \legend{BF, OPT};
            \end{axis}
        \end{tikzpicture}
  \end{subfigure}
  \hspace{1cm}
\begin{subfigure}[!htb]{.4\textwidth}
    \centering
        \begin{tikzpicture}[trim axis right,trim axis left]
            \pgfplotsset{width=7cm, height=6cm}
            \begin{axis}[grid=major, xlabel={${I}^+_{im}$}, ylabel={$f$}, /pgf/number format/.cd, legend style={at={(0.98,0.15)},anchor=south east,legend columns=1, draw=none, inner sep=0pt,fill=gray!10},xtick={-1,-0.5,...,1}, ytick={0.9,0.95,...,1.1}, scatter/classes={a={mark=o,draw=black, mark size=1pt}, b={mark=x,draw=red, mark size=2pt}, c={mark=square,draw=orange, mark size=1.5pt}},  scatter src=explicit symbolic, axis line style = very thick, legend style={at={(0.97,0.03)},anchor=south east}]
            \addplot[thick, scatter, only marks, each nth point = 100] table[x=x, y=y, meta=label, col sep=comma] {Data/I1_im_LLG.csv};
            \addplot[thick, scatter, only marks] table[x=x, y=y, meta=label, col sep=comma] {Data/I1_im_LLG_2.csv};
            \addplot[thick, scatter, only marks] table[x=x, y=y, meta=label, col sep=comma] {Data/I1_im_LLG_3.csv};
            \legend{BF, OPT, ROPT};
            \end{axis}
        \end{tikzpicture}
  \end{subfigure}
  \vspace{0.5cm}
\begin{subfigure}[!htb]{.4\textwidth}
    \centering
        \begin{tikzpicture}[trim axis right,trim axis left]
            \pgfplotsset{width=7cm, height=6cm}
            \begin{axis}[grid=major, xlabel={${I}^-_{re}$}, ylabel={$f$}, /pgf/number format/.cd, legend style={at={(0.98,0.15)},anchor=south east,legend columns=1, draw=none, inner sep=0pt,fill=gray!10}, xtick={-1,-0.5,...,1}, ytick={0.9,0.95,...,1.1}, scatter/classes={a={mark=o,draw=black, mark size=1pt}, b={mark=x,draw=red, mark size=2pt}, c={mark=square,draw=orange, mark size=1.5pt}},  scatter src=explicit symbolic, axis line style = very thick, legend style={at={(1.03,-0.03)},anchor=north west}]
            \addplot[thick, scatter, only marks, each nth point = 100] table[x=x, y=y, meta=label, col sep=comma] {Data/I2_re_LLG.csv};
            \addplot[thick, scatter, only marks] table[x=x, y=y, meta=label, col sep=comma] {Data/I2_re_LLG_3.csv};
            \addplot[thick, scatter, only marks] table[x=x, y=y, meta=label, col sep=comma] {Data/I2_re_LLG_2.csv};
            \end{axis}
        \end{tikzpicture}
  \end{subfigure}
  \hspace{1cm}
\begin{subfigure}[!htb]{.4\textwidth}
    \centering
        \begin{tikzpicture}[trim axis right,trim axis left]
            \pgfplotsset{width=7cm, height=6cm}
            \begin{axis}[grid=major, xlabel={${I}^-_{im}$}, ylabel={$f$}, /pgf/number format/.cd, legend style={at={(0.98,0.15)},anchor=south east,legend columns=1, draw=none, inner sep=0pt,fill=gray!10}, xtick={-1,-0.5,...,1}, ytick={0.9,0.95,...,1.1}, scatter/classes={a={mark=o,draw=black, mark size=1pt}, b={mark=x,draw=red, mark size=2pt}, c={mark=square,draw=orange, mark size=1.5pt}},  scatter src=explicit symbolic, axis line style = very thick, legend style={at={(1.03,-0.03)},anchor=north west}]
            \addplot[thick, scatter, only marks, each nth point = 100] table[x=x, y=y, meta=label, col sep=comma] {Data/I2_im_LLG.csv};
            \addplot[thick, scatter, only marks] table[x=x, y=y, meta=label, col sep=comma] {Data/I2_im_LLG_3.csv};
            \addplot[thick, scatter, only marks] table[x=x, y=y, meta=label, col sep=comma] {Data/I2_im_LLG_2.csv};
            \end{axis}
        \end{tikzpicture}
  \end{subfigure}
  \caption{Influence of the currents on the objective function for the double line to ground fault. BF: brute force, OPT: solution to the optimization problem, ROPT: solution to the optimization problem restricted to only injecting reactive power.}
  \label{fig:LLGx1}
\end{figure}
Again, only injecting reactive currents (ROPF case) has been experimentally proved to become a suboptimal yet highly convenient choice. The objective functions present tiny variations in this case. Even though the real currents in both sequence differ a bit, for the imaginary ones the OPT and the ROPT computations generate almost the same value. Therefore, the injection of real currents becomes secondary for highly inductive impedances. 

Finally, figure \ref{fig:full_LLG} displays what the distribution of voltages and the subsequent objective function look like for variations in the $\lambda$ parameter.

\begin{figure}[!htb]\centering \footnotesize
\begin{tikzpicture}
\begin{axis}[%
    colormap name=whitered,
    width=12cm,
    height=8.5cm,
    view={45}{30},
    enlargelimits=false,
    grid=major,
    domain=-1:4,
    y domain=-1:4,
    samples=26,
    ztick={0.0,0.15,...,1},
    zmin=0.00,
    zmax=1.0,
    xlabel=$\lambda^+ \equiv \lambda$,
    ylabel=$\lambda^- \equiv 1 -\lambda$,
    zlabel={$|V_c|$},
    axis line style = thick,
    x dir=reverse,
    legend style={at={(1.06,0.5)},anchor=south west,legend columns=1, draw=none, inner sep=0pt,fill=gray!10},
    colorbar,
    colorbar style={
        at={(1.06,0.03)},
        anchor=south west,
        height=0.30*\pgfkeysvalueof{/pgfplots/parent axis height},
        title={$f$}
    }
]

\addplot3 [domain=-0:1,samples=31, samples y=0, very thick, smooth, densely dashed, black]  table[x=x, y=y, z=z, col sep=comma] {Data/constant/V1_LLG.csv};
\addplot3 [domain=-0:1,samples=31, samples y=0, very thick, smooth, densely dotted, black] table[x=x, y=y, z=z, col sep=comma] {Data/constant/V2_LLG.csv};
\addplot3 [domain=-0:1,samples=31, samples y=0, very thick, smooth, densely dashed, gray]  table[x=x, y=y, z=z, col sep=comma] {Data/constant/RV1_LLG.csv};
\addplot3 [domain=-0:1,samples=31, samples y=0, very thick, smooth, densely dotted, gray] table[x=x, y=y, z=z, col sep=comma] {Data/constant/RV2_LLG.csv};
\addplot3 [scatter, only marks, ycomb, each nth point = 2] table[x=x, y=y, z=z, col sep=comma, forget plot] {Data/constant/ff_LLG.csv};
\addplot3 [scatter, only marks, ycomb, each nth point = 2] table[x=x, y=y, z=z, col sep=comma, forget plot] {Data/constant/Rff_LLG.csv};
\addplot3 [gray, no markers, line width=1pt] table[x=x, y=y, z=z, col sep=comma, forget plot] {Data/constant/terra_LLG.csv};

% \node at (0.5,0.1,0.3) [pin=165:$P(x_1)$] {};
% \node at (0.5,0.1,0.2) [pin=85:$P(x_2)$] {};
% \node at (0.5,0.5,0.1) [pin=165:$P(x_3)$] {};

\legend{$V^+_c$ OPT, $V^-_c$ OPT, $V^+_c$ ROPT, $V^-_c$ ROPT};

\end{axis}
\end{tikzpicture}
\caption{Sequence voltages together with the objective function for the double line to ground fault}
\label{fig:full_LLG}
\end{figure}

Table \ref{tab:results} gathers the numerical results for all the studied faults.
\begin{table}[!htb] \centering
    \begin{tabular}{lccccccc}
       \hline
       Fault & $f$ & $|V^+|$ & $|V^-|$ & ${I}^+_{re}$ & ${I}^+_{im}$ & ${I}^-_{re}$ & ${I}^-_{im}$\\
       \hline
       Balanced & 0.200 & 0.800 & 0.000 & 0.107 & -0.994 & 0.000 & 0.000\\
       Line to ground & 0.084 & 0.966 & 0.049 & 0.176 & -0.535 & -0.085 & -0.461\\
       Line to line & 0.361 & 0.820 & 0.180 & 0.030 & -0.571 & -0.030 & 0.582\\
       Double line to ground & 0.884 & 0.525 & 0.408 & 0.064 & -0.574 & -0.064 & 0.574\\
       \hline
    \end{tabular}
    \caption{Main results for the balanced and the unbalanced faults}
    \label{tab:results}
\end{table}
\newpage
\clearpage
% \section{Variable resistance/inductance ratio}
In this section we try to answer to the question of what happens when the $R/X$ ratio varies. This way we experiment with cases where the resistive part is considerably larger than in the aforementioned analysis. The fault impedance $\underline{Z}_f$ has been set at $0.1$, which is probably more realistic than $0.0 + 0.1j$. 

\subsection{Balanced fault}
Figure \ref{fig:3x1_c} depicts the optimal currents for the balanced fault.

\begin{figure}[!htb]\centering \footnotesize
    \begin{subfigure}[!htb]{.4\textwidth}
      \centering
          \begin{tikzpicture}[trim axis right,trim axis left]
              \pgfplotsset{width=7cm, height=5.5cm}
              \begin{axis}[grid=major, xlabel={$R/X$}, ylabel={${I}^+_{re}$}, /pgf/number format/.cd, legend style={at={(0.98,0.15)},anchor=south east,legend columns=1, draw=none, inner sep=0pt,fill=gray!10}, xtick={0,1,...,5}, axis line style = thick, ytick={0.0, 0.2,...,1.0}, xmin=0, xmax=5]
              \addplot[very thick, black] table[x=x, y=y, meta=label, col sep=comma] {Data/RX/I1_re_3x.csv};
              \addplot[very thick, gray] table[x=x, y=y, meta=label, col sep=comma] {Data/RX/RI1_re_3x.csv};
              % \legend{BF, OPT};
              \end{axis}
          \end{tikzpicture}
    \end{subfigure}
    \hspace{1.5cm}
    \begin{subfigure}[!htb]{.4\textwidth}
        \centering
            \begin{tikzpicture}[trim axis right,trim axis left]
                \pgfplotsset{width=7cm, height=5.5cm}
                \begin{axis}[grid=major, xlabel={$R/X$}, ylabel={${I}^+_{im}$}, /pgf/number format/.cd, legend style={at={(0.98,0.15)},anchor=south east,legend columns=1, draw=none, inner sep=0pt,fill=gray!10}, xtick={0,1,...,5}, axis line style = thick,  xmin=0, xmax=5]
                \addplot[very thick, black] table[x=x, y=y, meta=label, col sep=comma] {Data/RX/I1_im_3x.csv};
                \addplot[very thick, gray] table[x=x, y=y, meta=label, col sep=comma] {Data/RX/RI1_im_3x.csv};
                % \legend{OPT};
                \end{axis}
            \end{tikzpicture}
      \end{subfigure}

      \vspace{0.5cm}

      \begin{subfigure}[!htb]{.4\textwidth}
        \centering
            \begin{tikzpicture}[trim axis right,trim axis left]
                \pgfplotsset{width=7cm, height=5.5cm}
                \begin{axis}[grid=major, xlabel={$R/X$}, ylabel={${I}^-_{re}$}, /pgf/number format/.cd, legend style={at={(0.98,0.15)},anchor=south east,legend columns=1, draw=none, inner sep=0pt,fill=gray!10}, xtick={0,1,...,5}, ymin = -0.5, ymax=0.5, axis line style = thick,  xmin=0, xmax=5]
                \addplot[very thick, black] table[x=x, y=y, meta=label, col sep=comma] {Data/RX/I2_re_3x.csv};
                \addplot[very thick, gray] table[x=x, y=y, meta=label, col sep=comma] {Data/RX/RI2_re_3x.csv};
                % \legend{BF, OPT};
                \end{axis}
            \end{tikzpicture}
      \end{subfigure}
      \hspace{1.5cm}
      \begin{subfigure}[!htb]{.4\textwidth}
          \centering
              \begin{tikzpicture}[trim axis right,trim axis left]
                  \pgfplotsset{width=7cm, height=5.5cm}
                  \begin{axis}[grid=major, xlabel={$R/X$}, ylabel={${I}^-_{im}$}, /pgf/number format/.cd, legend style={at={(0.98,0.15)},anchor=south east,legend columns=1, draw=none, inner sep=0pt,fill=gray!10}, xtick={0,1,...,5}, ymin = -0.5, ymax=0.5, axis line style = thick,  xmin=0, xmax=5]
                  \addplot[very thick, black] table[x=x, y=y, meta=label, col sep=comma] {Data/RX/I2_im_3x.csv};
                  \addplot[very thick, gray] table[x=x, y=y, meta=label, col sep=comma] {Data/RX/RI2_im_3x.csv};
                  % \legend{BF, OPT};
                  \end{axis}
              \end{tikzpicture}
        \end{subfigure}

        \vspace{0.2cm}

    \begin{center}
        \begin{subfigure}[!htb]{0.7\textwidth}
                \begin{tikzpicture}[trim axis right,trim axis left]
                    \pgfplotsset{width=12.5cm, height=5.5cm}
                    \begin{axis}[grid=major, xlabel={$R/X$}, ylabel={$f$}, /pgf/number format/.cd, legend style={at={(0.98,0.2)},anchor=south east,legend columns=1, draw=none, inner sep=0pt,fill=gray!10}, xtick={0,0.5,...,5}, ymin = 0.2, axis line style = thick, yticklabel style={/pgf/number format/fixed, /pgf/number format/precision=2}, ytick={0.0,0.1,...,0.5},  xmin=0, xmax=5]
                   \addplot[very thick, black] table[x=x, y=y, meta=label, col sep=comma] {Data/RX/ff_3x.csv};
                   \addplot[very thick, gray] table[x=x, y=y, meta=label, col sep=comma] {Data/RX/Rff_3x.csv};
                    \legend{OPT, ROPT};
                    \end{axis}
                \end{tikzpicture}
        \end{subfigure}
    \end{center}
    \caption{Influence of the currents on the objective function for the balanced fault with $\underline{Z}_f=0.05$ and a changing $R/X$ ratio. OPT: solution to the optimization problem, ROPT: solution to the optimization problem restricted to only injecting reactive power.}
    \label{fig:3x1_c}
  \end{figure}
It becomes clear that the optimal current is highly dependent on the $R/X$ ratio. That is, when $X>R$, the imaginary positive sequence is dominant, and vice versa for $R>X$. This phenomena makes complete sense when considering that we want to amplify the voltage drop in the positive sequence. However, even when there are serious differences between $R$ and $X$ in magnitude, neither the real nor the imaginary part go to zero. Thus, the objective function for the OPT case becomes slightly lower when compared to the ROPT case. Besides, since the negative sequence equivalent circuit is decoupled from the positive sequence one, the negative sequence currents are null in all cases.  

The fault impedance has a noticeable effect on the results. When it is extremely small, the objective function almost does not vary. Opting for a too large fault impedance may cause the presence of multiple solutions. Hence, the real and imaginary currents may take unexpected values so it is hard to build intuition around the problem. We have checked that $\underline{Z}_f=0.05$ for the balanced fault was a convenient value. 

In addition, Figure \ref{fig:full_RX_3x} shows the absolute value of the voltages depending on the $R/X$ ratio and together with the objective function.

\pgfplotsset{
colormap={whitered}{color(0cm)=(white); color(1cm)=(orange!75!red)}
}

\begin{figure}[!htb]\centering \footnotesize
\begin{tikzpicture}
\begin{axis}[%
    colormap name=whitered,
    width=12cm,
    height=10cm,
    view={45}{30},
    enlargelimits=false,
    grid=major,
    domain=-1:4,
    y domain=-1:4,
    samples=26,
    ztick={0.0,0.15,...,1},
    zmin=-0.05,
    zmax=1.0,
    xlabel=$R/X$,
    ylabel=$R/X$,
    zlabel={$|V_c|$},
    axis line style = thick,
    legend style={at={(1.06,0.5)},anchor=south west,legend columns=1, draw=none, inner sep=0pt,fill=gray!10},
    colorbar,
    colorbar style={
        at={(1.06,0.03)},
        anchor=south west,
        height=0.30*\pgfkeysvalueof{/pgfplots/parent axis height},
        title={$f$}
    }
]

\addplot3 [domain=-0:1,samples=31, samples y=0, very thick, smooth, densely dashed, black]  table[x=x, y=y, z=z, col sep=comma] {Data/RX/V1_3x.csv};
\addplot3 [domain=-0:1,samples=31, samples y=0, very thick, smooth, densely dotted, black] table[x=x, y=y, z=z, col sep=comma] {Data/RX/V2_3x.csv};
\addplot3 [domain=-0:1,samples=31, samples y=0, very thick, smooth, densely dashed, gray]  table[x=x, y=y, z=z, col sep=comma] {Data/RX/RV1_3x.csv};
\addplot3 [domain=-0:1,samples=31, samples y=0, very thick, smooth, densely dotted, gray] table[x=x, y=y, z=z, col sep=comma] {Data/RX/RV2_3x.csv};
\addplot3 [scatter, only marks, ycomb, each nth point = 2, gray!60] table[x=x, y=y, z=z, col sep=comma, forget plot] {Data/RX/ffG_3x.csv};
\addplot3 [scatter, only marks, ycomb, each nth point = 2, gray!60] table[x=x, y=y, z=z, col sep=comma, forget plot] {Data/RX/RffG_3x.csv};
\addplot3 [gray, no markers, line width=1pt] table[x=x, y=y, z=z, col sep=comma, forget plot] {Data/RX/terra_RX.csv};

% \node at (0.5,0.1,0.3) [pin=165:$P(x_1)$] {};
% \node at (0.5,0.1,0.2) [pin=85:$P(x_2)$] {};
% \node at (0.5,0.5,0.1) [pin=165:$P(x_3)$] {};

\legend{$V^+_c$ OPT, $V^-_c$ OPT, $V^+_c$ ROPT, $V^-_c$ ROPT};

\end{axis}
\end{tikzpicture}
\caption{Sequence voltages together with the objective function for the balanced fault with $\underline{Z}_f=0.05$ and a varying $R/X$ ratio}
\label{fig:full_RX_3x}
\end{figure}
As already noted in Figure \ref{fig:3x1_c}, the objective function for the OPT case is inferior than the one for the ROPT case, and hence, closer to the ideal situation. Nevertheless, independently on having constraints on the active current, the negative sequence voltages remain at exactly zero for all the $R/X$ range. They end up being superposed. The positive sequence voltages, instead, experience some variations depending on the $R/X$ value. In some sense the positive sequence voltage trend is the contrary of the objective function pattern.

The best possible situation is having the reactive part larger than the real part of the impedance. This is when the positive sequence voltage tends to 0.75. Since the resistive characteristics of the impedance are so minimized, the ROPT case yields the same results as the OPT. When the resistance increases, the differences are exaggerated due to the fact that no active current can be injected in the ROPT situation.

\subsection{Line to ground fault}
Figure \ref{fig:LGx1_c} depicts the optimal currents for the line to ground fault.

\begin{figure}[!htb]\centering \footnotesize
    \begin{subfigure}[!htb]{.4\textwidth}
      \centering
          \begin{tikzpicture}[trim axis right,trim axis left]
              \pgfplotsset{width=7cm, height=6.0cm}
              \begin{axis}[grid=major, xlabel={$R/X$}, ylabel={${I}^+_{re}$}, /pgf/number format/.cd, legend style={at={(0.98,0.15)},anchor=south east,legend columns=1, draw=none, inner sep=0pt,fill=gray!10}, xtick={0,1,...,5}, axis line style = thick, ytick={-0.0,0.1,...,0.5}, yticklabel style={/pgf/number format/fixed, /pgf/number format/precision=2}, xmin=0, xmax=5]
              \addplot[very thick, black] table[x=x, y=y, meta=label, col sep=comma] {Data/RX/I1_re_LG.csv};
              \addplot[very thick, gray] table[x=x, y=y, meta=label, col sep=comma] {Data/RX/RI1_re_LG.csv};
              % \legend{BF, OPT};
              \end{axis}
          \end{tikzpicture}
    \end{subfigure}
    \hspace{1.5cm}
    \begin{subfigure}[!htb]{.4\textwidth}
        \centering
            \begin{tikzpicture}[trim axis right,trim axis left]
                \pgfplotsset{width=7cm, height=6.0cm}
                \begin{axis}[grid=major, xlabel={$R/X$}, ylabel={${I}^+_{im}$}, /pgf/number format/.cd, legend style={at={(0.98,0.15)},anchor=south east,legend columns=1, draw=none, inner sep=0pt,fill=gray!10}, xtick={-0,1,...,5}, axis line style = thick, ytick={-0.6,-0.5,...,0}, xmin=0, xmax=5]
                \addplot[very thick, black] table[x=x, y=y, meta=label, col sep=comma] {Data/RX/I1_im_LG.csv};
                \addplot[very thick, gray] table[x=x, y=y, meta=label, col sep=comma] {Data/RX/RI1_im_LG.csv};
                % \legend{OPT};
                \end{axis}
            \end{tikzpicture}
      \end{subfigure}

      \vspace{0.5cm}

      \begin{subfigure}[!htb]{.4\textwidth}
        \centering
            \begin{tikzpicture}[trim axis right,trim axis left]
                \pgfplotsset{width=7cm, height=6.0cm}
                \begin{axis}[grid=major, xlabel={$R/X$}, ylabel={${I}^-_{re}$}, /pgf/number format/.cd, legend style={at={(0.98,0.15)},anchor=south east,legend columns=1, draw=none, inner sep=0pt,fill=gray!10}, xtick={0,1,...,5}, ytick={0,0.1,...,0.5}, axis line style = thick, xmin=0, xmax=5]
                \addplot[very thick, black] table[x=x, y=y, meta=label, col sep=comma] {Data/RX/I2_re_LG.csv};
                \addplot[very thick, gray] table[x=x, y=y, meta=label, col sep=comma] {Data/RX/RI2_re_LG.csv};
                % \legend{BF, OPT};
                \end{axis}
            \end{tikzpicture}
      \end{subfigure}
      \hspace{1.5cm}
      \begin{subfigure}[!htb]{.4\textwidth}
          \centering
              \begin{tikzpicture}[trim axis right,trim axis left]
                  \pgfplotsset{width=7cm, height=6.0cm}
                  \begin{axis}[grid=major, xlabel={$R/X$}, ylabel={${I}^-_{im}$}, /pgf/number format/.cd, legend style={at={(0.98,0.15)},anchor=south east,legend columns=1, draw=none, inner sep=0pt,fill=gray!10}, xtick={0,1,...,5}, ytick={-0.4,-0.2,...,0.4}, axis line style =  thick, xmin=0, xmax=5]
                  \addplot[very thick, black] table[x=x, y=y, meta=label, col sep=comma] {Data/RX/I2_im_LG.csv};
                  \addplot[very thick, gray] table[x=x, y=y, meta=label, col sep=comma] {Data/RX/RI2_im_LG.csv};
                  % \legend{BF, OPT};
                  \end{axis}
              \end{tikzpicture}
        \end{subfigure}

        \vspace{0.2cm}

    \begin{center}
        \begin{subfigure}[!htb]{0.7\textwidth}
                \begin{tikzpicture}[trim axis right,trim axis left]
                    \pgfplotsset{width=12.5cm, height=6.0cm}
                    \begin{axis}[grid=major, xlabel={$R/X$}, ylabel={$f$}, /pgf/number format/.cd, legend style={at={(0.98,0.4)},anchor=south east,legend columns=1, draw=none, inner sep=0pt,fill=gray!10}, xtick={0,0.5,...,5}, ytick={0.56,0.58,...,0.68}, ymin = 0.55, ymax=0.68, axis line style = thick, yticklabel style={/pgf/number format/fixed, /pgf/number format/precision=3}, xmin=0, xmax=5]
                   \addplot[very thick, black] table[x=x, y=y, meta=label, col sep=comma] {Data/RX/ff_LG.csv};
                   \addplot[very thick, gray] table[x=x, y=y, meta=label, col sep=comma] {Data/RX/Rff_LG.csv};
                    \legend{OPT, ROPT};
                    \end{axis}
                \end{tikzpicture}
        \end{subfigure}
    \end{center}
    \caption{Influence of the currents on the objective function for the line to ground fault and a changing $R/X$ ratio and $\underline{Z}_f=0.001$. OPT: solution to the optimization problem, ROPT: solution to the optimization problem restricted to only injecting reactive power.}
    \label{fig:LGx1_c}
  \end{figure}
The line to ground fault has been analyzed for a fault impedance of 0.001 because otherwise the fault may not be severe enough. This is deduced from the presence of multiple optimal points for a same $R/X$ ratio while the objective function improves substantially due to the injection of currents. Instead, in this case the objective function does not vary much for all the $R/X$ range. Notice also that it is not far apart from the objective function in the balanced fault case. Consequently, we are able to conclude that the severity of the fault is similar thanks to the convenient adjustment of the fault impedance.

On the other hand, the differences between the OPT and the ROPT are relevant. Despite that, there is not much room for improvement in the sense that real currents, for both the positive and negative sequences, take ideally values close to 0.5 for big enough $R/X$ ratios. Imposing the constraint of not injecting any real current causes that all influence on the voltages is achieved by means of the imaginary currents, which are not enough to improve the voltages. For instance, for $R/X\approx 5$, the maximum absolute value of the $abc$ currents is one order of magnitude lower than the maximum allowed current $I_{max}$. Therefore, this suggests that no combination of currents is able to reduce more the objective function.

Even though we can expect that the positive sequence voltage is far from the unit value and the negative sequence is also distant from zero, the evaluation of voltages becomes worth of a particular study. Figure \ref{fig:full_RX_LG} shows the objective function along with the positive and negative sequence for the OPT and the ROPT cases.

\begin{figure}[!htb]\centering \footnotesize
\begin{tikzpicture}
\begin{axis}[%
    colormap name=whitered,
    width=12cm,
    height=10cm,
    view={45}{30},
    enlargelimits=false,
    grid=major,
    domain=-1:4,
    y domain=-1:4,
    samples=26,
    ztick={0.0,0.15,...,1},
    zmin=-0.05,
    zmax=1.0,
    xlabel=$R/X$,
    ylabel=$R/X$,
    zlabel={$|V_c|$},
    axis line style = thick,
    legend style={at={(1.06,0.5)},anchor=south west,legend columns=1, draw=none, inner sep=0pt,fill=gray!10},
    colorbar,
    colorbar style={
        at={(1.06,0.03)},
        anchor=south west,
        height=0.30*\pgfkeysvalueof{/pgfplots/parent axis height},
        title={$f$}
    }
]

\addplot3 [domain=-0:1,samples=31, samples y=0, very thick, smooth, densely dashed, black]  table[x=x, y=y, z=z, col sep=comma] {Data/RX/V1_LG.csv};
\addplot3 [domain=-0:1,samples=31, samples y=0, very thick, smooth, densely dotted, black] table[x=x, y=y, z=z, col sep=comma] {Data/RX/V2_LG.csv};
\addplot3 [domain=-0:1,samples=31, samples y=0, very thick, smooth, densely dashed, gray]  table[x=x, y=y, z=z, col sep=comma] {Data/RX/RV1_LG.csv};
\addplot3 [domain=-0:1,samples=31, samples y=0, very thick, smooth, densely dotted, gray] table[x=x, y=y, z=z, col sep=comma] {Data/RX/RV2_LG.csv};
\addplot3 [scatter, only marks, ycomb, each nth point = 2, gray!60] table[x=x, y=y, z=z, col sep=comma, forget plot] {Data/RX/ffG_LG.csv};
\addplot3 [scatter, only marks, ycomb, each nth point = 2, gray!60] table[x=x, y=y, z=z, col sep=comma, forget plot] {Data/RX/RffG_LG.csv};
\addplot3 [gray, no markers, line width=1pt] table[x=x, y=y, z=z, col sep=comma, forget plot] {Data/RX/terra_RX.csv};

% \node at (0.5,0.1,0.3) [pin=165:$P(x_1)$] {};
% \node at (0.5,0.1,0.2) [pin=85:$P(x_2)$] {};
% \node at (0.5,0.5,0.1) [pin=165:$P(x_3)$] {};

\legend{$V^+_c$ OPT, $V^-_c$ OPT, $V^+_c$ ROPT, $V^-_c$ ROPT};

\end{axis}
\end{tikzpicture}
\caption{Sequence voltages together with the objective function for the line to ground fault with $\underline{Z}_f=0.001$ and a varying $R/X$ ratio}
\label{fig:full_RX_LG}
\end{figure}
When looking at the bigger picture the objective function remains almost always the same, yet there exists a permanent difference between the OPT and the ROPT cases. The positive sequence voltages in the OPT situation are always above the ROPT ones for about 0.04 pu. For the negative sequence voltage the pattern is reversed. It is relevant to take into account that even if the voltages turn out to be practically constant, the currents experience large variations, as shown in Figure \ref{fig:LGx1_c}. Extracting conclusions regarding the fault by only observing the voltages may be misleading, as they can be maintained at the expense of injecting the specific optimal currents. Besides, just like it happened with the balanced fault, the shape of the objective function ressembles the shape of the voltages. This is logical when considering the proportionality between the objective function and the voltages.


\subsection{Line to line fault}
Figure \ref{fig:LLx1_c} depicts the optimal currents for the line to line fault.

\begin{figure}[!htb]\centering \footnotesize
    \begin{subfigure}[!htb]{.4\textwidth}
      \centering
          \begin{tikzpicture}[trim axis right,trim axis left]
              \pgfplotsset{width=7cm, height=6.0cm}
              \begin{axis}[grid=major, xlabel={$R/X$}, ylabel={${I}^+_{re}$}, /pgf/number format/.cd, legend style={at={(0.98,0.15)},anchor=south east,legend columns=1, draw=none, inner sep=0pt,fill=gray!10}, xtick={0,1,...,5}, axis line style = thick, ytick={0,0.05,...,0.3}, scaled y ticks=false, yticklabel style={/pgf/number format/fixed, /pgf/number format/precision=2}]
              \addplot[very thick, black] table[x=x, y=y, meta=label, col sep=comma] {Data/RX/I1_re_LL.csv};
              \addplot[very thick, gray] table[x=x, y=y, meta=label, col sep=comma] {Data/RX/RI1_re_LL.csv};
              % \legend{BF, OPT};
              \end{axis}
          \end{tikzpicture}
    \end{subfigure}
    \hspace{1.5cm}
    \begin{subfigure}[!htb]{.4\textwidth}
        \centering
            \begin{tikzpicture}[trim axis right,trim axis left]
                \pgfplotsset{width=7cm, height=6.0cm}
                \begin{axis}[grid=major, xlabel={$R/X$}, ylabel={${I}^+_{im}$}, /pgf/number format/.cd, legend style={at={(0.98,0.15)},anchor=south east,legend columns=1, draw=none, inner sep=0pt,fill=gray!10}, xtick={0,1,...,5}, ytick={-1,-0.75,...,0}, axis line style = thick]
                \addplot[very thick, black] table[x=x, y=y, meta=label, col sep=comma] {Data/RX/I1_im_LL.csv};
                \addplot[very thick, gray] table[x=x, y=y, meta=label, col sep=comma] {Data/RX/RI1_im_LL.csv};
                % \legend{OPT};
                \end{axis}
            \end{tikzpicture}
      \end{subfigure}

      \vspace{0.5cm}

      \begin{subfigure}[!htb]{.4\textwidth}
        \centering
            \begin{tikzpicture}[trim axis right,trim axis left]
                \pgfplotsset{width=7cm, height=6.0cm}
                \begin{axis}[grid=major, xlabel={$R/X$}, ylabel={${I}^-_{re}$}, /pgf/number format/.cd, legend style={at={(0.98,0.15)},anchor=south east,legend columns=1, draw=none, inner sep=0pt,fill=gray!10}, xtick={0,1,...,5}, ytick={-1,-0.75,...,0}, axis line style = thick]
                \addplot[very thick, black] table[x=x, y=y, meta=label, col sep=comma] {Data/RX/I2_re_LL.csv};
                \addplot[very thick, gray] table[x=x, y=y, meta=label, col sep=comma] {Data/RX/RI2_re_LL.csv};
                % \legend{BF, OPT};
                \end{axis}
            \end{tikzpicture}
      \end{subfigure}
      \hspace{1.5cm}
      \begin{subfigure}[!htb]{.4\textwidth}
          \centering
              \begin{tikzpicture}[trim axis right,trim axis left]
                  \pgfplotsset{width=7cm, height=6.0cm}
                  \begin{axis}[grid=major, xlabel={$R/X$}, ylabel={${I}^-_{im}$}, /pgf/number format/.cd, legend style={at={(0.98,0.15)},anchor=south east,legend columns=1, draw=none, inner sep=0pt,fill=gray!10}, xtick={0,1,...,5}, ytick={0.1,0.2,...,0.6}, axis line style = thick]
                  \addplot[very thick, black] table[x=x, y=y, meta=label, col sep=comma] {Data/RX/I2_im_LL.csv};
                  \addplot[very thick, gray] table[x=x, y=y, meta=label, col sep=comma] {Data/RX/RI2_im_LL.csv};
                  % \legend{BF, OPT};
                  \end{axis}
              \end{tikzpicture}
        \end{subfigure}

        \vspace{0.2cm}

    \begin{center}
        \begin{subfigure}[!htb]{0.7\textwidth}
                \begin{tikzpicture}[trim axis right,trim axis left]
                    \pgfplotsset{width=12.5cm, height=6cm}
                    \begin{axis}[grid=major, xlabel={$R/X$}, ylabel={$f$}, /pgf/number format/.cd, legend style={at={(0.98,0.8)},anchor=south east,legend columns=1, draw=none, inner sep=0pt,fill=gray!10}, xtick={0,0.5,...,5}, ytick={0.38,0.39,...,0.45}, axis line style = thick, yticklabel style={/pgf/number format/fixed, /pgf/number format/precision=2}]
                   \addplot[very thick, black] table[x=x, y=y, meta=label, col sep=comma] {Data/RX/ff_LL.csv};
                   \addplot[very thick, gray] table[x=x, y=y, meta=label, col sep=comma] {Data/RX/Rff_LL.csv};
                    \legend{OPT, ROPT};
                    \end{axis}
                \end{tikzpicture}
        \end{subfigure}
    \end{center}
    \caption{Influence of the currents on the objective function for the line to line fault and a changing $R/X$ ratio. OPT: solution to the optimization problem, ROPT: solution to the optimization problem restricted to only injecting reactive power.}
    \label{fig:LLx1_c}
  \end{figure}

\subsection{Double line to ground fault}
Figure \ref{fig:LLGx1_c} depicts the optimal currents for the balanced fault.

\begin{figure}[!htb]\centering \footnotesize
    \begin{subfigure}[!htb]{.4\textwidth}
      \centering
          \begin{tikzpicture}[trim axis right,trim axis left]
              \pgfplotsset{width=7cm, height=6.0cm}
              \begin{axis}[grid=major, xlabel={$R/X$}, ylabel={${I}^+_{re}$}, /pgf/number format/.cd, legend style={at={(0.98,0.15)},anchor=south east,legend columns=1, draw=none, inner sep=0pt,fill=gray!10}, xtick={0,1,...,5}, ytick={0,0.01,...,0.05}, axis line style = very thick, yticklabel style={/pgf/number format/fixed, /pgf/number format/precision=5}, scaled y ticks=false]
              \addplot[very thick, black] table[x=x, y=y, meta=label, col sep=comma] {Data/RX/I1_re_LLG.csv};
              \addplot[very thick, gray] table[x=x, y=y, meta=label, col sep=comma] {Data/RX/RI1_re_LLG.csv};
              % \legend{BF, OPT};
              \end{axis}
          \end{tikzpicture}
    \end{subfigure}
    \hspace{1.5cm}
    \begin{subfigure}[!htb]{.4\textwidth}
        \centering
            \begin{tikzpicture}[trim axis right,trim axis left]
                \pgfplotsset{width=7cm, height=6.0cm}
                \begin{axis}[grid=major, xlabel={$R/X$}, ylabel={${I}^+_{im}$}, /pgf/number format/.cd, legend style={at={(0.98,0.15)},anchor=south east,legend columns=1, draw=none, inner sep=0pt,fill=gray!10}, xtick={0,1,...,5}, ytick={-0.5775,-0.5770,...,-0.5750}, yticklabel style={/pgf/number format/fixed, /pgf/number format/precision=5}, axis line style = very thick]
                \addplot[very thick, black] table[x=x, y=y, meta=label, col sep=comma] {Data/RX/I1_im_LLG.csv};
                \addplot[very thick, gray] table[x=x, y=y, meta=label, col sep=comma] {Data/RX/RI1_im_LLG.csv};
                % \legend{OPT};
                \end{axis}
            \end{tikzpicture}
      \end{subfigure}

      \vspace{0.5cm}

      \begin{subfigure}[!htb]{.4\textwidth}
        \centering
            \begin{tikzpicture}[trim axis right,trim axis left]
                \pgfplotsset{width=7cm, height=6.0cm}
                \begin{axis}[grid=major, xlabel={$R/X$}, ylabel={${I}^-_{re}$}, /pgf/number format/.cd, legend style={at={(0.98,0.15)},anchor=south east,legend columns=1, draw=none, inner sep=0pt,fill=gray!10}, xtick={0,1,...,5}, ytick={-0.05,-0.04,...,0}, axis line style = very thick, yticklabel style={/pgf/number format/fixed, /pgf/number format/precision=5}, scaled y ticks=false]
                \addplot[very thick, black] table[x=x, y=y, meta=label, col sep=comma] {Data/RX/I2_re_LLG.csv};
                \addplot[very thick, gray] table[x=x, y=y, meta=label, col sep=comma] {Data/RX/RI2_re_LLG.csv};
                % \legend{BF, OPT};
                \end{axis}
            \end{tikzpicture}
      \end{subfigure}
      \hspace{1.5cm}
      \begin{subfigure}[!htb]{.4\textwidth}
          \centering
              \begin{tikzpicture}[trim axis right,trim axis left]
                  \pgfplotsset{width=7cm, height=6.0cm}
                  \begin{axis}[grid=major, xlabel={$R/X$}, ylabel={${I}^-_{im}$}, /pgf/number format/.cd, legend style={at={(0.98,0.15)},anchor=south east,legend columns=1, draw=none, inner sep=0pt,fill=gray!10}, xtick={0,1,...,5}, ytick={0.5750,0.5755,...,0.5780}, axis line style = very thick, yticklabel style={/pgf/number format/fixed, /pgf/number format/precision=4}]
                  \addplot[very thick, black] table[x=x, y=y, meta=label, col sep=comma] {Data/RX/I2_im_LLG.csv};
                  \addplot[very thick, gray] table[x=x, y=y, meta=label, col sep=comma] {Data/RX/RI2_im_LLG.csv};
                  % \legend{BF, OPT};
                  \end{axis}
              \end{tikzpicture}
        \end{subfigure}

        \vspace{0.2cm}

    \begin{center}
        \begin{subfigure}[!htb]{0.7\textwidth}
                \begin{tikzpicture}[trim axis right,trim axis left]
                    \pgfplotsset{width=12.5cm, height=6cm}
                    \begin{axis}[grid=major, xlabel={$R/X$}, ylabel={$f$}, /pgf/number format/.cd, legend style={at={(0.98,0.5)},anchor=south east,legend columns=1, draw=none, inner sep=0pt,fill=gray!10}, xtick={0,0.5,...,5}, yticklabel style={/pgf/number format/fixed, /pgf/number format/precision=5}, ytick={0.8840,0.8841,...,0.8844}, axis line style = very thick]
                   \addplot[very thick, black] table[x=x, y=y, meta=label, col sep=comma] {Data/RX/ff_LLG.csv};
                   \addplot[very thick, gray] table[x=x, y=y, meta=label, col sep=comma] {Data/RX/Rff_LLG.csv};
                    \legend{OPT, ROPT};
                    \end{axis}
                \end{tikzpicture}
        \end{subfigure}
    \end{center}
    \caption{Influence of the currents on the objective function for the double line to ground fault and a changing $R/X$ ratio. OPT: solution to the optimization problem, ROPT: solution to the optimization problem restricted to only injecting reactive power.}
    \label{fig:LLGx1_c}
  \end{figure}
About the discussion of the plots, for the balanced fault it becomes clear that injecting a large imaginary positive sequence current (in absolute value) is the preferred option to minimize the objective function. In the OPT case some of this current is traded for some real positive sequence current that allows the objective function to become slightly lower when compared to the ROPT case. Since the negative sequence equivalent circuit is decoupled from the positive sequence one, the negative sequence currents are null in all cases. Besides, the larger $R$ becomes with respect to $X$, the larger the real positive current has to be as well. This phenomena makes complete sense when considering that we want to amplify the voltage drop in the positive sequence. However, even when $R$ is about five times larger than $X$, the imaginary positive sequence current is dominant.

As explained in the initial discussion where no sweep was taking place, the line to ground case may have multiple minimums. The plots in Figure \ref{fig:LGx1_c} indicate that this could also be the case here, as the evolution of currents does not follow a smooth trajectory. Rather, it is not continuously differentiable across all its path. This phenomena happens for all currents at the same $R/X$ values but is not noticeable from the objective function plot. Such continuity on the objective function could be a great indicator of the presence of various equal minimums for a single $R/X$ value.

The line to line fault ends up presenting a larger $f$ value than in the two abovementioned cases. That is, it becomes a more severe fault. Note from the objective function that the OPT is superior to the ROPT case for all values of the $R/X$ ratio. However, the distribution of currents hugely varies for small $R/X$ ratios. When $R$ tends to be way smaller than $X$, the negative sequence current is prioritized in the OPT case. When $R$ becomes larger, this diminishes and then the imaginary positive and negative sequence currents are the major ones. Contrarily to the other faults, the bigger $R$, the better. 

In the double line to ground fault the differences across all the sweep values is minimal. The objective functions remain at almost always the same values, and even though the OPT is again the winner, the difference with the ROPT situation does not become substantial. The imaginary positive and negative sequence currents tend to take the same values but with the opposite sign, whereas the real positive sequence currents is merely the reflection of the real negative sequence current. We can deduce that since the double line to ground fault involves a solid connection between phases, it makes this turns out to be the most severe situation.
\newpage
% \section{Submarine cable}
We now deal with a submarine cable, which is modelled with a $\pi$ equivalent, where the series impedance is still $\underline{Z}_a$ while the parallel impedance is $\underline{Z}_c=-jX_c$. As follows, we sweep across $X_c$.

\subsection{Balanced fault}
Figure \ref{fig:3x1_s} depicts the optimal currents for the balanced fault.

\begin{figure}[!htb]\centering \footnotesize
    \begin{subfigure}[!htb]{.4\textwidth}
      \centering
          \begin{tikzpicture}[trim axis right,trim axis left]
              \pgfplotsset{width=7cm, height=5.5cm}
              \begin{axis}[grid=major, xlabel={$X_c$}, ylabel={${I}^+_{re}$}, /pgf/number format/.cd, legend style={at={(0.98,0.15)},anchor=south east,legend columns=1, draw=none, inner sep=0pt,fill=gray!10}, xtick={0,20,...,100}, ytick={0.0,0.05,...,0.25}, axis line style = very thick, yticklabel style={/pgf/number format/fixed, /pgf/number format/precision=2}]
              \addplot[very thick, black] table[x=x, y=y, meta=label, col sep=comma] {Data/submarine/I1_re_3x.csv};
              \addplot[very thick, gray] table[x=x, y=y, meta=label, col sep=comma] {Data/submarine/RI1_re_3x.csv};
              % \legend{BF, OPT};
              \end{axis}
          \end{tikzpicture}
    \end{subfigure}
    \hspace{1.5cm}
    \begin{subfigure}[!htb]{.4\textwidth}
        \centering
            \begin{tikzpicture}[trim axis right,trim axis left]
                \pgfplotsset{width=7cm, height=5.5cm}
                \begin{axis}[grid=major, xlabel={$X_c$}, ylabel={${I}^+_{im}$}, /pgf/number format/.cd, legend style={at={(0.98,0.15)},anchor=south east,legend columns=1, draw=none, inner sep=0pt,fill=gray!10}, xtick={0,20,...,100}, ytick={-1,-0.995,...,-0.975}, axis line style = very thick, yticklabel style={/pgf/number format/fixed, /pgf/number format/precision=3}]
                \addplot[very thick, black] table[x=x, y=y, meta=label, col sep=comma] {Data/submarine/I1_im_3x.csv};
                \addplot[very thick, gray] table[x=x, y=y, meta=label, col sep=comma] {Data/submarine/RI1_im_3x.csv};
                % \legend{OPT};
                \end{axis}
            \end{tikzpicture}
      \end{subfigure}

      \vspace{0.5cm}

      \begin{subfigure}[!htb]{.4\textwidth}
        \centering
            \begin{tikzpicture}[trim axis right,trim axis left]
                \pgfplotsset{width=7cm, height=5.5cm}
                \begin{axis}[grid=major, xlabel={$X_c$}, ylabel={${I}^-_{re}$}, /pgf/number format/.cd, legend style={at={(0.98,0.15)},anchor=south east,legend columns=1, draw=none, inner sep=0pt,fill=gray!10}, xtick={0,20,...,100}, ymin = -0.2, ymax=0.2, axis line style = very thick]
                \addplot[very thick, black] table[x=x, y=y, meta=label, col sep=comma] {Data/submarine/I2_re_3x.csv};
                \addplot[very thick, gray] table[x=x, y=y, meta=label, col sep=comma] {Data/submarine/RI2_re_3x.csv};
                % \legend{BF, OPT};
                \end{axis}
            \end{tikzpicture}
      \end{subfigure}
      \hspace{1.5cm}
      \begin{subfigure}[!htb]{.4\textwidth}
          \centering
              \begin{tikzpicture}[trim axis right,trim axis left]
                  \pgfplotsset{width=7cm, height=5.5cm}
                  \begin{axis}[grid=major, xlabel={$X_c$}, ylabel={${I}^-_{im}$}, /pgf/number format/.cd, legend style={at={(0.98,0.15)},anchor=south east,legend columns=1, draw=none, inner sep=0pt,fill=gray!10}, xtick={0,20,...,100}, ymin = -0.2, ymax=0.2, axis line style = very thick]
                  \addplot[very thick, black] table[x=x, y=y, meta=label, col sep=comma] {Data/submarine/I2_im_3x.csv};
                  \addplot[very thick, gray] table[x=x, y=y, meta=label, col sep=comma] {Data/submarine/RI2_im_3x.csv};
                  % \legend{BF, OPT};
                  \end{axis}
              \end{tikzpicture}
        \end{subfigure}

        \vspace{0.2cm}

    \begin{center}
        \begin{subfigure}[!htb]{0.7\textwidth}
                \begin{tikzpicture}[trim axis right,trim axis left]
                    \pgfplotsset{width=12.5cm, height=5.5cm}
                    \begin{axis}[grid=major, xlabel={$X_c$}, ylabel={$f$}, /pgf/number format/.cd, legend style={at={(0.98,0.2)},anchor=south east,legend columns=1, draw=none, inner sep=0pt,fill=gray!10}, xtick={0,10,...,100}, axis line style = very thick, yticklabel style={/pgf/number format/fixed, /pgf/number format/precision=3}]
                   \addplot[very thick, black] table[x=x, y=y, meta=label, col sep=comma] {Data/submarine/ff_3x.csv};
                   \addplot[very thick, gray] table[x=x, y=y, meta=label, col sep=comma] {Data/submarine/Rff_3x.csv};
                    \legend{OPT, ROPT};
                    \end{axis}
                \end{tikzpicture}
        \end{subfigure}
    \end{center}
    \caption{Influence of the currents on the objective function for the balanced fault and a submarine cable. OPT: solution to the optimization problem, ROPT: solution to the optimization problem restricted to only injecting reactive power.}
    \label{fig:3x1_s}
  \end{figure}

\subsection{Line to ground fault}
Figure \ref{fig:LGx1_s} depicts the optimal currents for the line to ground fault.
\begin{figure}[!htb]\centering \footnotesize
    \begin{subfigure}[!htb]{.4\textwidth}
      \centering
          \begin{tikzpicture}[trim axis right,trim axis left]
              \pgfplotsset{width=7cm, height=5.9cm}
              \begin{axis}[grid=major, xlabel={$R/X$}, ylabel={${I}^+_{re}$}, /pgf/number format/.cd, legend style={at={(0.98,0.15)},anchor=south east,legend columns=1, draw=none, inner sep=0pt,fill=gray!10}, xtick={0,20,...,100}, ytick={0,0.02,...,0.1}, axis line style = very thick, yticklabel style={/pgf/number format/fixed, /pgf/number format/precision=2}]
              \addplot[very thick, black] table[x=x, y=y, meta=label, col sep=comma] {Data/submarine/I1_re_LG.csv};
              \addplot[very thick, gray] table[x=x, y=y, meta=label, col sep=comma] {Data/submarine/RI1_re_LG.csv};
              % \legend{BF, OPT};
              \end{axis}
          \end{tikzpicture}
    \end{subfigure}
    \hspace{1.5cm}
    \begin{subfigure}[!htb]{.4\textwidth}
        \centering
            \begin{tikzpicture}[trim axis right,trim axis left]
                \pgfplotsset{width=7cm, height=5.9cm}
                \begin{axis}[grid=major, xlabel={$R/X$}, ylabel={${I}^+_{im}$}, /pgf/number format/.cd, legend style={at={(0.98,0.15)},anchor=south east,legend columns=1, draw=none, inner sep=0pt,fill=gray!10}, xtick={0,20,...,100}, ytick={-0.8,-0.7,...,-0.3}, axis line style = very thick]
                \addplot[very thick, black] table[x=x, y=y, meta=label, col sep=comma] {Data/submarine/I1_im_LG.csv};
                \addplot[very thick, gray] table[x=x, y=y, meta=label, col sep=comma] {Data/submarine/RI1_im_LG.csv};
                % \legend{OPT};
                \end{axis}
            \end{tikzpicture}
      \end{subfigure}

      \vspace{0.5cm}

      \begin{subfigure}[!htb]{.4\textwidth}
        \centering
            \begin{tikzpicture}[trim axis right,trim axis left]
                \pgfplotsset{width=7cm, height=5.9cm}
                \begin{axis}[grid=major, xlabel={$R/X$}, ylabel={${I}^-_{re}$}, /pgf/number format/.cd, legend style={at={(0.98,0.15)},anchor=south east,legend columns=1, draw=none, inner sep=0pt,fill=gray!10}, xtick={0,20,...,100}, axis line style = very thick, scaled y ticks=false, yticklabel style={/pgf/number format/fixed,/pgf/number format/precision=2}]
                \addplot[very thick, black] table[x=x, y=y, meta=label, col sep=comma] {Data/submarine/I2_re_LG.csv};
                \addplot[very thick, gray] table[x=x, y=y, meta=label, col sep=comma] {Data/submarine/RI2_re_LG.csv};
                % \legend{BF, OPT};
                \end{axis}
            \end{tikzpicture}
      \end{subfigure}
      \hspace{1.5cm}
      \begin{subfigure}[!htb]{.4\textwidth}
          \centering
              \begin{tikzpicture}[trim axis right,trim axis left]
                  \pgfplotsset{width=7cm, height=5.9cm}
                  \begin{axis}[grid=major, xlabel={$R/X$}, ylabel={${I}^-_{im}$}, /pgf/number format/.cd, legend style={at={(0.98,0.15)},anchor=south east,legend columns=1, draw=none, inner sep=0pt,fill=gray!10}, xtick={0,20,...,100}, ytick={-0.7,-0.6,...,-0.2}, axis line style = very thick]
                  \addplot[very thick, black] table[x=x, y=y, meta=label, col sep=comma] {Data/submarine/I2_im_LG.csv};
                  \addplot[very thick, gray] table[x=x, y=y, meta=label, col sep=comma] {Data/submarine/RI2_im_LG.csv};
                  % \legend{BF, OPT};
                  \end{axis}
              \end{tikzpicture}
        \end{subfigure}

        \vspace{0.2cm}

    \begin{center}
        \begin{subfigure}[!htb]{0.7\textwidth}
                \begin{tikzpicture}[trim axis right,trim axis left]
                    \pgfplotsset{width=12.5cm, height=5.9cm}
                    \begin{axis}[grid=major, xlabel={$R/X$}, ylabel={$f$}, /pgf/number format/.cd, legend style={at={(0.98,0.2)},anchor=south east,legend columns=1, draw=none, inner sep=0pt,fill=gray!10}, xtick={0,10,...,100}, ytick={0.13,0.14,...,0.17}, axis line style = very thick, yticklabel style={/pgf/number format/fixed, /pgf/number format/precision=3}]
                   \addplot[very thick, black] table[x=x, y=y, meta=label, col sep=comma] {Data/submarine/ff_LG.csv};
                   \addplot[very thick, gray] table[x=x, y=y, meta=label, col sep=comma] {Data/submarine/Rff_LG.csv};
                    \legend{OPT, ROPT};
                    \end{axis}
                \end{tikzpicture}
        \end{subfigure}
    \end{center}
    \caption{Influence of the currents on the objective function for the line to ground fault and a submarine cable. OPT: solution to the optimization problem, ROPT: solution to the optimization problem restricted to only injecting reactive power.}
    \label{fig:LGx1_s}
  \end{figure}

\subsection{Line to line fault}
Figure \ref{fig:LLx1_s} depicts the optimal currents for the line to line fault.

\begin{figure}[!htb]\centering \footnotesize
    \begin{subfigure}[!htb]{.4\textwidth}
      \centering
          \begin{tikzpicture}[trim axis right,trim axis left]
              \pgfplotsset{width=7cm, height=6.0cm}
              \begin{axis}[grid=major, xlabel={$X_c$}, ylabel={${I}^+_{re}$}, /pgf/number format/.cd, legend style={at={(0.98,0.15)},anchor=south east,legend columns=1, draw=none, inner sep=0pt,fill=gray!10}, xtick={0,20,...,100}, ymax = 0.2, ymin=-0.2, axis line style = very thick, scaled y ticks=false, yticklabel style={/pgf/number format/fixed, /pgf/number format/precision=2}]
              \addplot[very thick, black] table[x=x, y=y, meta=label, col sep=comma] {Data/submarine/I1_re_LL.csv};
              \addplot[very thick, gray] table[x=x, y=y, meta=label, col sep=comma] {Data/submarine/RI1_re_LL.csv};
              % \legend{BF, OPT};
              \end{axis}
          \end{tikzpicture}
    \end{subfigure}
    \hspace{1.5cm}
    \begin{subfigure}[!htb]{.4\textwidth}
        \centering
            \begin{tikzpicture}[trim axis right,trim axis left]
                \pgfplotsset{width=7cm, height=6.0cm}
                \begin{axis}[grid=major, xlabel={$X_c$}, ylabel={${I}^+_{im}$}, /pgf/number format/.cd, legend style={at={(0.98,0.15)},anchor=south east,legend columns=1, draw=none, inner sep=0pt,fill=gray!10}, xtick={0,20,...,100}, ytick={-1,-0.75,...,0}, axis line style = very thick]
                \addplot[very thick, black] table[x=x, y=y, meta=label, col sep=comma] {Data/submarine/I1_im_LL.csv};
                \addplot[very thick, gray] table[x=x, y=y, meta=label, col sep=comma] {Data/submarine/RI1_im_LL.csv};
                % \legend{OPT};
                \end{axis}
            \end{tikzpicture}
      \end{subfigure}

      \vspace{0.5cm}

      \begin{subfigure}[!htb]{.4\textwidth}
        \centering
            \begin{tikzpicture}[trim axis right,trim axis left]
                \pgfplotsset{width=7cm, height=6.0cm}
                \begin{axis}[grid=major, xlabel={$X_c$}, ylabel={${I}^-_{re}$}, /pgf/number format/.cd, legend style={at={(0.98,0.15)},anchor=south east,legend columns=1, draw=none, inner sep=0pt,fill=gray!10}, xtick={0,20,...,100}, ytick={-1,-0.75,...,0}, axis line style = very thick]
                \addplot[very thick, black] table[x=x, y=y, meta=label, col sep=comma] {Data/submarine/I2_re_LL.csv};
                \addplot[very thick, gray] table[x=x, y=y, meta=label, col sep=comma] {Data/submarine/RI2_re_LL.csv};
                % \legend{BF, OPT};
                \end{axis}
            \end{tikzpicture}
      \end{subfigure}
      \hspace{1.5cm}
      \begin{subfigure}[!htb]{.4\textwidth}
          \centering
              \begin{tikzpicture}[trim axis right,trim axis left]
                  \pgfplotsset{width=7cm, height=6.0cm}
                  \begin{axis}[grid=major, xlabel={$X_c$}, ylabel={${I}^-_{im}$}, /pgf/number format/.cd, legend style={at={(0.98,0.15)},anchor=south east,legend columns=1, draw=none, inner sep=0pt,fill=gray!10}, xtick={0,20,...,100}, axis line style = very thick, yticklabel style={/pgf/number format/fixed, /pgf/number format/precision=2}]
                  \addplot[very thick, black] table[x=x, y=y, meta=label, col sep=comma] {Data/submarine/I2_im_LL.csv};
                  \addplot[very thick, gray] table[x=x, y=y, meta=label, col sep=comma] {Data/submarine/RI2_im_LL.csv};
                  % \legend{BF, OPT};
                  \end{axis}
              \end{tikzpicture}
        \end{subfigure}

        \vspace{0.2cm}

    \begin{center}
        \begin{subfigure}[!htb]{0.7\textwidth}
                \begin{tikzpicture}[trim axis right,trim axis left]
                    \pgfplotsset{width=12.5cm, height=6cm}
                    \begin{axis}[grid=major, xlabel={$X_c$}, ylabel={$f$}, /pgf/number format/.cd, legend style={at={(0.98,0.5)},anchor=south east,legend columns=1, draw=none, inner sep=0pt,fill=gray!10}, xtick={0,10,...,100}, axis line style = very thick, yticklabel style={/pgf/number format/fixed, /pgf/number format/precision=2}]
                   \addplot[very thick, black] table[x=x, y=y, meta=label, col sep=comma] {Data/submarine/ff_LL.csv};
                   \addplot[very thick, gray] table[x=x, y=y, meta=label, col sep=comma] {Data/submarine/Rff_LL.csv};
                    \legend{OPT, ROPT};
                    \end{axis}
                \end{tikzpicture}
        \end{subfigure}
    \end{center}
    \caption{Influence of the currents on the objective function for the line to line fault and a submarine cable. OPT: solution to the optimization problem, ROPT: solution to the optimization problem restricted to only injecting reactive power.}
    \label{fig:LLx1_s}
  \end{figure}

\subsection{Double line to ground fault}
Figure \ref{fig:LLGx1_s} depicts the optimal currents for the balanced fault.

\begin{figure}[!htb]\centering \footnotesize
    \begin{subfigure}[!htb]{.4\textwidth}
      \centering
          \begin{tikzpicture}[trim axis right,trim axis left]
              \pgfplotsset{width=7cm, height=6.0cm}
              \begin{axis}[grid=major, xlabel={$X_c$}, ylabel={${I}^+_{re}$}, /pgf/number format/.cd, legend style={at={(0.98,0.15)},anchor=south east,legend columns=1, draw=none, inner sep=0pt,fill=gray!10},  xtick={0,20,...,100}, ytick={-0.01,-0.0075,...,0}, axis line style = very thick, yticklabel style={/pgf/number format/fixed, /pgf/number format/precision=4}, scaled y ticks=false]
              \addplot[very thick, black] table[x=x, y=y, meta=label, col sep=comma] {Data/submarine/I1_re_LLG.csv};
              \addplot[very thick, gray] table[x=x, y=y, meta=label, col sep=comma] {Data/submarine/RI1_re_LLG.csv};
              % \legend{BF, OPT};
              \end{axis}
          \end{tikzpicture}
    \end{subfigure}
    \hspace{1.5cm}
    \begin{subfigure}[!htb]{.4\textwidth}
        \centering
            \begin{tikzpicture}[trim axis right,trim axis left]
                \pgfplotsset{width=7cm, height=6.0cm}
                \begin{axis}[grid=major, xlabel={$X_c$}, ylabel={${I}^+_{im}$}, /pgf/number format/.cd, legend style={at={(0.98,0.15)},anchor=south east,legend columns=1, draw=none, inner sep=0pt,fill=gray!10}, xtick={0,20,...,100}, ytick={-1,-0.75,...,0}, yticklabel style={/pgf/number format/fixed, /pgf/number format/precision=5}, axis line style = very thick]
                \addplot[very thick, black] table[x=x, y=y, meta=label, col sep=comma] {Data/submarine/I1_im_LLG.csv};
                \addplot[very thick, gray] table[x=x, y=y, meta=label, col sep=comma] {Data/submarine/RI1_im_LLG.csv};
                % \legend{OPT};
                \end{axis}
            \end{tikzpicture}
      \end{subfigure}

      \vspace{0.5cm}

      \begin{subfigure}[!htb]{.4\textwidth}
        \centering
            \begin{tikzpicture}[trim axis right,trim axis left]
                \pgfplotsset{width=7cm, height=6.0cm}
                \begin{axis}[grid=major, xlabel={$X_c$}, ylabel={${I}^-_{re}$}, /pgf/number format/.cd, legend style={at={(0.98,0.15)},anchor=south east,legend columns=1, draw=none, inner sep=0pt,fill=gray!10}, xtick={0,20,...,100}, ytick={-1,-0.75,...,0}, axis line style = very thick, yticklabel style={/pgf/number format/fixed, /pgf/number format/precision=5}, scaled y ticks=false]
                \addplot[very thick, black] table[x=x, y=y, meta=label, col sep=comma] {Data/submarine/I2_re_LLG.csv};
                \addplot[very thick, gray] table[x=x, y=y, meta=label, col sep=comma] {Data/submarine/RI2_re_LLG.csv};
                % \legend{BF, OPT};
                \end{axis}
            \end{tikzpicture}
      \end{subfigure}
      \hspace{1.5cm}
      \begin{subfigure}[!htb]{.4\textwidth}
          \centering
              \begin{tikzpicture}[trim axis right,trim axis left]
                  \pgfplotsset{width=7cm, height=6.0cm}
                  \begin{axis}[grid=major, xlabel={$X_c$}, ylabel={${I}^-_{im}$}, /pgf/number format/.cd, legend style={at={(0.98,0.15)},anchor=south east,legend columns=1, draw=none, inner sep=0pt,fill=gray!10}, xtick={0,20,...,100}, ytick={0.01,0.04,...,0.16}, axis line style = very thick, yticklabel style={/pgf/number format/fixed, /pgf/number format/precision=2}, scaled y ticks=false]
                  \addplot[very thick, black] table[x=x, y=y, meta=label, col sep=comma] {Data/submarine/I2_im_LLG.csv};
                  \addplot[very thick, gray] table[x=x, y=y, meta=label, col sep=comma] {Data/submarine/RI2_im_LLG.csv};
                  % \legend{BF, OPT};
                  \end{axis}
              \end{tikzpicture}
        \end{subfigure}

        \vspace{0.2cm}

    \begin{center}
        \begin{subfigure}[!htb]{0.7\textwidth}
                \begin{tikzpicture}[trim axis right,trim axis left]
                    \pgfplotsset{width=12.5cm, height=6cm}
                    \begin{axis}[grid=major, xlabel={$X_c$}, ylabel={$f$}, /pgf/number format/.cd, legend style={at={(0.98,0.2)},anchor=south east,legend columns=1, draw=none, inner sep=0pt,fill=gray!10}, xtick={0,10,...,100}, ytick={0.2,0.21,...,0.24}, yticklabel style={/pgf/number format/fixed, /pgf/number format/precision=5}, axis line style = very thick]
                   \addplot[very thick, black] table[x=x, y=y, meta=label, col sep=comma] {Data/submarine/ff_LLG.csv};
                   \addplot[very thick, gray] table[x=x, y=y, meta=label, col sep=comma] {Data/submarine/Rff_LLG.csv};
                    \legend{OPT, ROPT};
                    \end{axis}
                \end{tikzpicture}
        \end{subfigure}
    \end{center}
    \caption{Influence of the currents on the objective function for the double line to ground fault and a submarine cable. OPT: solution to the optimization problem, ROPT: solution to the optimization problem restricted to only injecting reactive power.}
    \label{fig:LLGx1_s}
  \end{figure}
The plots for the submarine cable indicate that while the distribution of currents changes substantially when considering the restriction in the active current, the objective functions tend to take similar values. 

For instance, in the balanced fault, the best strategy in the ROPT case is to inject a maximum imaginary positive sequence current. On the contrary, the imaginary positive sequence current does not reach the limits of one. Instead, some part of the current is dedicated to the real part. As in the $R/X$ case of study, the negative sequence currents remain null for the full sweep. One can check that the objective function for a small $R/X$ ratio seems to coincide with the function when $X_c$ takes a large value. Again, the OPT scenario is slightly better than the ROPT one. 

In the line to ground fault the objective functions is not far apart from the results from Figure \ref{fig:LGx1_c}. However, this time we have not obtained a discontinuous profile in the evolution of the currents. This shows that maybe in this case there is only a single minimum, or also, that the optimal values follow the same trajectory due to the initialization. Oddly enough, in the OPT situation the imaginary positive sequence current takes rather small values. The real negative sequence current becomes predominant. It is shocking that despite the enormous differences in the distribution of currents, the objective functions are not far apart one from the other. 

The results for the line to line fault together with the ones from the double line to ground fault seem to be the most dubious. First, in the line to line fault the objective functions are somewhat larger than what it could be expected from Figure \ref{fig:LLx1_c}. In any case, while the real positive sequence and the imaginary negative sequence currents take extremely similar values, the differences are acute for the remaining two currents. The OPT case prioritizes the real negative sequence current whereas the ROPT opts for the imaginary positive sequence current. 

No more intuitively sound seem to be the double line to ground fault values. The objective function becomes considerably smaller than in the case of the $R/X$ analysis, and again, the distribution of currents reminds of the one for the line to line fault. This could be expected. However, the extreme differences are hardly justifiable. 
\newpage
% \section{Conclusion}
This study has covered the analysis of a simple system to determine the optimal currents that ought to be injected to raise the positive sequence voltage and decrease the negative sequence voltage in case of a fault. Four types of faults have been considered. Despite the particularities, since the system is mainly inductive, the optimal active currents tend to be close to zero, while the optimal reactive currents become substantial. As a direct consequence, we conclude by saying that injecting only reactive powers, as imposed by the grid codes, is likely to be a convenient strategy for systems where lines are highly inductive. However, it is mandatory to determine the type of fault in order to properly deduce the values of the injected currents.

In addition to that, this work proposes two methodologies to arrive to the optimal solution. One consists of computing combinations where currents can take a wide range of values. When the intervals are small enough, such computationally intensive approach matches with the solution coming from solving directly the optimization problem. The results indicate that the double line to ground fault is the hardest in terms of minimizing the objective function. Besides, we have discussed the presence of multiple optimal points for the line to ground fault. 

\newpage
\printbibliography
\newpage
% \appendix
\section{$abc$ circuits}

\begin{figure}[!htb] \centering
\begin{circuitikz}[european]
\thicklines

\draw (0,0) to [american controlled current source,  *-] (2,0);
\draw (0,-2) to [american controlled current source, *-] (2,-2);
\draw (0,-4) to [american controlled current source, *-] (2,-4);
\draw (2,0) to [R, l=$\underline{Z}_a$, -] (4,0);
\draw (2,-2) to [R, l=$\underline{Z}_a$, -] (4,-2);
\draw (2,-4) to [R, l=$\underline{Z}_a$, -] (4,-4);
\draw (4,0) to [short] (6,0);
\draw (4,-2) to [short] (6,-2);
\draw (4,-4) to [short] (6,-4);
\draw (0,0) to [short] (0,-4);

\draw (6,0) to [R, l=$\underline{Z}_{th}$, -] (8,0);
\draw (6,-2) to [R, l=$\underline{Z}_{th}$, -] (8,-2);
\draw (6,-4) to [R, l=$\underline{Z}_{th}$, -] (8,-4);
\draw (10,0) to [sV, v_=$\underline{V}^{a}_{th}$, *-] (8,0);
\draw (10,-2) to [sV, v_=$\underline{V}^{b}_{th}$, *-] (8,-2);
\draw (10,-4) to [sV, v_=$\underline{V}^{c}_{th}$, *-] (8,-4);
\draw (10, 0) to [short] (10, -4);
\draw (10,-2) to [short] (11.5,-2);
\draw (11.5,-2) to [short] (11.5, -4.5);
\draw (11.5,-4.5) -- (11.5,-5.00) node[ground]{};


\draw (1.5,-0.4) node[]{$\underline{I}^a$};
\draw (2,0.3) to [open,v=$\underline{V}^a_c$] (0,0.3);
\draw (1.5,-2.4) node[]{$\underline{I}^b$};
\draw (2,-1.7) to [open,v=$\underline{V}^b_c$] (0,-1.7);
\draw (1.5,-4.4) node[]{$\underline{I}^c$};
\draw (2,-3.7) to [open,v=$\underline{V}^c_c$] (0,-3.7);

\draw (4.0,0) to [short, *-] (4.0,-4);
\draw (5.0,-2) to [short, *-] (5.0,-4);

\draw (4.0,-4) to [R, l=$\underline{Z}_f$, -*] (4.0,-6);
\draw (5.0,-4) to [R, l=$\underline{Z}_f$, -*] (5.0,-6);
\draw (6.0,-4) to [R, l=$\underline{Z}_f$, *-*] (6.0,-6);
\draw (4,-6) to [short] (6,-6);
 
\end{circuitikz}
\caption{Balanced fault schematic}
\label{fig:3_trif}
\end{figure}


\begin{figure}[!htb] \centering
\begin{circuitikz}[european]
\thicklines

\draw (0,0) to [american controlled current source,  *-] (2,0);
\draw (0,-2) to [american controlled current source, *-] (2,-2);
\draw (0,-4) to [american controlled current source, *-] (2,-4);
\draw (2,0) to [R, l=$\underline{Z}_a$, -] (4,0);
\draw (2,-2) to [R, l=$\underline{Z}_a$, -] (4,-2);
\draw (2,-4) to [R, l=$\underline{Z}_a$, -] (4,-4);
\draw (4,0) to [short] (6,0);
\draw (4,-2) to [short] (6,-2);
\draw (4,-4) to [short] (6,-4);
\draw (0,0) to [short] (0,-4);

\draw (6,0) to [R, l=$\underline{Z}_{th}$, -] (8,0);
\draw (6,-2) to [R, l=$\underline{Z}_{th}$, -] (8,-2);
\draw (6,-4) to [R, l=$\underline{Z}_{th}$, -] (8,-4);
\draw (10,0) to [sV, v_=$\underline{V}^{a}_{th}$, *-] (8,0);
\draw (10,-2) to [sV, v_=$\underline{V}^{b}_{th}$, *-] (8,-2);
\draw (10,-4) to [sV, v_=$\underline{V}^{c}_{th}$, *-] (8,-4);
\draw (10, 0) to [short] (10, -4);
\draw (10,-2) to [short] (11.5,-2);
\draw (11.5,-2) to [short] (11.5, -4.5);
\draw (11.5,-4.5) -- (11.5,-5.00) node[ground]{};


\draw (1.5,-0.4) node[]{$\underline{I}^a$};
\draw (2,0.3) to [open,v=$\underline{V}^a_c$] (0,0.3);
\draw (1.5,-2.4) node[]{$\underline{I}^b$};
\draw (2,-1.7) to [open,v=$\underline{V}^b_c$] (0,-1.7);
\draw (1.5,-4.4) node[]{$\underline{I}^c$};
\draw (2,-3.7) to [open,v=$\underline{V}^c_c$] (0,-3.7);

\draw (5.0, 0) to [short, *-] (5.0, -4);
\draw (5.0,-4) to [R, l=$\underline{Z}_f$] (5.0,-5.5);
\draw (5.0,-5.5) -- (5.0,-5.5) node[ground]{};
 
\end{circuitikz}
\caption{Line to ground fault schematic}
\label{fig:3_lg}
\end{figure}


\begin{figure}[!htb] \centering
\begin{circuitikz}[european]
\thicklines

\draw (0,0) to [american controlled current source,  *-] (2,0);
\draw (0,-2) to [american controlled current source, *-] (2,-2);
\draw (0,-4) to [american controlled current source, *-] (2,-4);
\draw (2,0) to [R, l=$\underline{Z}_a$, -] (4,0);
\draw (2,-2) to [R, l=$\underline{Z}_a$, -] (4,-2);
\draw (2,-4) to [R, l=$\underline{Z}_a$, -] (4,-4);
\draw (4,0) to [short] (6,0);
\draw (4,-2) to [short] (6,-2);
\draw (4,-4) to [short] (6,-4);
\draw (0,0) to [short] (0,-4);

\draw (6,0) to [R, l=$\underline{Z}_{th}$, -] (8,0);
\draw (6,-2) to [R, l=$\underline{Z}_{th}$, -] (8,-2);
\draw (6,-4) to [R, l=$\underline{Z}_{th}$, -] (8,-4);
\draw (10,0) to [sV, v_=$\underline{V}^{a}_{th}$, *-] (8,0);
\draw (10,-2) to [sV, v_=$\underline{V}^{b}_{th}$, *-] (8,-2);
\draw (10,-4) to [sV, v_=$\underline{V}^{c}_{th}$, *-] (8,-4);
\draw (10, 0) to [short] (10, -4);
\draw (10,-2) to [short] (11.5,-2);
\draw (11.5,-2) to [short] (11.5, -4.5);
\draw (11.5,-4.5) -- (11.5,-5.00) node[ground]{};


\draw (1.5,-0.4) node[]{$\underline{I}^a$};
\draw (2,0.3) to [open,v=$\underline{V}^a_c$] (0,0.3);
\draw (1.5,-2.4) node[]{$\underline{I}^b$};
\draw (2,-1.7) to [open,v=$\underline{V}^b_c$] (0,-1.7);
\draw (1.5,-4.4) node[]{$\underline{I}^c$};
\draw (2,-3.7) to [open,v=$\underline{V}^c_c$] (0,-3.7);

\draw (4.0,-2) to [short, *-] (4.0, -5.0);
\draw (6.0,-4) to [short, *-] (6.0, -5.0);
\draw (4,-5.0) to [R, l=$\underline{Z}_f$] (6, -5.0);

 
\end{circuitikz}
\caption{Line to line fault schematic}
\label{fig:3_ll}
\end{figure}


\begin{figure}[!htb] \centering
\begin{circuitikz}[european]
\thicklines

\draw (0,0) to [american controlled current source,  *-] (2,0);
\draw (0,-2) to [american controlled current source, *-] (2,-2);
\draw (0,-4) to [american controlled current source, *-] (2,-4);
\draw (2,0) to [R, l=$\underline{Z}_a$, -] (4,0);
\draw (2,-2) to [R, l=$\underline{Z}_a$, -] (4,-2);
\draw (2,-4) to [R, l=$\underline{Z}_a$, -] (4,-4);
\draw (4,0) to [short] (6,0);
\draw (4,-2) to [short] (6,-2);
\draw (4,-4) to [short] (6,-4);
\draw (0,0) to [short] (0,-4);

\draw (6,0) to [R, l=$\underline{Z}_{th}$, -] (8,0);
\draw (6,-2) to [R, l=$\underline{Z}_{th}$, -] (8,-2);
\draw (6,-4) to [R, l=$\underline{Z}_{th}$, -] (8,-4);
\draw (10,0) to [sV, v_=$\underline{V}^{a}_{th}$, *-] (8,0);
\draw (10,-2) to [sV, v_=$\underline{V}^{b}_{th}$, *-] (8,-2);
\draw (10,-4) to [sV, v_=$\underline{V}^{c}_{th}$, *-] (8,-4);
\draw (10, 0) to [short] (10, -4);
\draw (10,-2) to [short] (11.5,-2);
\draw (11.5,-2) to [short] (11.5, -4.5);
\draw (11.5,-4.5) -- (11.5,-5.00) node[ground]{};


\draw (1.5,-0.4) node[]{$\underline{I}^a$};
\draw (2,0.3) to [open,v=$\underline{V}^a_c$] (0,0.3);
\draw (1.5,-2.4) node[]{$\underline{I}^b$};
\draw (2,-1.7) to [open,v=$\underline{V}^b_c$] (0,-1.7);
\draw (1.5,-4.4) node[]{$\underline{I}^c$};
\draw (2,-3.7) to [open,v=$\underline{V}^c_c$] (0,-3.7);

\draw (4.0,-2) to [short, *-] (4.0, -5.0);
\draw (6.0,-4) to [short, *-] (6.0, -5.0);
\draw (4,-5.0) to [short, -*] (5, -5.0);
\draw (5,-5.0) to [short, ] (6, -5.0);
\draw (5,-5) to [R, l=$\underline{Z}_f$] (5,-6.5);
\draw (5,-6.5) -- (5,-6.5) node[ground]{};

 
\end{circuitikz}
\caption{Double line to ground fault schematic}
\label{fig:3_llg}
\end{figure}

\clearpage
\newpage
\section{$+-0$ schemes}

\begin{figure}[!htb] \centering
\begin{circuitikz}[european]
\thicklines

\draw (0,6) to [sV, v_=$\underline{V}_{th}^+$] (0,8);
\draw (-3,8) to [R, l=$\underline{Z}_{th}$] (0,8);
\draw (-6,6) to [short] (0.0,6);
\draw (-6,8) to [R, l=$\underline{Z}_a$] (-3,8);
\draw (-6,6.5) to [american controlled current source, l_=$I^+$] (-6,7.5);
\draw (-6,6) to [short] (-6,6.5);
\draw (-6,7.5) to [short] (-6,8);
\draw (-6.3,8) to [open,v=$\underline{V}^+_c$] (-6.3,6);

\draw (0,3) to [short] (0,3.5);
\draw (0,3.5) to [short, *-*] (0,4.5);
\draw (0,4.5) to [short] (0,5);
\draw (-3,5) to [R, l=$\underline{Z}_{th}$] (0,5);
\draw (-6,3) to [short] (0.0,3);
\draw (-6,5) to [R, l=$\underline{Z}_a$] (-3,5);
\draw (-6,3.5) to [american controlled current source, l_=$I^-$] (-6,4.5);
\draw (-6,3) to [short] (-6,3.5);
\draw (-6,4.5) to [short] (-6,5);
\draw (-6.3,5) to [open,v=$\underline{V}^-_c$] (-6.3,3);

\draw (0,0) to [short] (0,0.5);
\draw (0,0.5) to [short, *-*] (0,1.5);
\draw (0,1.5) to [short] (0,2);
\draw (-3,2) to [R, l=$\underline{Z}_{th}$] (0,2);
\draw (-6,0) to [short] (0.0,0);
\draw (-6,2) to [R, l=$\underline{Z}_a$] (-3,2);
\draw (-6,0) to [short, -*] (-6,0.5);
\draw (-6,1.5) to [short, *-] (-6,2);
\draw (-6.3,2) to [open,v=$\underline{V}^0_c$] (-6.3,0);

\draw (-3,8) to [R, l=$\underline{Z}_{f}$] (-3,6);
\draw (-3,5) to [R, l=$\underline{Z}_{f}$] (-3,3);
\draw (-3,2) to [R, l=$\underline{Z}_{f}$] (-3,0);


\end{circuitikz}
\caption{Equivalent circuit for the balanced fault analysis}
\label{fig:sys_3x}
\end{figure}


\begin{figure}[!htb] \centering
\begin{circuitikz}[european]
\thicklines

\draw (0,6) to [sV, v_=$\underline{V}_{th}^+$] (0,8);
\draw (-3,8) to [R, l=$\underline{Z}_{th}$] (0,8);
\draw (-6,6) to [short] (0.0,6);
\draw (-6,8) to [R, l=$\underline{Z}_a$] (-3,8);
\draw (-6,6.5) to [american controlled current source, l_=$I^+$] (-6,7.5);
\draw (-6,6) to [short] (-6,6.5);
\draw (-6,7.5) to [short] (-6,8);
\draw (-6.3,8) to [open,v=$\underline{V}^+_c$] (-6.3,6);

\draw (0,3) to [short] (0,3.5);
\draw (0,3.5) to [short, *-*] (0,4.5);
\draw (0,4.5) to [short] (0,5);
\draw (-3,5) to [R, l=$\underline{Z}_{th}$] (0,5);
\draw (-6,3) to [short] (0.0,3);
\draw (-6,5) to [R, l=$\underline{Z}_a$] (-3,5);
\draw (-6,3.5) to [american controlled current source, l_=$I^-$] (-6,4.5);
\draw (-6,3) to [short] (-6,3.5);
\draw (-6,4.5) to [short] (-6,5);
\draw (-6.3,5) to [open,v=$\underline{V}^-_c$] (-6.3,3);

\draw (0,0) to [short] (0,0.5);
\draw (0,0.5) to [short, *-*] (0,1.5);
\draw (0,1.5) to [short] (0,2);
\draw (-3,2) to [R, l=$\underline{Z}_{th}$] (0,2);
\draw (-6,0) to [short] (0.0,0);
\draw (-6,2) to [R, l=$\underline{Z}_a$] (-3,2);
\draw (-6,0) to [short, -*] (-6,0.5);
\draw (-6,1.5) to [short, *-] (-6,2);
\draw (-6.3,2) to [open,v=$\underline{V}^0_c$] (-6.3,0);

\draw (-3,2) to [short, *-*] (-3,3);
\draw (-3,5) to [short, *-*] (-3,6);
\draw (-3,8) to [short, *-] (-3, 9);
\draw (-3,0) to [short, *-] (-3, -0.5);

\draw (-3,9) to [short] (2,9);
\draw (-3,-0.5) to [short] (2,-0.5);
\draw (2,9) to [R, l=$3\underline{Z}_f$] (2, -0.5);

\end{circuitikz}
\caption{Equivalent circuit for the line to ground fault analysis}
\label{fig:sys_LG}
\end{figure}

\begin{figure}[!htb] \centering
\begin{circuitikz}[european]
\thicklines

\draw (0,6) to [sV, v_=$\underline{V}_{th}^+$] (0,8);
\draw (-2.5,8) to [R, l=$\underline{Z}_{th}$] (0,8);
\draw (-5,6) to [short] (0.0,6);
\draw (-5,8) to [R, l=$\underline{Z}_a$] (-2.5,8);
\draw (-5,6.5) to [american controlled current source, l_=$I^+$] (-5,7.5);
\draw (-5,6) to [short] (-5,6.5);
\draw (-5,7.5) to [short] (-5,8);
\draw (-5.3,8) to [open,v=$\underline{V}^+_c$] (-5.3,6);

\draw (8,6) to [short] (8,6.5);
\draw (8,6.5) to [short, *-*] (8,7.5);
\draw (8,7.5) to [short] (8,8);
\draw (5.5,8) to [R, l=$\underline{Z}_{th}$] (8,8);
\draw (3,6) to [short] (8.0,6);
\draw (3,8) to [R, l=$\underline{Z}_a$] (5.5,8);
\draw (3,6.5) to [american controlled current source, l_=$I^-$] (3,7.5);
\draw (3,6) to [short] (3,6.5);
\draw (3,7.5) to [short] (3,8);
\draw (2.7,8) to [open,v=$\underline{V}^-_c$] (2.7,6);

\draw (-2.5, 8) to [short, *-] (-2.5, 9);
\draw (-2.5, 9) to [R, l=$\underline{Z}_{f}$] (5.5,9);
\draw (5.5,9) to [short, -*] (5.5, 8);

\draw (-2.5, 6) to [short, *-] (-2.5, 5.5);
\draw (-2.5, 5.5) to [short] (5.5, 5.5);
\draw (5.5, 5.5) to [short, -*] (5.5, 6);

\end{circuitikz}
\caption{Equivalent circuit for the line to line fault analysis}
\label{fig:sys_LL}
\end{figure}


\begin{figure}[!htb] \centering
\begin{circuitikz}[european]
\thicklines

\draw (0,8) to [sV, v_=$\underline{V}_{th}^+$] (0,10);
\draw (-3,10) to [R, l=$\underline{Z}_{th}$] (0,10);
\draw (-6,8) to [short] (0.0,8);
\draw (-6,10) to [R, l=$\underline{Z}_a$] (-3,10);
\draw (-6,8.5) to [american controlled current source, l_=$I^+$] (-6,9.5);
\draw (-6,8) to [short] (-6,8.5);
\draw (-6,9.5) to [short] (-6,10);
\draw (-6.3,10) to [open,v=$\underline{V}^+_c$] (-6.3,8);

\draw (0,4) to [short] (0,4.5);
\draw (0,4.5) to [short, *-*] (0,5.5);
\draw (0,5.5) to [short] (0,6);
\draw (-3,6) to [R, l=$\underline{Z}_{th}$] (0,6);
\draw (-6,4) to [short] (0.0,4);
\draw (-6,6) to [R, l=$\underline{Z}_a$] (-3,6);
\draw (-6,4.5) to [american controlled current source, l_=$I^-$] (-6,5.5);
\draw (-6,4) to [short] (-6,4.5);
\draw (-6,5.5) to [short] (-6,6);
\draw (-6.3,6) to [open,v=$\underline{V}^-_c$] (-6.3,4);

\draw (0,0) to [short] (0,0.5);
\draw (0,0.5) to [short, *-*] (0,1.5);
\draw (0,1.5) to [short] (0,2);
\draw (-3,2) to [R, l=$\underline{Z}_{th}$] (0,2);
\draw (-6,0) to [short] (0.0,0);
\draw (-6,2) to [R, l=$\underline{Z}_a$] (-3,2);
\draw (-6,0) to [short, -*] (-6,0.5);
\draw (-6,1.5) to [short, *-] (-6,2);
\draw (-6.3,2) to [open,v=$\underline{V}^0_c$] (-6.3,0);

\draw (-3,0) to [short, *-] (-3,-0.5);
\draw (-3,4) to [short, *-] (-3,3.5);
\draw (-3,8) to [short, *-] (-3,7.5);

\draw (-3,2) to [short, *-] (-3,3);
\draw (-3,6) to [short, *-] (-3,7);
\draw (-3,10) to [short, *-] (-3,11);

\draw (-3,3) to [short] (2,3);
\draw (-3,7) to [short] (2,7);
\draw (-3,11) to [short] (2,11);

\draw (-3,-0.5) to [short] (-8,-0.5);
\draw (-3,3.5) to [short] (-8,3.5);
\draw (-3,7.5) to [short] (-8,7.5);

\draw (-8,-0.5) to [short, -*] (-8, 3.5);
\draw (-8, 3.5) to [short] (-8, 7.5);

\draw (2,3) to [R, l_=$3\underline{Z}_f$, -*] (2,7);
\draw (2,11) to [short] (2,7);


\end{circuitikz}
\caption{Equivalent circuit for the double line to ground fault analysis}
\label{fig:sys_LLG}
\end{figure}



\clearpage
\newpage
\section{Expressions}

\subsection{Balanced fault}
\begin{equation}
    \begin{cases}
        \underline{V}^a_c = \dfrac{1}{\underline{Z}_f + \underline{Z}_{th}}[\underline{V}^a_{th}\underline{Z}_f + \underline{I}_a(\underline{Z}_a\underline{Z}_{th} + \underline{Z}_{th}\underline{Z}_f + \underline{Z}_f\underline{Z}_a)]\\
        \underline{V}^b_c = \dfrac{1}{\underline{Z}_f + \underline{Z}_{th}}[\underline{V}^b_{th}\underline{Z}_f + \underline{I}_b(\underline{Z}_a\underline{Z}_{th} + \underline{Z}_{th}\underline{Z}_f + \underline{Z}_f\underline{Z}_a)]\\
        \underline{V}^c_c = \dfrac{1}{\underline{Z}_f + \underline{Z}_{th}}[\underline{V}^c_{th}\underline{Z}_f + \underline{I}_c(\underline{Z}_a\underline{Z}_{th} + \underline{Z}_{th}\underline{Z}_f + \underline{Z}_f\underline{Z}_a)]
    \end{cases}
\end{equation}


\begin{equation}
    \begin{cases}
        \underline{V}^+_c= \dfrac{1}{\underline{Z}_f + \underline{Z}_{th}}[\underline{V}^+_{th}\underline{Z}_f + \underline{I}^+(\underline{Z}_a\underline{Z}_f + \underline{Z}_a\underline{Z}_{th} + \underline{Z}_f\underline{Z}_{th})] \\ 
 \underline{V}^-_c=\dfrac{1}{\underline{Z}_{f} + \underline{Z}_{th}}[\underline{I}^-(\underline{Z}_{th}\underline{Z}_f + \underline{Z}_a\underline{Z}_{th} + \underline{Z}_a\underline{Z}_f)] \\
 \underline{V}^0_c=0
    \end{cases}
\end{equation}


\subsection{Line to ground fault}
\begin{equation}
    \begin{cases}
        \underline{V}^a_c = \dfrac{1}{\underline{Z}_{th} + \underline{Z}_f}[\underline{I}_a(\underline{Z}_a\underline{Z}_{th} + \underline{Z}_a\underline{Z}_f + \underline{Z}_{th}\underline{Z}_f) + \underline{V}^a_{th}\underline{Z}_f] \\
        \underline{V}^b_c = \underline{V}^b_{th} + \underline{I}_b(\underline{Z}_a + \underline{Z}_{th}) \\
        \underline{V}^c_c =  \underline{V}^c_{th} + \underline{I}_c(\underline{Z}_a + \underline{Z}_{th})
    \end{cases}
\end{equation}

\begin{equation}
    \begin{cases}
        \underline{V}^+_c=\underline{I}^+(\underline{Z}_a + \underline{Z}_{th}) + \underline{V}^+_{th} - \dfrac{\underline{Z}_{th}}{3\underline{Z}_f + 3\underline{Z}_{th}}[\underline{I}^+\underline{Z}_{th} + \underline{I}^-\underline{Z}_{th} + \underline{V}^+_{th}]\\ 
        \underline{V}^-_c=\underline{I}^-(\underline{Z}_a + \underline{Z}_{th}) - \dfrac{\underline{Z}_{th}}{3\underline{Z}_f + 3\underline{Z}_{th}}[\underline{I}^+\underline{Z}_{th} + \underline{I}^-\underline{Z}_{th} + \underline{V}^+_{th}]\\
        \underline{V}^0_c=-\dfrac{\underline{Z}_{th}}{3\underline{Z}_f + 3\underline{Z}_{th}}[\underline{V}^+_{th} + \underline{I}^+\underline{Z}_{th} + \underline{I}^-\underline{Z}_{th}]
   \end{cases}
\end{equation}


\subsection{Line to line fault}
\begin{equation}
    \begin{cases}
        \underline{V}^a_c = \underline{I}_a (\underline{Z}_a + \underline{Z}_{th}) + \underline{V}^a_{th}  \\
        \underline{V}^b_c = \underline{I}_b\underline{Z}_a + \dfrac{1}{(\underline{Z}_{f} + \underline{Z}_{th}) (\underline{Z}_f + 2\underline{Z}_{th})}[\underline{I}_b (\underline{Z}_{th}\underline{Z}_f\underline{Z}_f + 2\underline{Z}_{th}\underline{Z}_{th}\underline{Z}_{f} + \underline{Z}_{th}\underline{Z}_{th}\underline{Z}_{th}) \\+ \underline{I}_c(\underline{Z}_{th}\underline{Z}_{th}\underline{Z}_f + \underline{Z}_{th}\underline{Z}_{th}\underline{Z}_{th}) + \underline{V}^b_{th} (\underline{Z}_f\underline{Z}_f + 2 \underline{Z}_f\underline{Z}_{th} + \underline{Z}_{th}\underline{Z}_{th}) + \underline{V}^c_{th}(\underline{Z}_{th}\underline{Z}_f + \underline{Z}_{th} \underline{Z}_{th})  ] \\
        \underline{V}^c_c = \underline{I}_c\underline{Z}_a + \dfrac{1}{\underline{Z}_f + 2\underline{Z}_{th}}[\underline{I}_c(\underline{Z}_{th}(\underline{Z}_f + \underline{Z}_{th})) + \underline{V}^c_{th} (\underline{Z}_f + \underline{Z}_{th}) + \underline{I}_b\underline{Z}_{th}\underline{Z}_{th} + \underline{V}^b_{th}\underline{Z}_{th}]
    \end{cases}
\end{equation}

\begin{equation}
    \begin{cases}
        \underline{V}^+_c= \underline{V}^+_{th} + \underline{I}^+(\underline{Z}_{a} + \underline{Z}_{th}) - \dfrac{\underline{Z}_{th}}{2\underline{Z}_{th}+\underline{Z}_f}[\underline{V}^+_{th} + \underline{I}^+\underline{Z}_{th}-\underline{I}^-\underline{Z}_{th}]  \\ 
        \underline{V}^-_c= \underline{V}^+_{th} + \underline{I}^+\underline{Z}_{th}+ \underline{I}^-\underline{Z}_a -\dfrac{\underline{Z}_{th}+\underline{Z}_f}{2\underline{Z}_{th}+\underline{Z}_f}[\underline{V}^+_{th} + \underline{I}^+\underline{Z}_{th} - \underline{I}^-\underline{Z}_{th}] \\
        \underline{V}^0_c=0
    \end{cases}
\end{equation} 


\subsection{Double line to ground fault}
\begin{equation}
    \begin{cases}
        \underline{V}^a_c = \underline{I}_a (\underline{Z}_a + \underline{Z}_{th}) + \underline{V}^a_{th}  \\
        \underline{V}^b_c = \underline{I}_b\underline{Z}_a + \dfrac{\underline{Z}_{th}\underline{Z}_f (\underline{I}_b+\underline{I}_c) + \underline{Z}_f(\underline{V}^b_{th} + \underline{V}^c_{th})}{2\underline{Z}_f + \underline{Z}_{th}} \\
        \underline{V}^c_c = \underline{I}_c\underline{Z}_a + \dfrac{\underline{Z}_{th}\underline{Z}_f (\underline{I}_b+\underline{I}_c) + \underline{Z}_f(\underline{V}^b_{th} + \underline{V}^c_{th})}{2\underline{Z}_f + \underline{Z}_{th}} 
    \end{cases}
\end{equation}

\begin{equation}
    \begin{cases}
        \underline{V}^+_c=\underline{I}^+\underline{Z}_a + \dfrac{\underline{Z}_{th} + 3\underline{Z}_f}{3\underline{Z}_{th} + 6\underline{Z}_f}[\underline{I}^+\underline{Z}_{th} + \underline{I}^-\underline{Z}_{th} + \underline{V}^+_{th}]    \\ 
        \underline{V}^-_c= \underline{I}^-\underline{Z}_a + \dfrac{\underline{Z}_{th} + 3\underline{Z}_f}{3\underline{Z}_{th} + 6\underline{Z}_f}[\underline{I}^+\underline{Z}_{th} + \underline{I}^-\underline{Z}_{th} + \underline{V}^+_{th}] \\
        \underline{V}^0_c=\dfrac{\underline{Z}_{th}}{3\underline{Z}_{th} + 6\underline{Z}_f}[\underline{I}^+\underline{Z}_{th} + \underline{I}^-\underline{Z}_{th} + \underline{V}^+_{th}]
    \end{cases}
\end{equation}

\end{document}

\section{Submarine cable}
Having seen the influence of varying the ratio between the resistive and inductive part of $\underline{Z}_{th}$ and $\underline{Z}_a$, we can now go a step further by including a hypothetical submarine cable between the grid and the point where the fault occurs. The system takes the form shown in Figure \ref{fig:sys_pc}.

\begin{figure}[!htb] \centering
\begin{circuitikz}[european]
\thicklines

\draw (3,0) to [sV, v_=$\underline{V}_{th}$] (3,2);
\draw (-0,2) to [R, l=$\underline{Z}_{th}$] (3,2);
\draw (0,2) to [R, l_=$\underline{Z}_s$] (-2,2);
\draw (0.25,2) to [R, l_=$\underline{Z}_p$] (0.25,0);
\draw (-2.25,2) to [R, l=$\underline{Z}_p$] (-2.25,0);
\draw (-2,2) to [short] (-3,2);

\draw (3.25,0) to [short] (2.75,0);
\draw (0,0) to [short] (0.5,0);
\draw (-2.5,0) to [short] (-2.0,0);

\draw (-3.25,0) to [short] (-2.75,0);
\node at (-6,1.3) {PCC};
\node at (-6,2.7) {$\underline{V}_c$};
\draw (-3,2) to [short] (-3.2,1);
\draw (-3.2,1) to [short] (-2.8,1);
\draw[-{Latex[length=3mm]}] (-2.8,1) to [short] (-3,0);
\draw (-6,2) to [R, l=$\underline{Z}_a$] (-3,2);
\draw[line width=0.65mm] (-6,2.5) to [short] (-6,1.5);
\draw[line width=0.65mm] (-3,2.5) to [short] (-3,1.5);
\draw (-9,2) to [R, l=$\underline{Z}_c$, i=$\underline{I}$] (-6,2);
\draw (-10.0,2) to [sdcac] (-9.0,2);


\end{circuitikz}
\caption{Single-phase representation of the simple system under a fault with a submarine cable}
\label{fig:sys_pc}
\end{figure}
This system can be studied without added complexity once a Thevenin equivalent is formed on the right hand side of the fault. Opting for this will imply that we will be able to recycle the expressions developed before. The new Thevenin voltage and impedance, denoted by $\underline{V}'_{th}$ and $\underline{Z}'_{th}$, are given by:
\begin{equation}
    \begin{cases}
        \underline{V}'_{th} = \dfrac{\underline{Z}_p\underline{Z_p}}{2\underline{Z}_{th}\underline{Z}_p + \underline{Z}_p\underline{Z}_s + \underline{Z}_p\underline{Z}_p + \underline{Z}_{th}\underline{Z}_s}\underline{V}_{th},\\[18pt]
        \underline{Z}'_{th} = \dfrac{\underline{Z}_p\underline{Z}_p\underline{Z}_{th} + \underline{Z}_s\underline{Z}_p\underline{Z}_p + \underline{Z}_{th}\underline{Z}_s\underline{Z}_p}{2\underline{Z}_{th}\underline{Z}_p + \underline{Z}_p\underline{Z}_s + \underline{Z}_p\underline{Z}_p + \underline{Z}_{th}\underline{Z}_s}.
    \end{cases}
\end{equation}
The distance of the cable is going to take multiple values in order to weight its influence. Realistic data for the $\underline{Z}_s$ and $\underline{Z}_p$ impedances are taken from \cite{cheah2017offshore} and adapted. They appear in Table \ref{tab:imp_cable}.
\begin{table}[!htb]\centering
    \begin{tabular}{ccc}
        \hline
        Magnitude & Value & Units\\
        \hline
        $\underline{Z}_s$ & $6.674\cdot 10^{-5} + j2.597\cdot 10^{-4}$ & pu/km\\
        $\underline{Z}_p$ & $-j77.372$ & pu$\cdot$km\\
        \hline
    \end{tabular}
    \caption{Impedances values to be used in the submarine cable analysis}
    \label{tab:imp_cable}
\end{table}

Figure \ref{fig:dist_th} displays the resulting Thevenin voltage and impedance for a varying cable distance.

\begin{figure}[!htb]\centering \footnotesize
    \begin{subfigure}[!htb]{.4\textwidth}
      \centering
          \begin{tikzpicture}[trim axis right,trim axis left]
              \pgfplotsset{width=7cm, height=6.0cm}
              \begin{axis}[grid=major, xlabel={Distance (km)}, ylabel={Voltage (pu)}, /pgf/number format/.cd, legend style={at={(0.95,0.15)},anchor=south east,legend columns=1, draw=none, inner sep=0pt,fill=gray!10}, xtick={0,200,...,1000}, ytick={-5, -4,...,5}, axis line style = very thick, yticklabel style={/pgf/number format/fixed, /pgf/number format/precision=2}, xmin=0, xmax=1000, /pgf/number format/1000 sep={}]
              \addplot[very thick, black] table[x=x, y=y, col sep=comma] {Data/cable/Vth_re.csv};
              \addplot[very thick, gray] table[x=x, y=y, col sep=comma] {Data/cable/Vth_im.csv};
               \legend{$\Re (\underline{V}'_{th})$, $\Im (\underline{V}'_{th})$};
              \end{axis}
          \end{tikzpicture}
    \end{subfigure}
    \hspace{1.5cm}
    \begin{subfigure}[!htb]{.4\textwidth}
        \centering
            \begin{tikzpicture}[trim axis right,trim axis left]
                \pgfplotsset{width=7cm, height=6.0cm}
                \begin{axis}[grid=major, xlabel={Distance (km)}, ylabel={Impedance (pu)}, /pgf/number format/.cd, legend style={at={(0.95,0.55)},anchor=south east,legend columns=1, draw=none, inner sep=0pt,fill=gray!10}, xtick={0,200,...,1000}, axis line style = thick, yticklabel style={/pgf/number format/fixed, /pgf/number format/precision=4}, scaled y ticks = false, xmin=0, xmax=1000, ytick={-0.5,-0.25,...,0.75}, /pgf/number format/1000 sep={}]
                \addplot[very thick, black] table[x=x, y=y, col sep=comma] {Data/cable/Zth_re.csv};
                \addplot[very thick, gray] table[x=x, y=y, col sep=comma] {Data/cable/Zth_im.csv};
                \legend{$\Re (\underline{Z}'_{th})$, $\Im (\underline{Z}'_{th})$};
                \end{axis}
            \end{tikzpicture}
      \end{subfigure}
    \caption{Influence of the cable distance on the Thevenin voltage and impedance}
    \label{fig:dist_th}
  \end{figure}
The real part of the impedance always remains above zero. However, the imaginary part starts being inductive and then becomes capacitive for large cable distances. The voltages can present extremely high values, but they coincide with a situation where the impedance is also high. Notice how abrupt changes take place around 400 km in both plots.

% We now deal with a submarine cable, which is modelled with a $\pi$ equivalent, where the series impedance is still $\underline{Z}_a$ while the parallel impedance is $\underline{Z}_c=-jX_c$. As follows, we sweep across $X_c$.

\subsection{Balanced fault}
Figure \ref{fig:3x1_s} depicts the optimal currents for the balanced fault.

\begin{figure}[!htb]\centering \footnotesize
    \begin{subfigure}[!htb]{.4\textwidth}
      \centering
          \begin{tikzpicture}[trim axis right,trim axis left]
              \pgfplotsset{width=7cm, height=5.6cm}
              \begin{axis}[grid=major, xlabel={Distance (km)}, ylabel={${I}^+_{re}$}, /pgf/number format/.cd, legend style={at={(0.98,0.15)},anchor=south east,legend columns=1, draw=none, inner sep=0pt,fill=gray!10}, xtick={0,20,...,100}, ytick={0.0,0.1,...,0.5}, axis line style = thick, yticklabel style={/pgf/number format/fixed, /pgf/number format/precision=2}, xmin=0, xmax=100]
              \addplot[very thick, black] table[x=x, y=y, meta=label, col sep=comma] {Data/dist/I1_re_3x.csv};
              \addplot[very thick, red] table[x=x, y=y, meta=label, col sep=comma] {Data/dist/GI1_re_3x.csv};
              \addplot[very thick, gray, dashed] table[x=x, y=y, meta=label, col sep=comma] {Data/dist/RI1_re_3x.csv};
              % \legend{BF, OPT};
              \end{axis}
          \end{tikzpicture}
    \end{subfigure}
    \hspace{1.5cm}
    \begin{subfigure}[!htb]{.4\textwidth}
        \centering
            \begin{tikzpicture}[trim axis right,trim axis left]
                \pgfplotsset{width=7cm, height=5.6cm}
                \begin{axis}[grid=major, xlabel={Distance (km)}, ylabel={${I}^+_{im}$}, /pgf/number format/.cd, legend style={at={(0.98,0.15)},anchor=south east,legend columns=1, draw=none, inner sep=0pt,fill=gray!10}, xtick={0,20,...,100}, ytick={-1,-0.97,...,-0.85}, axis line style = thick, yticklabel style={/pgf/number format/fixed, /pgf/number format/precision=3}, xmin=0, xmax=100]
                \addplot[very thick, black] table[x=x, y=y, meta=label, col sep=comma] {Data/dist/I1_im_3x.csv};
                \addplot[very thick, red] table[x=x, y=y, meta=label, col sep=comma] {Data/dist/GI1_im_3x.csv};
                \addplot[very thick, gray, dashed] table[x=x, y=y, meta=label, col sep=comma] {Data/dist/RI1_im_3x.csv};
                % \legend{OPT};
                \end{axis}
            \end{tikzpicture}
      \end{subfigure}

      \vspace{0.5cm}

      \begin{subfigure}[!htb]{.4\textwidth}
        \centering
            \begin{tikzpicture}[trim axis right,trim axis left]
                \pgfplotsset{width=7cm, height=5.6cm}
                \begin{axis}[grid=major, xlabel={Distance (km)}, ylabel={${I}^-_{re}$}, /pgf/number format/.cd, legend style={at={(0.98,0.15)},anchor=south east,legend columns=1, draw=none, inner sep=0pt,fill=gray!10}, xtick={0,20,...,100}, ymin = -0.2, ymax=0.2, axis line style = thick, xmin=0, xmax=100]
                \addplot[very thick, black] table[x=x, y=y, meta=label, col sep=comma] {Data/dist/I2_re_3x.csv};
                \addplot[very thick, red] table[x=x, y=y, meta=label, col sep=comma] {Data/dist/GI2_re_3x.csv};
                \addplot[very thick, gray, dashed] table[x=x, y=y, meta=label, col sep=comma] {Data/dist/RI2_re_3x.csv};
                % \legend{BF, OPT};
                \end{axis}
            \end{tikzpicture}
      \end{subfigure}
      \hspace{1.5cm}
      \begin{subfigure}[!htb]{.4\textwidth}
          \centering
              \begin{tikzpicture}[trim axis right,trim axis left]
                  \pgfplotsset{width=7cm, height=5.6cm}
                  \begin{axis}[grid=major, xlabel={Distance (km)}, ylabel={${I}^-_{im}$}, /pgf/number format/.cd, legend style={at={(0.98,0.15)},anchor=south east,legend columns=1, draw=none, inner sep=0pt,fill=gray!10}, xtick={0,20,...,100}, ymin = -0.2, ymax=0.2, axis line style = thick, xmin=0, xmax=100]
                  \addplot[very thick, black] table[x=x, y=y, meta=label, col sep=comma] {Data/dist/I2_im_3x.csv};
                  \addplot[very thick, red] table[x=x, y=y, meta=label, col sep=comma] {Data/dist/GI2_im_3x.csv};
                  \addplot[very thick, gray, dashed] table[x=x, y=y, meta=label, col sep=comma] {Data/dist/RI2_im_3x.csv};
                  % \legend{BF, OPT};
                  \end{axis}
              \end{tikzpicture}
        \end{subfigure}

        \vspace{0.2cm}

    \begin{center}
        \begin{subfigure}[!htb]{0.7\textwidth}
                \begin{tikzpicture}[trim axis right,trim axis left]
                    \pgfplotsset{width=12.5cm, height=5.6cm}
                    \begin{axis}[grid=major, xlabel={Distance (km)}, ylabel={$f$}, /pgf/number format/.cd, legend style={at={(0.98,0.2)},anchor=south east,legend columns=1, draw=none, inner sep=0pt,fill=gray!10}, xtick={0,10,...,100}, axis line style = thick, yticklabel style={/pgf/number format/fixed, /pgf/number format/precision=3}, xmin=0, xmax=100, ytick={0.45, 0.475,...,0.6}]
                   \addplot[very thick, black] table[x=x, y=y, meta=label, col sep=comma] {Data/dist/ff_3x.csv};
                   \addplot[very thick, red] table[x=x, y=y, meta=label, col sep=comma] {Data/dist/Gff_3x.csv};
                   \addplot[very thick, gray, dashed] table[x=x, y=y, meta=label, col sep=comma] {Data/dist/Rff_3x.csv};
                    \legend{OPT, Grid Code, ROPT};
                    \end{axis}
                \end{tikzpicture}
        \end{subfigure}
    \end{center}
    \caption{Influence of the currents on the objective function for the balanced fault with $\underline{Z}_f=0.03$ and a submarine cable. OPT: solution to the optimization problem, ROPT: solution to the optimization problem restricted to only injecting reactive power.}
    \label{fig:3x1_s}
  \end{figure}
As it can be deduced, the imaginary part of the positive sequence current is the one to be prioritized, specially for short distances. Notice that when the distance increases, as shown in Figure \ref{fig:dist_th}, both the real and imaginary part of the Thevenin impedance increase. However, in proportion, the real part becomes a bit more relevant. Therefore, in the OPT case, some of the imaginary positive sequence current is traded for a bit more real current. The negative sequence currents are null for all ranges of distances because of the nature of the balanced fault. As it has been explained before, it makes no sense to inject a negative sequence current for a balanced fault due to having initially an already null negative sequence voltage. 

The ROPT turns out to be a more unfavorable case. This is the result of not being able to inject any real current. Thus, the positive sequence current becomes totally reactive as it remains constant at -1, which corresponds to the maximum allowed current. Increasing the cable distance suggests that the longer the cable, the more severe the fault is. Nevertheless, the variations are relatively small. We can conclude that for this fault, injecting or not active current does not have a huge influence on the final results. 

In addition to all that, the imaginary sequence current in the Grid Code mimics the one obtained for the ROPT case for distances larger than $20$ km. For short distances, the Grid Code considers the fault to not be too extreme, so not all current capability of the converter is used. This has also happened in some cases when varying the $R/X$ ratio. 

Figure \ref{fig:full_dist_3x} shows the voltages profile. The positive sequence absolute value of the voltage more or less resembles the reverse trend of the objective function $f$. It experiences relatively small variations as well.


\begin{figure}[!htb]\centering \footnotesize
    \begin{subfigure}[!htb]{.4\textwidth}
      \centering
          \begin{tikzpicture}[trim axis right,trim axis left]
              \pgfplotsset{width=7cm, height=7.0cm}
              \begin{axis}[grid=major, xlabel={Distance (km)}, ylabel={$|V^+|$}, /pgf/number format/.cd, legend style={at={(0.98,0.15)},anchor=south east,legend columns=1, draw=none, inner sep=0pt,fill=gray!10}, xtick={0,20,...,100}, axis line style = thick, ytick distance={0.02}, yticklabel style={/pgf/number format/fixed, /pgf/number format/precision=2}, xmin=0, xmax=100]
              \addplot[very thick, black] table[x=x, y=y, meta=label, col sep=comma] {Data/dist/V1_3x.csv};
              \addplot[very thick, red] table[x=x, y=y, meta=label, col sep=comma] {Data/dist/GV1_3x.csv};
              \addplot[very thick, gray, dashed] table[x=x, y=y, meta=label, col sep=comma] {Data/dist/RV1_3x.csv};
              % \legend{BF, OPT};
              \end{axis}
          \end{tikzpicture}
    \end{subfigure}
    \hspace{1.5cm}
    \begin{subfigure}[!htb]{.4\textwidth}
        \centering
            \begin{tikzpicture}[trim axis right,trim axis left]
                \pgfplotsset{width=7cm, height=7.0cm}
                \begin{axis}[grid=major, xlabel={Distance (km)}, ylabel={$|{V}^-|$}, /pgf/number format/.cd, legend style={at={(0.98,0.03)},anchor=south east,legend columns=1, draw=none, inner sep=0pt,fill=gray!10}, xtick={0, 20,...,100}, axis line style = thick, xmin=0, xmax=100, ytick distance={0.05}, yticklabel style={/pgf/number format/fixed, /pgf/number format/precision=3}, ymax = 0.1, ymin = -0.1]
                \addplot[very thick, black] table[x=x, y=y, meta=label, col sep=comma] {Data/dist/V2_3x.csv};
                \addplot[very thick, red] table[x=x, y=y, meta=label, col sep=comma] {Data/dist/GV2_3x.csv};
                \addplot[very thick, gray, dashed] table[x=x, y=y, meta=label, col sep=comma] {Data/dist/RV2_3x.csv};
                \legend{OPT, Grid Code, ROPT};
                \end{axis}
            \end{tikzpicture}
      \end{subfigure}
    \caption{Influence of the $R/X$ ratio on the voltages for the balanced fault with $\underline{Z}_f=0.03$. OPT: solution to the optimization problem, ROPT: solution to the optimization problem restricted to only injecting reactive power.}
    \label{fig:full_dist_3x}
  \end{figure}


The negative sequence voltage is of course 0 in all situations, as previously discussed. The positive sequence voltages tend to decrease with longer distances, which makes sense because the impedance of the cable increases. For all distances there exists a rather constant difference between the voltages. This difference is the same as in the objective function. We have found out that the fault impedance is by far the most influential impedance in the system. If we had opted for higher values, the objective functions would have been reduced and the objective voltages would have improved. Sometimes, when $\underline{Z}_f$ is big enough, the objective function can become zero. We have adjusted such fault impedance conveniently to cause a severe but not exceedingly strong fault. 


\subsection{Line to ground fault}
Figure \ref{fig:LG1_s} depicts the optimal currents for the line to ground fault.

\begin{figure}[!htb]\centering \footnotesize
    \begin{subfigure}[!htb]{.4\textwidth}
      \centering
          \begin{tikzpicture}[trim axis right,trim axis left]
              \pgfplotsset{width=7cm, height=6.2cm}
              \begin{axis}[grid=major, xlabel={Distance (km)}, ylabel={${I}^+_{re}$}, /pgf/number format/.cd, legend style={at={(0.98,0.15)},anchor=south east,legend columns=1, draw=none, inner sep=0pt,fill=gray!10}, xtick={0,20,...,100}, axis line style = thick, yticklabel style={/pgf/number format/fixed, /pgf/number format/precision=3}, xmin=0, xmax=100, ytick distance = 0.01, scaled y ticks = false]
              \addplot[very thick, black] table[x=x, y=y, meta=label, col sep=comma] {Data/dist/I1_re_LG.csv};
              \addplot[very thick, red] table[x=x, y=y, meta=label, col sep=comma] {Data/dist/GI1_re_LG.csv};
              \addplot[very thick, gray, dashed] table[x=x, y=y, meta=label, col sep=comma] {Data/dist/RI1_re_LG.csv};
              % \legend{BF, OPT};
              \end{axis}
          \end{tikzpicture}
    \end{subfigure}
    \hspace{1.5cm}
    \begin{subfigure}[!htb]{.4\textwidth}
        \centering
            \begin{tikzpicture}[trim axis right,trim axis left]
                \pgfplotsset{width=7cm, height=6.2cm}
                \begin{axis}[grid=major, xlabel={Distance (km)}, ylabel={${I}^+_{im}$}, /pgf/number format/.cd, legend style={at={(0.98,0.15)},anchor=south east,legend columns=1, draw=none, inner sep=0pt,fill=gray!10}, xtick={0,20,...,100}, axis line style = thick, yticklabel style={/pgf/number format/fixed, /pgf/number format/precision=3}, xmin=0, xmax=100, ytick distance = 0.1]
                \addplot[very thick, black] table[x=x, y=y, meta=label, col sep=comma] {Data/dist/I1_im_LG.csv};
                \addplot[very thick, red] table[x=x, y=y, meta=label, col sep=comma] {Data/dist/GI1_im_LG.csv};
                \addplot[very thick, gray, dashed] table[x=x, y=y, meta=label, col sep=comma] {Data/dist/RI1_im_LG.csv};
                % \legend{OPT};
                \end{axis}
            \end{tikzpicture}
      \end{subfigure}

      \vspace{0.5cm}

      \begin{subfigure}[!htb]{.4\textwidth}
        \centering
            \begin{tikzpicture}[trim axis right,trim axis left]
                \pgfplotsset{width=7cm, height=6.2cm}
                \begin{axis}[grid=major, xlabel={Distance (km)}, ylabel={${I}^-_{re}$}, /pgf/number format/.cd, legend style={at={(0.98,0.15)},anchor=south east,legend columns=1, draw=none, inner sep=0pt,fill=gray!10}, xtick={0,20,...,100}, axis line style = thick, xmin=0, xmax=100, ytick distance = {0.1}]
                \addplot[very thick, black] table[x=x, y=y, meta=label, col sep=comma] {Data/dist/I2_re_LG.csv};
                \addplot[very thick, red] table[x=x, y=y, meta=label, col sep=comma] {Data/dist/GI2_re_LG.csv};
                \addplot[very thick, gray, dashed] table[x=x, y=y, meta=label, col sep=comma] {Data/dist/RI2_re_LG.csv};
                % \legend{BF, OPT};
                \end{axis}
            \end{tikzpicture}
      \end{subfigure}
      \hspace{1.5cm}
      \begin{subfigure}[!htb]{.4\textwidth}
          \centering
              \begin{tikzpicture}[trim axis right,trim axis left]
                  \pgfplotsset{width=7cm, height=6.2cm}
                  \begin{axis}[grid=major, xlabel={Distance (km)}, ylabel={${I}^-_{im}$}, /pgf/number format/.cd, legend style={at={(0.98,0.15)},anchor=south east,legend columns=1, draw=none, inner sep=0pt,fill=gray!10}, xtick={0,20,...,100}, axis line style = thick, xmin=0, xmax=100, ytick distance = {0.1}]
                  \addplot[very thick, black] table[x=x, y=y, meta=label, col sep=comma] {Data/dist/I2_im_LG.csv};
                  \addplot[very thick, red] table[x=x, y=y, meta=label, col sep=comma] {Data/dist/GI2_im_LG.csv};
                  \addplot[very thick, gray, dashed] table[x=x, y=y, meta=label, col sep=comma] {Data/dist/RI2_im_LG.csv};
                  % \legend{BF, OPT};
                  \end{axis}
              \end{tikzpicture}
        \end{subfigure}

        \vspace{0.2cm}

    \begin{center}  
        \begin{subfigure}[!htb]{0.7\textwidth}
                \begin{tikzpicture}[trim axis right,trim axis left]
                    \pgfplotsset{width=12.5cm, height=6.2cm}
                    \begin{axis}[grid=major, xlabel={Distance (km)}, ylabel={$f$}, /pgf/number format/.cd, legend style={at={(0.98,0.48)},anchor=south east,legend columns=1, draw=none, inner sep=0pt,fill=gray!10}, xtick={0,10,...,100}, axis line style = thick, yticklabel style={/pgf/number format/fixed, /pgf/number format/precision=3}, xmin=0, xmax=100, ytick distance = {0.01}]
                   \addplot[very thick, black] table[x=x, y=y, meta=label, col sep=comma] {Data/dist/ff_LG.csv};
                   \addplot[very thick, red] table[x=x, y=y, meta=label, col sep=comma] {Data/dist/Gff_LG.csv};
                   \addplot[very thick, gray, dashed] table[x=x, y=y, meta=label, col sep=comma] {Data/dist/Rff_LG.csv};
                    \legend{OPT, Grid Code, ROPT};
                    \end{axis}
                \end{tikzpicture}
        \end{subfigure}
    \end{center}
    \caption{Influence of the currents on the objective function for the line to ground fault with $\underline{Z}_f=0.03$ and a submarine cable. OPT: solution to the optimization problem, ROPT: solution to the optimization problem restricted to only injecting reactive power.}
    \label{fig:LG1_s}
  \end{figure}
The most noticeable aspect in this situation is the evolution of the objective function $f$. The maximum occurs for a distance around 50 km, and then, the longer it is, the smaller the objective function becomes. There is about a 3\% improvement if we compare the objective functions for a 100 km length when compared to a distance close to 100 km. Again, there is a practically constant difference regarding the objective function between the OPT, the ROPT, and the Grid Code scenario. The latter turns out to be the worst for all lengths.

The currents vary slightly. For instance, the real positive sequence current diminishes a bit for the OPT case. It remains null for the Grid Code and the ROPT scenario, of course. The negative sequence current decreases at a similar rate as the positive one, but generally speaking it is kept nearly constant for all distances. Similar evolutions can be spotted for the imaginary currents. The prioritization in the ROPT case takes place for the negative sequence imaginary current, mostly. Analogously, the currents for the Grid Code evolve in an akin comparable manner. Notice that the increase in the negative sequence imaginary current is practically constant. We can already anticipate that the negative sequence voltage also evolves linearly. By the same token, the positive sequence voltage ought to increase in a rather exponential profile. 

Figure \ref{fig:full_dist_LG} presents the absolute value of the positive and negative sequence voltages and the objective function for all considered distances.

\begin{figure}[!htb]\centering \footnotesize
    \begin{subfigure}[!htb]{.4\textwidth}
      \centering
          \begin{tikzpicture}[trim axis right,trim axis left]
              \pgfplotsset{width=7cm, height=7.0cm}
              \begin{axis}[grid=major, xlabel={Distance (km)}, ylabel={$|V^+|$}, /pgf/number format/.cd, legend style={at={(0.98,0.15)},anchor=south east,legend columns=1, draw=none, inner sep=0pt,fill=gray!10}, xtick={0,20,...,100}, axis line style = thick, ytick distance={0.02}, yticklabel style={/pgf/number format/fixed, /pgf/number format/precision=2}, xmin=0, xmax=100]
              \addplot[very thick, black] table[x=x, y=y, meta=label, col sep=comma] {Data/dist/V1_LG.csv};
              \addplot[very thick, red] table[x=x, y=y, meta=label, col sep=comma] {Data/dist/GV1_LG.csv};
              \addplot[very thick, gray, dashed] table[x=x, y=y, meta=label, col sep=comma] {Data/dist/RV1_LG.csv};
              % \legend{BF, OPT};
              \end{axis}
          \end{tikzpicture}
    \end{subfigure}
    \hspace{1.5cm}
    \begin{subfigure}[!htb]{.4\textwidth}
        \centering
            \begin{tikzpicture}[trim axis right,trim axis left]
                \pgfplotsset{width=7cm, height=7.0cm}
                \begin{axis}[grid=major, xlabel={Distance (km)}, ylabel={$|{V}^-|$}, /pgf/number format/.cd, legend style={at={(0.98,0.03)},anchor=south east,legend columns=1, draw=none, inner sep=0pt,fill=gray!10}, xtick={0, 20,...,100}, axis line style = thick, xmin=0, xmax=100, ytick distance={0.02}, yticklabel style={/pgf/number format/fixed, /pgf/number format/precision=3}]
                \addplot[very thick, black] table[x=x, y=y, meta=label, col sep=comma] {Data/dist/V2_LG.csv};
                \addplot[very thick, red] table[x=x, y=y, meta=label, col sep=comma] {Data/dist/GV2_LG.csv};
                \addplot[very thick, gray, dashed] table[x=x, y=y, meta=label, col sep=comma] {Data/dist/RV2_LG.csv};
                \legend{OPT, Grid Code, ROPT};
                \end{axis}
            \end{tikzpicture}
      \end{subfigure}
    \caption{Influence of the $R/X$ ratio on the voltages for the line to ground fault with $\underline{Z}_f=0.03$. OPT: solution to the optimization problem, ROPT: solution to the optimization problem restricted to only injecting reactive power.}
    \label{fig:full_dist_LG}
  \end{figure}

In general, we have already seen how the objective function remains quite constant for all distances. However, larger distances imply that the positive sequence voltage can be improved, yet the negative sequence voltage also increases. Fortunately, the increase in the first is superior to the increase in the latter, specially for long cables, as the rate of change at 100 km is substantially larger than at 20 km, for instance. The positive sequence for the Grid Code turns out to be more favorable than for the ROPT case. The drawback of that has to do with having a negative sequence voltage way worse than in the OPT and the ROPT situations. The same happened for the line to ground fault, with the same fault impedance, when the $R/X$ ratio was changing. 

This time injecting active currents turns out to be more beneficial than in the balanced fault, where differences were hard to notice. Besides, we have tested that decreasing the fault impedance by a factor of 10 has a relatively moderate effect on the results. Under these conditions, the objective function is around 0.5, but most importantly, it always decreases the longer the cable is. Perhaps the increase in the Thevenin voltage depicted in Figure \ref{fig:dist_th} is responsible for that. 



\subsection{Line to line fault}
Figure \ref{fig:LLx1_s} depicts the optimal currents for the line to line fault.

\begin{figure}[!htb]\centering \footnotesize
    \begin{subfigure}[!htb]{.4\textwidth}
      \centering
          \begin{tikzpicture}[trim axis right,trim axis left]
              \pgfplotsset{width=7cm, height=6.2cm}
              \begin{axis}[grid=major, xlabel={Distance (km)}, ylabel={${I}^+_{re}$}, /pgf/number format/.cd, legend style={at={(0.98,0.15)},anchor=south east,legend columns=1, draw=none, inner sep=0pt,fill=gray!10}, xtick={0,20,...,100}, axis line style = thick, yticklabel style={/pgf/number format/fixed, /pgf/number format/precision=2}, xmin=0, xmax=100, ytick distance = {0.05}]
              \addplot[very thick, black] table[x=x, y=y, meta=label, col sep=comma] {Data/dist/I1_re_LL.csv};
              \addplot[very thick, red] table[x=x, y=y, meta=label, col sep=comma] {Data/dist/GI1_re_LL.csv};
              \addplot[very thick, gray, dashed] table[x=x, y=y, meta=label, col sep=comma] {Data/dist/RI1_re_LL.csv};
              % \legend{BF, OPT};
              \end{axis}
          \end{tikzpicture}
    \end{subfigure}
    \hspace{1.5cm}
    \begin{subfigure}[!htb]{.4\textwidth}
        \centering
            \begin{tikzpicture}[trim axis right,trim axis left]
                \pgfplotsset{width=7cm, height=6.2cm}
                \begin{axis}[grid=major, xlabel={Distance (km)}, ylabel={${I}^+_{im}$}, /pgf/number format/.cd, legend style={at={(0.98,0.15)},anchor=south east,legend columns=1, draw=none, inner sep=0pt,fill=gray!10}, xtick={0,20,...,100}, axis line style = thick, yticklabel style={/pgf/number format/fixed, /pgf/number format/precision=3}, xmin=0, xmax=100, ytick distance = {0.1}]
                \addplot[very thick, black] table[x=x, y=y, meta=label, col sep=comma] {Data/dist/I1_im_LL.csv};
                \addplot[very thick, red] table[x=x, y=y, meta=label, col sep=comma] {Data/dist/GI1_im_LL.csv};
                \addplot[very thick, gray, dashed] table[x=x, y=y, meta=label, col sep=comma] {Data/dist/RI1_im_LL.csv};
                % \legend{OPT};
                \end{axis}
            \end{tikzpicture}
      \end{subfigure}

      \vspace{0.5cm}

      \begin{subfigure}[!htb]{.4\textwidth}
        \centering
            \begin{tikzpicture}[trim axis right,trim axis left]
                \pgfplotsset{width=7cm, height=6.2cm}
                \begin{axis}[grid=major, xlabel={Distance (km)}, ylabel={${I}^-_{re}$}, /pgf/number format/.cd, legend style={at={(0.98,0.15)},anchor=south east,legend columns=1, draw=none, inner sep=0pt,fill=gray!10}, xtick={0,20,...,100}, axis line style = thick, xmin=0, xmax=100, yticklabel style={/pgf/number format/fixed, /pgf/number format/precision=3}]
                \addplot[very thick, black] table[x=x, y=y, meta=label, col sep=comma] {Data/dist/I2_re_LL.csv};
                \addplot[very thick, red] table[x=x, y=y, meta=label, col sep=comma] {Data/dist/GI2_re_LL.csv};
                \addplot[very thick, gray, dashed] table[x=x, y=y, meta=label, col sep=comma] {Data/dist/RI2_re_LL.csv};
                % \legend{BF, OPT};
                \end{axis}
            \end{tikzpicture}
      \end{subfigure}
      \hspace{1.5cm}
      \begin{subfigure}[!htb]{.4\textwidth}
          \centering
              \begin{tikzpicture}[trim axis right,trim axis left]
                  \pgfplotsset{width=7cm, height=6.2cm}
                  \begin{axis}[grid=major, xlabel={Distance (km)}, ylabel={${I}^-_{im}$}, /pgf/number format/.cd, legend style={at={(0.98,0.15)},anchor=south east,legend columns=1, draw=none, inner sep=0pt,fill=gray!10}, xtick={0,20,...,100}, axis line style = thick, xmin=0, xmax=100,]
                  \addplot[very thick, black] table[x=x, y=y, meta=label, col sep=comma] {Data/dist/I2_im_LL.csv};
                  \addplot[very thick, red] table[x=x, y=y, meta=label, col sep=comma] {Data/dist/GI2_im_LL.csv};
                  \addplot[very thick, gray, dashed] table[x=x, y=y, meta=label, col sep=comma] {Data/dist/RI2_im_LL.csv};
                  % \legend{BF, OPT};
                  \end{axis}
              \end{tikzpicture}
        \end{subfigure}

        \vspace{0.2cm}

    \begin{center}
        \begin{subfigure}[!htb]{0.7\textwidth}
                \begin{tikzpicture}[trim axis right,trim axis left]
                    \pgfplotsset{width=12.5cm, height=6.2cm}
                    \begin{axis}[grid=major, xlabel={Distance (km)}, ylabel={$f$}, /pgf/number format/.cd, legend style={at={(0.98,0.2)},anchor=south east,legend columns=1, draw=none, inner sep=0pt,fill=gray!10}, xtick={0,10,...,100}, axis line style = thick, yticklabel style={/pgf/number format/fixed, /pgf/number format/precision=3}, xmin=0, xmax=100, ytick={0.81,0.82,...,0.87}]
                   \addplot[very thick, black] table[x=x, y=y, meta=label, col sep=comma] {Data/dist/ff_LL.csv};
                   \addplot[very thick, red] table[x=x, y=y, meta=label, col sep=comma] {Data/dist/Gff_LL.csv};
                   \addplot[very thick, gray, dashed] table[x=x, y=y, meta=label, col sep=comma] {Data/dist/Rff_LL.csv};
                    \legend{OPT, ROPT};
                    \end{axis}
                \end{tikzpicture}
        \end{subfigure}
    \end{center}
    \caption{Influence of the currents on the objective function for the line to line fault with $\underline{Z}_f=0.03$ and a submarine cable. OPT: solution to the optimization problem, ROPT: solution to the optimization problem restricted to only injecting reactive power.}
    \label{fig:LLx1_s}
  \end{figure}
The line to line fault has been found to become the one where the current values are again more intuitive. That is, in the OPT situation, the real positive sequence current becomes positive while the imaginary positive sequence one is negative to provoke a positive voltage drop as well to maximize the positive sequence voltage at the PCC. The inverse reasoning applies to the negative sequence currents. When we limit the reactive current to zero, as in the ROPT case and in the Grid Code, not only the differences between the two active currents is acute, but they are also noticeable for the imaginary currents. The negative sequence real current is specially considerable, but despite that, the objective function is not largely affected if we constrain it. 

In general, the currents do not experience huge variations across all distances. At least, such variations are in line with the ones we have shown for the submarine cable. Changes were more acute for the $R/X$ varying ratio. Here the trend is to increase the reactive positive sequence current (in absolute terms) for the OPT and the ROPT cases. For the reactive negative sequence current, the contrary reasoning applies; the currents decrease. Nevertheless, the current profiles for the Grid Code are contrary to the OPT and the ROPT. In this sense, the negative sequence imaginary current is prioritized over the positive one. The differences in the objective function are small, around 1\%, despite the variations in current.

Figure \ref{fig:full_dist_LL} displays the positive and negative sequence voltages for the considered range of distances. 

\begin{figure}[!htb]\centering \footnotesize
    \begin{subfigure}[!htb]{.4\textwidth}
      \centering
          \begin{tikzpicture}[trim axis right,trim axis left]
              \pgfplotsset{width=7cm, height=7.0cm}
              \begin{axis}[grid=major, xlabel={Distance (km)}, ylabel={$|V^+|$}, /pgf/number format/.cd, legend style={at={(0.98,0.15)},anchor=south east,legend columns=1, draw=none, inner sep=0pt,fill=gray!10}, xtick={0,20,...,100}, axis line style = thick, ytick distance={0.02}, yticklabel style={/pgf/number format/fixed, /pgf/number format/precision=2}, xmin=0, xmax=100]
              \addplot[very thick, black] table[x=x, y=y, meta=label, col sep=comma] {Data/dist/V1_LL.csv};
              \addplot[very thick, red] table[x=x, y=y, meta=label, col sep=comma] {Data/dist/GV1_LL.csv};
              \addplot[very thick, gray, dashed] table[x=x, y=y, meta=label, col sep=comma] {Data/dist/RV1_LL.csv};
              % \legend{BF, OPT};
              \end{axis}
          \end{tikzpicture}
    \end{subfigure}
    \hspace{1.5cm}
    \begin{subfigure}[!htb]{.4\textwidth}
        \centering
            \begin{tikzpicture}[trim axis right,trim axis left]
                \pgfplotsset{width=7cm, height=7.0cm}
                \begin{axis}[grid=major, xlabel={Distance (km)}, ylabel={$|{V}^-|$}, /pgf/number format/.cd, legend style={at={(0.98,0.03)},anchor=south east,legend columns=1, draw=none, inner sep=0pt,fill=gray!10}, xtick={0, 20,...,100}, axis line style = thick, xmin=0, xmax=100, ytick distance={0.02}, yticklabel style={/pgf/number format/fixed, /pgf/number format/precision=3}]
                \addplot[very thick, black] table[x=x, y=y, meta=label, col sep=comma] {Data/dist/V2_LL.csv};
                \addplot[very thick, red] table[x=x, y=y, meta=label, col sep=comma] {Data/dist/GV2_LL.csv};
                \addplot[very thick, gray, dashed] table[x=x, y=y, meta=label, col sep=comma] {Data/dist/RV2_LL.csv};
                \legend{OPT, Grid Code, ROPT};
                \end{axis}
            \end{tikzpicture}
      \end{subfigure}
    \caption{Influence of the $R/X$ ratio on the voltages for the line to line fault with $\underline{Z}_f=0.03$. OPT: solution to the optimization problem, ROPT: solution to the optimization problem restricted to only injecting reactive power.}
    \label{fig:full_dist_LL}
  \end{figure}


Surprisingly, the positive sequence voltage is larger for the ROPT case than in the OPT situation, for all ranges of distance. This could already be anticipated by looking at Figure \ref{fig:LLx1_s}, where the positive sequence OPT currents are inferior when compared to the imaginary positive sequence ROPT current. Therefore, the ROPT case makes a bigger effort to maximize the positive sequence voltage at the expense of causing an also larger negative sequence voltage. If we balance both, we get to the conclusion that the objective function is slightly better off in the OPT case, just like we could anticipate. All the absolute values of the voltages tend to increase for longer distances. Compared to the variation in the $R/X$ ratio, the objective function now remains almost always the same. However, the fault is more severe and the differences between the OPT and the OPT scenarios have diminished.

Regarding the Grid Code, its positive sequence voltage is larger than the OPT and the ROPT ones for short distances. However, the negative sequence also is. Even though the distribution of voltages is apparently different from the ROPT scenario, their objective functions match across all distances. Only for lengths close to 100 km, the ROPT situation is better off, mainly due to its positive sequence voltage. 


\subsection{Double line to ground fault}
Figure \ref{fig:LLGx1_s} depicts the optimal currents for the double line to ground.

\begin{figure}[!htb]\centering \footnotesize
    \begin{subfigure}[!htb]{.4\textwidth}
      \centering
          \begin{tikzpicture}[trim axis right,trim axis left]
              \pgfplotsset{width=7cm, height=6.1cm}
              \begin{axis}[grid=major, xlabel={Distance (km)}, ylabel={${I}^+_{re}$}, /pgf/number format/.cd, legend style={at={(0.98,0.15)},anchor=south east,legend columns=1, draw=none, inner sep=0pt,fill=gray!10}, xtick={0,20,...,100}, axis line style = thick, yticklabel style={/pgf/number format/fixed, /pgf/number format/precision=3}, xmin=0, xmax=100, ytick distance = {0.025}]
              \addplot[very thick, black] table[x=x, y=y, meta=label, col sep=comma] {Data/dist/I1_re_LLG.csv};
              \addplot[very thick, red] table[x=x, y=y, meta=label, col sep=comma] {Data/dist/GI1_re_LLG.csv};
              \addplot[very thick, gray, dashed] table[x=x, y=y, meta=label, col sep=comma] {Data/dist/RI1_re_LLG.csv};
              % \legend{BF, OPT};
              \end{axis}
          \end{tikzpicture}
    \end{subfigure}
    \hspace{1.5cm}
    \begin{subfigure}[!htb]{.4\textwidth}
        \centering
            \begin{tikzpicture}[trim axis right,trim axis left]
                \pgfplotsset{width=7cm, height=6.1cm}
                \begin{axis}[grid=major, xlabel={Distance (km)}, ylabel={${I}^+_{im}$}, /pgf/number format/.cd, legend style={at={(0.98,0.15)},anchor=south east,legend columns=1, draw=none, inner sep=0pt,fill=gray!10}, xtick={0,20,...,100}, axis line style = thick, yticklabel style={/pgf/number format/fixed, /pgf/number format/precision=4}, xmin=0, xmax=100, ytick distance = {0.025}]
                \addplot[very thick, black] table[x=x, y=y, meta=label, col sep=comma] {Data/dist/I1_im_LLG.csv};
                \addplot[very thick, red] table[x=x, y=y, meta=label, col sep=comma] {Data/dist/GI1_im_LLG.csv};
                \addplot[very thick, gray, dashed] table[x=x, y=y, meta=label, col sep=comma] {Data/dist/RI1_im_LLG.csv};
                % \legend{OPT};
                \end{axis}
            \end{tikzpicture}
      \end{subfigure}

      \vspace{0.5cm}

      \begin{subfigure}[!htb]{.4\textwidth}
        \centering
            \begin{tikzpicture}[trim axis right,trim axis left]
                \pgfplotsset{width=7cm, height=6.1cm}
                \begin{axis}[grid=major, xlabel={Distance (km)}, ylabel={${I}^-_{re}$}, /pgf/number format/.cd, legend style={at={(0.98,0.15)},anchor=south east,legend columns=1, draw=none, inner sep=0pt,fill=gray!10}, xtick={0,20,...,100}, axis line style = thick, xmin=0, xmax=100, yticklabel style={/pgf/number format/fixed, /pgf/number format/precision=4}, ytick distance = {0.025}]
                \addplot[very thick, black] table[x=x, y=y, meta=label, col sep=comma] {Data/dist/I2_re_LLG.csv};
                \addplot[very thick, red] table[x=x, y=y, meta=label, col sep=comma] {Data/dist/GI2_re_LLG.csv};
                \addplot[very thick, gray, dashed] table[x=x, y=y, meta=label, col sep=comma] {Data/dist/RI2_re_LLG.csv};
                % \legend{BF, OPT};
                \end{axis}
            \end{tikzpicture}
      \end{subfigure}
      \hspace{1.5cm}
      \begin{subfigure}[!htb]{.4\textwidth}
          \centering
              \begin{tikzpicture}[trim axis right,trim axis left]
                  \pgfplotsset{width=7cm, height=6.1cm}
                  \begin{axis}[grid=major, xlabel={Distance (km)}, ylabel={${I}^-_{im}$}, /pgf/number format/.cd, legend style={at={(0.98,0.15)},anchor=south east,legend columns=1, draw=none, inner sep=0pt,fill=gray!10}, xtick={0,20,...,100}, axis line style = thick, xmin=0, xmax=100, yticklabel style={/pgf/number format/fixed, /pgf/number format/precision=3}]
                  \addplot[very thick, black] table[x=x, y=y, meta=label, col sep=comma] {Data/dist/I2_im_LLG.csv};
                  \addplot[very thick, red] table[x=x, y=y, meta=label, col sep=comma] {Data/dist/GI2_im_LLG.csv};
                  \addplot[very thick, gray, dashed] table[x=x, y=y, meta=label, col sep=comma] {Data/dist/RI2_im_LLG.csv};
                  % \legend{BF, OPT};
                  \end{axis}
              \end{tikzpicture}
        \end{subfigure}

        \vspace{0.2cm}

    \begin{center}
        \begin{subfigure}[!htb]{0.7\textwidth}
                \begin{tikzpicture}[trim axis right,trim axis left]
                    \pgfplotsset{width=12.5cm, height=6.1cm}
                    \begin{axis}[grid=major, xlabel={Distance (km)}, ylabel={$f$}, /pgf/number format/.cd, legend style={at={(0.98,0.2)},anchor=south east,legend columns=1, draw=none, inner sep=0pt,fill=gray!10}, xtick={0,10,...,100}, axis line style = thick, yticklabel style={/pgf/number format/fixed, /pgf/number format/precision=4}, xmin=0, xmax=100, ytick distance = {0.0005}]
                   \addplot[very thick, black] table[x=x, y=y, meta=label, col sep=comma] {Data/dist/ff_LLG.csv};
                   \addplot[very thick, red] table[x=x, y=y, meta=label, col sep=comma] {Data/dist/Gff_LLG.csv};
                   \addplot[very thick, gray, dashed] table[x=x, y=y, meta=label, col sep=comma] {Data/dist/Rff_LLG.csv};
                    \legend{OPT, Grid Code, ROPT};
                    \end{axis}
                \end{tikzpicture}
        \end{subfigure}
    \end{center}
    \caption{Influence of the currents on the objective function for the double line to ground fault with $\underline{Z}_f=0.5$ and a submarine cable. OPT: solution to the optimization problem, ROPT: solution to the optimization problem restricted to only injecting reactive power.}
    \label{fig:LLGx1_s}
  \end{figure}
As discussed before, the double line to ground fault is the most severe one by nature. Just like in the varying $R/X$ ratio, the objective funciton $f$ is around 0.94. This time, however, the objective function for all distances remains more constant. 

The imaginary currents take values close to an order of magnitude higher than the real parts. From the analysis of all faults, it seems that the imaginary current is strongly linked to the severity of the fault. Consequently, when the fault happens to be a strong one, injecting or not active currents has a minor influence. As follows, the OPT and the ROPT cases yield very similar operating conditions. Overall the same could be said about the Grid Code, even though its imaginary currents do evolve a bit. This also causes the objective function to evolve; unfortunately, it grows.

Figure \ref{fig:full_dist_LLG} displays the positive and negative sequence voltages. 

\begin{figure}[!htb]\centering \footnotesize
    \begin{subfigure}[!htb]{.4\textwidth}
      \centering
          \begin{tikzpicture}[trim axis right,trim axis left]
              \pgfplotsset{width=7cm, height=7.0cm}
              \begin{axis}[grid=major, xlabel={Distance (km)}, ylabel={$|V^+|$}, /pgf/number format/.cd, legend style={at={(0.98,0.15)},anchor=south east,legend columns=1, draw=none, inner sep=0pt,fill=gray!10}, xtick={0,20,...,100}, axis line style = thick, ytick distance={0.02}, yticklabel style={/pgf/number format/fixed, /pgf/number format/precision=2}, xmin=0, xmax=100]
              \addplot[very thick, black] table[x=x, y=y, meta=label, col sep=comma] {Data/dist/V1_LLG.csv};
              \addplot[very thick, red] table[x=x, y=y, meta=label, col sep=comma] {Data/dist/GV1_LLG.csv};
              \addplot[very thick, gray, dashed] table[x=x, y=y, meta=label, col sep=comma] {Data/dist/RV1_LLG.csv};
              % \legend{BF, OPT};
              \end{axis}
          \end{tikzpicture}
    \end{subfigure}
    \hspace{1.5cm}
    \begin{subfigure}[!htb]{.4\textwidth}
        \centering
            \begin{tikzpicture}[trim axis right,trim axis left]
                \pgfplotsset{width=7cm, height=7.0cm}
                \begin{axis}[grid=major, xlabel={Distance (km)}, ylabel={$|{V}^-|$}, /pgf/number format/.cd, legend style={at={(0.98,0.03)},anchor=south east,legend columns=1, draw=none, inner sep=0pt,fill=gray!10}, xtick={0, 20,...,100}, axis line style = thick, xmin=0, xmax=100, ytick distance={0.02}, yticklabel style={/pgf/number format/fixed, /pgf/number format/precision=3}]
                \addplot[very thick, black] table[x=x, y=y, meta=label, col sep=comma] {Data/dist/V2_LLG.csv};
                \addplot[very thick, red] table[x=x, y=y, meta=label, col sep=comma] {Data/dist/GV2_LLG.csv};
                \addplot[very thick, gray, dashed] table[x=x, y=y, meta=label, col sep=comma] {Data/dist/RV2_LLG.csv};
                \legend{OPT, Grid Code, ROPT};
                \end{axis}
            \end{tikzpicture}
      \end{subfigure}
    \caption{Influence of the $R/X$ ratio on the voltages for the double line to ground fault with $\underline{Z}_f=0.5$. OPT: solution to the optimization problem, ROPT: solution to the optimization problem restricted to only injecting reactive power.}
    \label{fig:full_dist_LLG}
  \end{figure}

Again, differences are hardly noticeable between both cases. The lines are almost overlapped for all distances. In spite of that, the trend is to obtain increasing voltages. Incrementing the positive sequence voltage means that we are improving the objective function, but in the meantime, the negative sequence voltage also grows. Balancing both contributions results in obtaining an almost constant objective function for all distances. 



%%%%%%%%%%%%---------------------//////////////////$$$$$$$$$$$$$$$$$$$$



% \subsection{Line to ground fault}
% Figure \ref{fig:LGx1_s} depicts the optimal currents for the line to ground fault.
% \begin{figure}[!htb]\centering \footnotesize
%     \begin{subfigure}[!htb]{.4\textwidth}
%       \centering
%           \begin{tikzpicture}[trim axis right,trim axis left]
%               \pgfplotsset{width=7cm, height=5.9cm}
%               \begin{axis}[grid=major, xlabel={$R/X$}, ylabel={${I}^+_{re}$}, /pgf/number format/.cd, legend style={at={(0.98,0.15)},anchor=south east,legend columns=1, draw=none, inner sep=0pt,fill=gray!10}, xtick={0,20,...,100}, ytick={0,0.02,...,0.1}, axis line style = very thick, yticklabel style={/pgf/number format/fixed, /pgf/number format/precision=2}]
%               \addplot[very thick, black] table[x=x, y=y, meta=label, col sep=comma] {Data/submarine/I1_re_LG.csv};
%               \addplot[very thick, gray] table[x=x, y=y, meta=label, col sep=comma] {Data/submarine/RI1_re_LG.csv};
%               % \legend{BF, OPT};
%               \end{axis}
%           \end{tikzpicture}
%     \end{subfigure}
%     \hspace{1.5cm}
%     \begin{subfigure}[!htb]{.4\textwidth}
%         \centering
%             \begin{tikzpicture}[trim axis right,trim axis left]
%                 \pgfplotsset{width=7cm, height=5.9cm}
%                 \begin{axis}[grid=major, xlabel={$R/X$}, ylabel={${I}^+_{im}$}, /pgf/number format/.cd, legend style={at={(0.98,0.15)},anchor=south east,legend columns=1, draw=none, inner sep=0pt,fill=gray!10}, xtick={0,20,...,100}, ytick={-0.8,-0.7,...,-0.3}, axis line style = very thick]
%                 \addplot[very thick, black] table[x=x, y=y, meta=label, col sep=comma] {Data/submarine/I1_im_LG.csv};
%                 \addplot[very thick, gray] table[x=x, y=y, meta=label, col sep=comma] {Data/submarine/RI1_im_LG.csv};
%                 % \legend{OPT};
%                 \end{axis}
%             \end{tikzpicture}
%       \end{subfigure}

%       \vspace{0.5cm}

%       \begin{subfigure}[!htb]{.4\textwidth}
%         \centering
%             \begin{tikzpicture}[trim axis right,trim axis left]
%                 \pgfplotsset{width=7cm, height=5.9cm}
%                 \begin{axis}[grid=major, xlabel={$R/X$}, ylabel={${I}^-_{re}$}, /pgf/number format/.cd, legend style={at={(0.98,0.15)},anchor=south east,legend columns=1, draw=none, inner sep=0pt,fill=gray!10}, xtick={0,20,...,100}, axis line style = very thick, scaled y ticks=false, yticklabel style={/pgf/number format/fixed,/pgf/number format/precision=2}]
%                 \addplot[very thick, black] table[x=x, y=y, meta=label, col sep=comma] {Data/submarine/I2_re_LG.csv};
%                 \addplot[very thick, gray] table[x=x, y=y, meta=label, col sep=comma] {Data/submarine/RI2_re_LG.csv};
%                 % \legend{BF, OPT};
%                 \end{axis}
%             \end{tikzpicture}
%       \end{subfigure}
%       \hspace{1.5cm}
%       \begin{subfigure}[!htb]{.4\textwidth}
%           \centering
%               \begin{tikzpicture}[trim axis right,trim axis left]
%                   \pgfplotsset{width=7cm, height=5.9cm}
%                   \begin{axis}[grid=major, xlabel={$R/X$}, ylabel={${I}^-_{im}$}, /pgf/number format/.cd, legend style={at={(0.98,0.15)},anchor=south east,legend columns=1, draw=none, inner sep=0pt,fill=gray!10}, xtick={0,20,...,100}, ytick={-0.7,-0.6,...,-0.2}, axis line style = very thick]
%                   \addplot[very thick, black] table[x=x, y=y, meta=label, col sep=comma] {Data/submarine/I2_im_LG.csv};
%                   \addplot[very thick, gray] table[x=x, y=y, meta=label, col sep=comma] {Data/submarine/RI2_im_LG.csv};
%                   % \legend{BF, OPT};
%                   \end{axis}
%               \end{tikzpicture}
%         \end{subfigure}

%         \vspace{0.2cm}

%     \begin{center}
%         \begin{subfigure}[!htb]{0.7\textwidth}
%                 \begin{tikzpicture}[trim axis right,trim axis left]
%                     \pgfplotsset{width=12.5cm, height=5.9cm}
%                     \begin{axis}[grid=major, xlabel={$R/X$}, ylabel={$f$}, /pgf/number format/.cd, legend style={at={(0.98,0.2)},anchor=south east,legend columns=1, draw=none, inner sep=0pt,fill=gray!10}, xtick={0,10,...,100}, ytick={0.13,0.14,...,0.17}, axis line style = very thick, yticklabel style={/pgf/number format/fixed, /pgf/number format/precision=3}]
%                    \addplot[very thick, black] table[x=x, y=y, meta=label, col sep=comma] {Data/submarine/ff_LG.csv};
%                    \addplot[very thick, gray] table[x=x, y=y, meta=label, col sep=comma] {Data/submarine/Rff_LG.csv};
%                     \legend{OPT, ROPT};
%                     \end{axis}
%                 \end{tikzpicture}
%         \end{subfigure}
%     \end{center}
%     \caption{Influence of the currents on the objective function for the line to ground fault and a submarine cable. OPT: solution to the optimization problem, ROPT: solution to the optimization problem restricted to only injecting reactive power.}
%     \label{fig:LGx1_s}
%   \end{figure}

% \subsection{Line to line fault}
% Figure \ref{fig:LLx1_s} depicts the optimal currents for the line to line fault.

% \begin{figure}[!htb]\centering \footnotesize
%     \begin{subfigure}[!htb]{.4\textwidth}
%       \centering
%           \begin{tikzpicture}[trim axis right,trim axis left]
%               \pgfplotsset{width=7cm, height=6.0cm}
%               \begin{axis}[grid=major, xlabel={$X_c$}, ylabel={${I}^+_{re}$}, /pgf/number format/.cd, legend style={at={(0.98,0.15)},anchor=south east,legend columns=1, draw=none, inner sep=0pt,fill=gray!10}, xtick={0,20,...,100}, ymax = 0.2, ymin=-0.2, axis line style = very thick, scaled y ticks=false, yticklabel style={/pgf/number format/fixed, /pgf/number format/precision=2}]
%               \addplot[very thick, black] table[x=x, y=y, meta=label, col sep=comma] {Data/submarine/I1_re_LL.csv};
%               \addplot[very thick, gray] table[x=x, y=y, meta=label, col sep=comma] {Data/submarine/RI1_re_LL.csv};
%               % \legend{BF, OPT};
%               \end{axis}
%           \end{tikzpicture}
%     \end{subfigure}
%     \hspace{1.5cm}
%     \begin{subfigure}[!htb]{.4\textwidth}
%         \centering
%             \begin{tikzpicture}[trim axis right,trim axis left]
%                 \pgfplotsset{width=7cm, height=6.0cm}
%                 \begin{axis}[grid=major, xlabel={$X_c$}, ylabel={${I}^+_{im}$}, /pgf/number format/.cd, legend style={at={(0.98,0.15)},anchor=south east,legend columns=1, draw=none, inner sep=0pt,fill=gray!10}, xtick={0,20,...,100}, ytick={-1,-0.75,...,0}, axis line style = very thick]
%                 \addplot[very thick, black] table[x=x, y=y, meta=label, col sep=comma] {Data/submarine/I1_im_LL.csv};
%                 \addplot[very thick, gray] table[x=x, y=y, meta=label, col sep=comma] {Data/submarine/RI1_im_LL.csv};
%                 % \legend{OPT};
%                 \end{axis}
%             \end{tikzpicture}
%       \end{subfigure}

%       \vspace{0.5cm}

%       \begin{subfigure}[!htb]{.4\textwidth}
%         \centering
%             \begin{tikzpicture}[trim axis right,trim axis left]
%                 \pgfplotsset{width=7cm, height=6.0cm}
%                 \begin{axis}[grid=major, xlabel={$X_c$}, ylabel={${I}^-_{re}$}, /pgf/number format/.cd, legend style={at={(0.98,0.15)},anchor=south east,legend columns=1, draw=none, inner sep=0pt,fill=gray!10}, xtick={0,20,...,100}, ytick={-1,-0.75,...,0}, axis line style = very thick]
%                 \addplot[very thick, black] table[x=x, y=y, meta=label, col sep=comma] {Data/submarine/I2_re_LL.csv};
%                 \addplot[very thick, gray] table[x=x, y=y, meta=label, col sep=comma] {Data/submarine/RI2_re_LL.csv};
%                 % \legend{BF, OPT};
%                 \end{axis}
%             \end{tikzpicture}
%       \end{subfigure}
%       \hspace{1.5cm}
%       \begin{subfigure}[!htb]{.4\textwidth}
%           \centering
%               \begin{tikzpicture}[trim axis right,trim axis left]
%                   \pgfplotsset{width=7cm, height=6.0cm}
%                   \begin{axis}[grid=major, xlabel={$X_c$}, ylabel={${I}^-_{im}$}, /pgf/number format/.cd, legend style={at={(0.98,0.15)},anchor=south east,legend columns=1, draw=none, inner sep=0pt,fill=gray!10}, xtick={0,20,...,100}, axis line style = very thick, yticklabel style={/pgf/number format/fixed, /pgf/number format/precision=2}]
%                   \addplot[very thick, black] table[x=x, y=y, meta=label, col sep=comma] {Data/submarine/I2_im_LL.csv};
%                   \addplot[very thick, gray] table[x=x, y=y, meta=label, col sep=comma] {Data/submarine/RI2_im_LL.csv};
%                   % \legend{BF, OPT};
%                   \end{axis}
%               \end{tikzpicture}
%         \end{subfigure}

%         \vspace{0.2cm}

%     \begin{center}
%         \begin{subfigure}[!htb]{0.7\textwidth}
%                 \begin{tikzpicture}[trim axis right,trim axis left]
%                     \pgfplotsset{width=12.5cm, height=6cm}
%                     \begin{axis}[grid=major, xlabel={$X_c$}, ylabel={$f$}, /pgf/number format/.cd, legend style={at={(0.98,0.5)},anchor=south east,legend columns=1, draw=none, inner sep=0pt,fill=gray!10}, xtick={0,10,...,100}, axis line style = very thick, yticklabel style={/pgf/number format/fixed, /pgf/number format/precision=2}]
%                    \addplot[very thick, black] table[x=x, y=y, meta=label, col sep=comma] {Data/submarine/ff_LL.csv};
%                    \addplot[very thick, gray] table[x=x, y=y, meta=label, col sep=comma] {Data/submarine/Rff_LL.csv};
%                     \legend{OPT, ROPT};
%                     \end{axis}
%                 \end{tikzpicture}
%         \end{subfigure}
%     \end{center}
%     \caption{Influence of the currents on the objective function for the line to line fault and a submarine cable. OPT: solution to the optimization problem, ROPT: solution to the optimization problem restricted to only injecting reactive power.}
%     \label{fig:LLx1_s}
%   \end{figure}

% \subsection{Double line to ground fault}
% Figure \ref{fig:LLGx1_s} depicts the optimal currents for the balanced fault.

% \begin{figure}[!htb]\centering \footnotesize
%     \begin{subfigure}[!htb]{.4\textwidth}
%       \centering
%           \begin{tikzpicture}[trim axis right,trim axis left]
%               \pgfplotsset{width=7cm, height=6.0cm}
%               \begin{axis}[grid=major, xlabel={$X_c$}, ylabel={${I}^+_{re}$}, /pgf/number format/.cd, legend style={at={(0.98,0.15)},anchor=south east,legend columns=1, draw=none, inner sep=0pt,fill=gray!10},  xtick={0,20,...,100}, ytick={-0.01,-0.0075,...,0}, axis line style = very thick, yticklabel style={/pgf/number format/fixed, /pgf/number format/precision=4}, scaled y ticks=false]
%               \addplot[very thick, black] table[x=x, y=y, meta=label, col sep=comma] {Data/submarine/I1_re_LLG.csv};
%               \addplot[very thick, gray] table[x=x, y=y, meta=label, col sep=comma] {Data/submarine/RI1_re_LLG.csv};
%               % \legend{BF, OPT};
%               \end{axis}
%           \end{tikzpicture}
%     \end{subfigure}
%     \hspace{1.5cm}
%     \begin{subfigure}[!htb]{.4\textwidth}
%         \centering
%             \begin{tikzpicture}[trim axis right,trim axis left]
%                 \pgfplotsset{width=7cm, height=6.0cm}
%                 \begin{axis}[grid=major, xlabel={$X_c$}, ylabel={${I}^+_{im}$}, /pgf/number format/.cd, legend style={at={(0.98,0.15)},anchor=south east,legend columns=1, draw=none, inner sep=0pt,fill=gray!10}, xtick={0,20,...,100}, ytick={-1,-0.75,...,0}, yticklabel style={/pgf/number format/fixed, /pgf/number format/precision=5}, axis line style = very thick]
%                 \addplot[very thick, black] table[x=x, y=y, meta=label, col sep=comma] {Data/submarine/I1_im_LLG.csv};
%                 \addplot[very thick, gray] table[x=x, y=y, meta=label, col sep=comma] {Data/submarine/RI1_im_LLG.csv};
%                 % \legend{OPT};
%                 \end{axis}
%             \end{tikzpicture}
%       \end{subfigure}

%       \vspace{0.5cm}

%       \begin{subfigure}[!htb]{.4\textwidth}
%         \centering
%             \begin{tikzpicture}[trim axis right,trim axis left]
%                 \pgfplotsset{width=7cm, height=6.0cm}
%                 \begin{axis}[grid=major, xlabel={$X_c$}, ylabel={${I}^-_{re}$}, /pgf/number format/.cd, legend style={at={(0.98,0.15)},anchor=south east,legend columns=1, draw=none, inner sep=0pt,fill=gray!10}, xtick={0,20,...,100}, ytick={-1,-0.75,...,0}, axis line style = very thick, yticklabel style={/pgf/number format/fixed, /pgf/number format/precision=5}, scaled y ticks=false]
%                 \addplot[very thick, black] table[x=x, y=y, meta=label, col sep=comma] {Data/submarine/I2_re_LLG.csv};
%                 \addplot[very thick, gray] table[x=x, y=y, meta=label, col sep=comma] {Data/submarine/RI2_re_LLG.csv};
%                 % \legend{BF, OPT};
%                 \end{axis}
%             \end{tikzpicture}
%       \end{subfigure}
%       \hspace{1.5cm}
%       \begin{subfigure}[!htb]{.4\textwidth}
%           \centering
%               \begin{tikzpicture}[trim axis right,trim axis left]
%                   \pgfplotsset{width=7cm, height=6.0cm}
%                   \begin{axis}[grid=major, xlabel={$X_c$}, ylabel={${I}^-_{im}$}, /pgf/number format/.cd, legend style={at={(0.98,0.15)},anchor=south east,legend columns=1, draw=none, inner sep=0pt,fill=gray!10}, xtick={0,20,...,100}, ytick={0.01,0.04,...,0.16}, axis line style = very thick, yticklabel style={/pgf/number format/fixed, /pgf/number format/precision=2}, scaled y ticks=false]
%                   \addplot[very thick, black] table[x=x, y=y, meta=label, col sep=comma] {Data/submarine/I2_im_LLG.csv};
%                   \addplot[very thick, gray] table[x=x, y=y, meta=label, col sep=comma] {Data/submarine/RI2_im_LLG.csv};
%                   % \legend{BF, OPT};
%                   \end{axis}
%               \end{tikzpicture}
%         \end{subfigure}

%         \vspace{0.2cm}

%     \begin{center}
%         \begin{subfigure}[!htb]{0.7\textwidth}
%                 \begin{tikzpicture}[trim axis right,trim axis left]
%                     \pgfplotsset{width=12.5cm, height=6cm}
%                     \begin{axis}[grid=major, xlabel={$X_c$}, ylabel={$f$}, /pgf/number format/.cd, legend style={at={(0.98,0.2)},anchor=south east,legend columns=1, draw=none, inner sep=0pt,fill=gray!10}, xtick={0,10,...,100}, ytick={0.2,0.21,...,0.24}, yticklabel style={/pgf/number format/fixed, /pgf/number format/precision=5}, axis line style = very thick]
%                    \addplot[very thick, black] table[x=x, y=y, meta=label, col sep=comma] {Data/submarine/ff_LLG.csv};
%                    \addplot[very thick, gray] table[x=x, y=y, meta=label, col sep=comma] {Data/submarine/Rff_LLG.csv};
%                     \legend{OPT, ROPT};
%                     \end{axis}
%                 \end{tikzpicture}
%         \end{subfigure}
%     \end{center}
%     \caption{Influence of the currents on the objective function for the double line to ground fault and a submarine cable. OPT: solution to the optimization problem, ROPT: solution to the optimization problem restricted to only injecting reactive power.}
%     \label{fig:LLGx1_s}
%   \end{figure}
% The plots for the submarine cable indicate that while the distribution of currents changes substantially when considering the restriction in the active current, the objective functions tend to take similar values. 

% For instance, in the balanced fault, the best strategy in the ROPT case is to inject a maximum imaginary positive sequence current. On the contrary, the imaginary positive sequence current does not reach the limits of one. Instead, some part of the current is dedicated to the real part. As in the $R/X$ case of study, the negative sequence currents remain null for the full sweep. One can check that the objective function for a small $R/X$ ratio seems to coincide with the function when $X_c$ takes a large value. Again, the OPT scenario is slightly better than the ROPT one. 

% In the line to ground fault the objective functions is not far apart from the results from Figure \ref{fig:LGx1_c}. However, this time we have not obtained a discontinuous profile in the evolution of the currents. This shows that maybe in this case there is only a single minimum, or also, that the optimal values follow the same trajectory due to the initialization. Oddly enough, in the OPT situation the imaginary positive sequence current takes rather small values. The real negative sequence current becomes predominant. It is shocking that despite the enormous differences in the distribution of currents, the objective functions are not far apart one from the other. 

% The results for the line to line fault together with the ones from the double line to ground fault seem to be the most dubious. First, in the line to line fault the objective functions are somewhat larger than what it could be expected from Figure \ref{fig:LLx1_c}. In any case, while the real positive sequence and the imaginary negative sequence currents take extremely similar values, the differences are acute for the remaining two currents. The OPT case prioritizes the real negative sequence current whereas the ROPT opts for the imaginary positive sequence current. 

% No more intuitively sound seem to be the double line to ground fault values. The objective function becomes considerably smaller than in the case of the $R/X$ analysis, and again, the distribution of currents reminds of the one for the line to line fault. This could be expected. However, the extreme differences are hardly justifiable. 
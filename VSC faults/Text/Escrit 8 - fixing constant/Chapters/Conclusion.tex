\section{Conclusion}
This study has covered the analysis of a simple system to determine the optimal currents that ought to be injected to raise the positive sequence voltage and decrease the negative sequence voltage in case of a fault. Four types of faults have been considered, and in each case, two situations have been proposed: one in which the converter can inject real and reactive currents, and another where the converter can only inject reactive currents. Such currents are constrained to not surpass the current limitations of the converter. The results show that injecting only reactive powers, as could be imposed by the grid codes, is likely to be a near-optimal strategy. In all cases (constant impedance, changing $R/X$ ratio and changing the cable distance) the objective function related to the ROPT situation has been close to the OPT one. However, it is mandatory to determine the type of fault in order to properly deduce the optimal values of the injected currents, which can vary significantly.

In addition to that, this work proposes two methodologies to arrive to the optimal solution. One has consisted of computing combinations where currents can take a wide range of values. When the intervals are small enough, such a computationally intensive approach matches with the solution coming from solving directly the optimization problem. The latter becomes the preferred option as it is faster and more precise. Generally speaking the double line to ground fault is the hardest in terms of minimizing the objective function.

Varying the $R/X$ ratio shows that the optimal currents can be highly dependent on the ratio. However, the objective functions do not experience substantial differences, as the OPT case is always slightly better of than the ROPT one. The same conclusion can be extracted from the submarine cable under study. In both situations, it seems that the injection of real currents may dominate over the imaginary currents. We have also deduced that the fault impedance is usually the most determining factor. When it becomes small, the objective function has tiny room for improvement, while in case it takes large values, it remains close to zero.

\newpage
\printbibliography